\documentclass{cernatsnote} % Specifies the document style.
%
%
% This texfile is a modified version from 
% http://frs.home.cern.ch/frs/Source/NL_MD_02.07.2011/NL-MD.tex
%
%
%
\usepackage{vmargin,times,graphicx,amsmath,amssymb,color} % ,draftcopy��use draftcopy for experiments
\usepackage{verbatim} % to allow for verbatim and comment
\usepackage{lineno} % to allow for linenumbers in the draft version
\usepackage{tabularx}
\usepackage{multirow}


\newcommand{\lumi}[1]{\ensuremath{{\mathcal{L}= 10^{#1}\,\mathrm{cm}^{-2}\,\mathrm{s}^{-1}}}}
\newcommand{\lumiapprox}[1]{\ensuremath{{\mathcal{L} \approx 10^{#1}\,\mathrm{cm}^{-2}\,\mathrm{s}^{-1}}}}
\newcommand{\lumif}[2]{\ensuremath{{\mathcal{L}= {#1} \times 10^{#2}\,\mathrm{cm}^{-2}\,\mathrm{s}^{-1}}}}
\newcommand{\ifb}{\mbox{fb$^{-1}$}}%  Inverse femtobarns.
\newcommand{\ipb}{\mbox{pb$^{-1}$}}%  Inverse picobarns.
\newcommand{\inb}{\mbox{nb$^{-1}$}}%  Inverse nanobarns.
\newcommand{\sph}{\mbox{$\mu$Sv/h}}%  mu Sv/h
\newcommand{\nseu}{$N_{\mathrm{SEU}}$}
\newcommand{\hehf}{\ensuremath{\Phi_{\mathrm{HEH}}}}
\newcommand{\thnf}{\ensuremath{\Phi_{\mathrm{thn}}}}
\newcommand{\neqf}{\ensuremath{\Phi_{\mathrm{neq}}}}
\newcommand{\mypercent}{\%}

% now optionally modify manually to fill better the page (drop at end
% ?)
\setmarginsrb{22mm}{10mm}{20mm}{10mm}{12pt}{11mm}{0pt}{11mm}
%
\title{Beam Halo induced Background Simulation Studies at IR1 for final HL-LHC Collimation Layout}
%
\author{R.~Kwee, R.~Bruce, L.~S.~Esposito}
%
\date{\today}
\email{\small{Regina.Kwee@rhul.ac.uk}}
%
\makeindex
%
\documentlabel{CERN-ATS-Note-2015-XXX PERF}\keywords{HL-LHC, collimation, IR1, halo, TCT5, SixTrack, FLUKA}
%
% End of preamble and beginning of text.
\begin{document}
%
% Produces the title block.
\maketitle

\renewcommand{\textfraction}{0.1}       % the minimum fraction of a text page that must be devoted to text
\renewcommand{\floatpagefraction}{0.8}  % the minimum fraction of a float page that must be occupied by floats
\renewcommand{\topfraction}{1.}         % the maximum fraction float on top, usually .7

%
%\linenumbers
% ----------------------------------------------------------------------------------------------
\begin{abstract}
%
This note is contains the simulation results for beam halo induced background at IR1 in the HL-LHC scenario using the final HL-LHC collimation layout. SixTrack and Fluka simulations were performed to quantify the effect of the installation of additional tertiary collimators in cell 5 upstream the tertiary collimator pair that is already installed to protect the inner triplet magnets in IR1 and 5.
\end{abstract}
%
% ----------------------------------------------------------------------------------------------
\tableofcontents
\newpage
\newpage

% uncomment to only show section header
\addtocontents{toc}{\protect\setcounter{tocdepth}{1}}
%-------------------------------------------------------------------------------------------
\section{Introduction}


\begin{figure}
\begin{center}
\includegraphics[width=0.9\textwidth]{figures/coll_loss_H5_HL_TCT5LOUT_relaxColl_hHaloB1_flatthin_fullring}
\end{center}
\begin{picture} (0.,0.)
\setlength{\unitlength}{1.0cm}
\small{
    \put ( 12.3,1.05){(f)}
}
\end{picture}
\vspace{-1.3cm}
 \caption{some plots here
  \label{plots1}}
\end{figure}

%-------------------------------------------------------------------------------------------
\section*{Acknowledgments}
%
The authors of this note would like to thank Alessio Mereghetti EN/STI/EET for helpful discussions. 
%-------------------------------------------------------------------------------------------
\newpage
\clearpage
\appendix

\section{Appendix }

\subsection{Statistical Uncertainties of Risk Factor}

The statistical uncertainties are very low when integrated over a small volume as shown in Table~\ref{optTabPP} and~\ref{optTabBG}. However, if one is interested in the fluctuations of the HEH radiation field, integrated still over a rack height of 2~m, the uncertainties as shown in Fig.~\ref{stats} offer more details. This figure shows the statistical uncertainties for pp induced HEH fluence for each option (left colum) and for the r-factor calculation (right colum). The uncertainties for beam-gas induced fluences are all smaller than 3~\% (not shown).

\newpage

\begin{thebibliography}{99}
\bibitem{IgorsWork} I. Baishev, \textit{Radiation Levels in RR Areas LHC Point 1/5}, available on \href{https://espace.cern.ch/info-r2e-documents/Library/LHC_IR1_RRs_IB1404.pdf}{R2E page}, 2003.

\end{thebibliography}
%
\end{document}
%
