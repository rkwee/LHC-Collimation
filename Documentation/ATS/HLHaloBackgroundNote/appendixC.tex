\section{Additional plots for Run I, Run II and HL-LHC shower simulations\label{run1run2app}}


\begin{figure}[!htb]
\begin{center}

\includegraphics[width=0.49\textwidth]{figures/4TeV/haloB1_20MeV/RadNDist_BH_4TeV_B1_20MeV.pdf}
\includegraphics[width=0.49\textwidth]{figures/4TeV/haloB1_20MeV/RadEnDist_BH_4TeV_B1_20MeV.pdf}

\end{center}
\vspace{-0.6cm}
 \caption{Betatron halo B1: radial distributions and energy in $r$ at the interface plane in the 4 TeV scenario. 
  \label{dist4TeVB12}}
\end{figure}

\begin{figure}[!htb]
\begin{center}
%\includegraphics[width=0.4\textwidth]{figures/4TeV/compB1B2/perTCThit/ratioEkinMuons.pdf}
%\includegraphics[width=0.4\textwidth]{figures/4TeV/compB1B2/perTCThit/ratioPhiEnAll.pdf}
\includegraphics[width=0.46\textwidth]{figures/4TeV/compB1B2/perTCThit/ratioPhiEnMuons.pdf}
\includegraphics[width=0.46\textwidth]{figures/4TeV/compB1B2/perTCThit/ratioPhiEnProtons.pdf}
\end{center}
\vspace{-0.6cm}
\caption{Comparison of B1 and B2 betatron halo shower distributions at the interface plane. The numbers in the ratio plot is the ratio of both integrals of the top distributions to indicate by how much numerator or denominator is larger. The error bars indicate statistical uncertainties.
  \label{comp4TeVB1B2}}
\end{figure}


\begin{figure}%[!htb]
\begin{center}
\includegraphics[width=0.49\textwidth]{figures/4TeV/bs_20MeV/RadNDist_BG_4TeV_20MeV_bs.pdf}
\includegraphics[width=0.49\textwidth]{figures/4TeV/bs_20MeV/RadEnDist_BG_4TeV_20MeV_bs.pdf}
\end{center}
\vspace{-0.6cm}
 \caption{Beam-gas induced background shown for a flat pressure.
  \label{dist4TeVBGbs2}}
\end{figure}

% ------------------------------------
% comp Run1 

\begin{figure}
\begin{center}
  \includegraphics[width=0.41\textwidth]{figures/4TeV/compBG_3p5_vs_4TeV/perBGint_bs/ratioPhiEnAll.pdf}
  \includegraphics[width=0.41\textwidth]{figures/4TeV/compBG_3p5_vs_4TeV/perBGint_bs/ratioPhiEnPhotons.pdf}
  \includegraphics[width=0.41\textwidth]{figures/4TeV/compBG_3p5_vs_4TeV/perBGint_bs/ratioPhiEnMuons.pdf}
  \includegraphics[width=0.41\textwidth]{figures/4TeV/compBG_3p5_vs_4TeV/perBGint_bs/ratioPhiEnMuE100.pdf}
\end{center}
\vspace{-0.6cm}
 \caption{Run I beam-gas data: Same conclusions as from Fig.~\ref{xingCompBG}. Clear effect of crossing angle as also when comparing to the beam-gas data with beamsize (bs) (top) and no influence on muons (middle). Since the TCTs were at 11.8~$\sigma$ at 3.5~TeV and 9~$\sigma$ comparison of BG from before the TCTs may tell also about the influence of TCT settings on beam-gas. The bottom plots show there is no significant influence.
  \label{xingCompBG2}}
\end{figure}

% ------------------------------------
% crossing angle comparison
\begin{figure}
  \begin{center}
    \includegraphics[width=0.41\textwidth]{figures/4TeV/compB2_3p5vs4TeV/ratioPhiEnAll.pdf}
    \includegraphics[width=0.41\textwidth]{figures/4TeV/compB2_3p5vs4TeV/ratioPhiEnPhotons.pdf}
    %% \includegraphics[width=0.41\textwidth]{figures/4TeV/compB2_3p5vs4TeV/ratioPhiEnAll.pdf}
    %% \includegraphics[width=0.41\textwidth]{figures/4TeV/compB2_3p5vs4TeV/ratioPhiEnPhotons.pdf}
    %% \includegraphics[width=0.41\textwidth]{figures/4TeV/compB2_3p5vs4TeV/ratioPhiEnMuMinus.pdf}
    %% \includegraphics[width=0.41\textwidth]{figures/4TeV/compB2_3p5vs4TeV/ratioPhiEnMuPlus.pdf}
  \end{center}
  \vspace{-0.6cm}
  \caption{Betatron halo B2 comparing data at 3.5~TeV and 4~TeV at the interface plane: no clear sign of a crossing angle effect visible.
    \label{xingCompBHB2}}
\end{figure}

\begin{figure}%[!htb]
\begin{center}
  \includegraphics[width=0.411\textwidth]{figures/4TeV/beamsizeRatio/ratioPhiNProtonsE100.pdf}
  \includegraphics[width=0.411\textwidth]{figures/4TeV/beamsizeRatio/ratioPhiEnProtons.pdf}
  \includegraphics[width=0.411\textwidth]{figures/4TeV/beamsizeRatio/ratioPhiNMuonsE100.pdf}
  \includegraphics[width=0.411\textwidth]{figures/4TeV/beamsizeRatio/ratioPhiEnMuons.pdf}
\end{center}
\vspace{-0.6cm}
 \caption{Effect of beam size in 4 TeV beam-gas: azimuthal distributions of multiplicity (left) of high energy protons (top) and muons (bottom) and all proton and muon energies at the interface plane. 
  \label{bsRatioPhiMP}}
\end{figure}

\begin{figure}%[!htb]
\begin{center}
  \includegraphics[width=0.24\textwidth]{figures/4TeV/beamsizeRatio/ratioPhiEnPrZ1.pdf}
  \includegraphics[width=0.24\textwidth]{figures/4TeV/beamsizeRatio/ratioPhiEnPrZ2.pdf}
  \includegraphics[width=0.24\textwidth]{figures/4TeV/beamsizeRatio/ratioPhiEnPrZ3.pdf}
  \includegraphics[width=0.24\textwidth]{figures/4TeV/beamsizeRatio/ratioPhiEnPrZ4.pdf}
\end{center}
\vspace{-0.6cm}
 \caption{Azimuthal energy distributions of protons in different s-regions at the interface plane.
  \label{bsZPr}}
\end{figure}

\begin{figure}%[!htb]
\begin{center}
  \includegraphics[width=0.411\textwidth]{figures/4TeV/beamsizeRatio/ratioEkinAll.pdf}
  \includegraphics[width=0.411\textwidth]{figures/4TeV/beamsizeRatio/ratioEkinMuons.pdf}
  \includegraphics[width=0.411\textwidth]{figures/4TeV/beamsizeRatio/ratioEkinNeutrons.pdf}
  \includegraphics[width=0.411\textwidth]{figures/4TeV/beamsizeRatio/ratioEkinProtons.pdf}
\end{center}
\vspace{-0.6cm}
 \caption{Effect of beam size in 4 TeV beam-gas: kinetic energy of particles at s~=~22.6~m.
  \label{bsRatioEkin}}
\end{figure}

\begin{figure}%[!htb]
\begin{center}
  \includegraphics[width=0.411\textwidth]{figures/4TeV/beamsizeRatio/ratioRadNAll.pdf}
  \includegraphics[width=0.411\textwidth]{figures/4TeV/beamsizeRatio/ratioRadNMuons.pdf}
  \includegraphics[width=0.411\textwidth]{figures/4TeV/beamsizeRatio/ratioRadNNeutrons.pdf}
  \includegraphics[width=0.411\textwidth]{figures/4TeV/beamsizeRatio/ratioRadNProtons.pdf}
\end{center}
\vspace{-0.6cm}
 \caption{Effect of beam size in 4 TeV beam-gas: radius for different particle types at s~=~22.6~m.
  \label{bsRatioRadN}}
\end{figure}

\begin{figure}%[!htb]
\begin{center}
  \includegraphics[width=0.411\textwidth]{figures/4TeV/beamsizeRatio/ratioRadEnAll.pdf}
  \includegraphics[width=0.411\textwidth]{figures/4TeV/beamsizeRatio/ratioRadEnMuons.pdf}
  \includegraphics[width=0.411\textwidth]{figures/4TeV/beamsizeRatio/ratioRadEnNeutrons.pdf}
  \includegraphics[width=0.411\textwidth]{figures/4TeV/beamsizeRatio/ratioRadEnProtons.pdf}
\end{center}
\vspace{-0.6cm}
 \caption{Effect of beam size in 4 TeV beam-gas: energy in $r$ of different particle types.
  \label{bsRatioRadEn}}
\end{figure}


\newpage

% -----------------------------------------------------------------------------------------------------
% beamgas before and after reweightening


%% \begin{figure}
%%   \centering
%%   \includegraphics[width=0.45\textwidth]{figures/4TeV/compBGflat/ratioEkinAll.pdf}
%%   \includegraphics[width=0.45\textwidth]{figures/4TeV/compBGflat/ratioEkinMuons.pdf}
%%   \includegraphics[width=0.45\textwidth]{figures/4TeV/compBGflat/ratioPhiEnAll.pdf}
%%   \includegraphics[width=0.45\textwidth]{figures/4TeV/compBGflat/ratioPhiEnMuons.pdf}
%%   \caption{Same as Fig.~\ref{fig:cv81EkinPhiEn4TeV} but with ratio at the bottom. Note the color change.
%%     \label{fig:cv16EkinPhiEn4TeV}}
%% \end{figure}
   


\begin{figure}
  
\begin{center}
  \includegraphics[width=0.75\textwidth]{figures/4TeV/reweighted/cv81_OrigZMuon_BG_4TeV_20MeV_bs}
  \includegraphics[width=0.75\textwidth]{figures/4TeV/reweighted/cv81_OrigZPhotons_BG_4TeV_20MeV_bs}
  \includegraphics[width=0.75\textwidth]{figures/4TeV/reweighted/cv81_OrigZProtons_BG_4TeV_20MeV_bs}
\end{center}
\vspace{-0.6cm}
 \caption{Origin of muon (top), photon (middle) and proton (bottom) production along s, in black before and in pink after re-normalising to the 2012 pressure profile. 
  \label{fig:OrigZ4TeV2}}
\end{figure}

\begin{figure}
\begin{center}
   \includegraphics[width=0.45\textwidth]{figures/4TeV/compBGflat/ratioEkinAll}
  %% \includegraphics[width=0.45\textwidth]{figures/4TeV/compBGflat/ratioEkinMuons}
   \includegraphics[width=0.45\textwidth]{figures/4TeV/compBGflat/ratioPhiEnAll}
  %% \includegraphics[width=0.45\textwidth]{figures/4TeV/compBGflat/ratioPhiEnMuons}
  \includegraphics[width=0.45\textwidth]{figures/4TeV/compBGflat/ratioRadNAll}
%%  \includegraphics[width=0.45\textwidth]{figures/4TeV/compBGflat/ratioRadNMuons}
  \includegraphics[width=0.45\textwidth]{figures/4TeV/compBGflat/ratioRadEnAll}
%%  \includegraphics[width=0.45\textwidth]{figures/4TeV/compBGflat/ratioRadEnMuons}
\end{center}
\vspace{-0.6cm}
 \caption{Same type of distributions as in Fig.~\ref{fig:cv81EkinPhiEn4TeV}, showing them here for all particles. They follow the same trend of muons. the transverse radius $r$ (top) and the energy in $r$ (bottom). 
  \label{fig:cv81EkinPhiEn4TeV2}} 
\end{figure}


\begin{figure}
\begin{center}
  \includegraphics[width=0.45\textwidth]{figures/4TeV/compBGflat/ratioEkinProtons}
  \includegraphics[width=0.45\textwidth]{figures/4TeV/compBGflat/ratioEkinNeutrons}
  \includegraphics[width=0.45\textwidth]{figures/4TeV/compBGflat/ratioPhiEnProtons}
  \includegraphics[width=0.45\textwidth]{figures/4TeV/compBGflat/ratioPhiEnNeutrons}
  \includegraphics[width=0.45\textwidth]{figures/4TeV/compBGflat/ratioRadNProtons}
  \includegraphics[width=0.45\textwidth]{figures/4TeV/compBGflat/ratioRadNNeutrons}
  \includegraphics[width=0.45\textwidth]{figures/4TeV/compBGflat/ratioRadEnProtons}
  \includegraphics[width=0.45\textwidth]{figures/4TeV/compBGflat/ratioRadEnNeutrons}
\end{center}
\vspace{-0.6cm}
 \caption{Same observables as in Fig.~\ref{fig:cv81EkinPhiEn4TeV} but for protons (left) and neutrons (right): energy spectrum, energy in $\phi$, transverse radius $r$, and energy in $r$ before reweightening (black) and reweighted to pressure profile. The black curve are scaled up to better compare the shapes. Very similar shapes before and after reweightening.
   \label{fig:cv81ProtNeut4TeV}}
\end{figure}

% -----------------------------------------------------------------------------------------------------
% offmomentum


\begin{figure}
  \begin{center}
    \includegraphics[width=0.49\textwidth]{figures/4TeV/offmom/20MeV/Ekin_offplus500Hz_4TeV_B2_20MeV.pdf}
    \includegraphics[width=0.49\textwidth]{figures/4TeV/offmom/20MeV/PhiEnDist_offplus500Hz_4TeV_B2_20MeV.pdf}
  \includegraphics[width=0.49\textwidth]{figures/4TeV/offmom/20MeV/RadNDist_offplus500Hz_4TeV_B2_20MeV.pdf}
  \includegraphics[width=0.49\textwidth]{figures/4TeV/offmom/20MeV/RadEnDist_offplus500Hz_4TeV_B2_20MeV.pdf}

\end{center}
\vspace{-0.6cm}
 \caption{Off-momentum induced particle distributions.
  \label{offmom4TeV2}}
\end{figure}




\begin{figure}%[!htb]
\begin{center}
  \includegraphics[width=0.30\textwidth]{figures/4TeV/offmom/comppm500Hz/ratioEkinMuons.pdf}
  \includegraphics[width=0.30\textwidth]{figures/4TeV/offmom/comppm500Hz/ratioEkinProtons.pdf}
  \includegraphics[width=0.30\textwidth]{figures/4TeV/offmom/comppm500Hz/ratioEkinNeutrons.pdf}
  \includegraphics[width=0.30\textwidth]{figures/4TeV/offmom/comppm500Hz/ratioEkinPhotons.pdf}
  \includegraphics[width=0.30\textwidth]{figures/4TeV/offmom/comppm500Hz/ratioEkinElecPosi.pdf}
\end{center}
\vspace{-0.6cm}
 \caption{4 TeV: Comparison of IR3 cleaning induced showers at the interface plane generated with a positive and negative frequency shift for muons, protons, neutrons, photons and electrons from top left to bottom right.
  \label{compPM_ekin}}
\end{figure}

\begin{figure}%[!htb]
\begin{center}
  \includegraphics[width=0.30\textwidth]{figures/4TeV/offmom/comppm500Hz/ratioPhiNAll.pdf}
  \includegraphics[width=0.30\textwidth]{figures/4TeV/offmom/comppm500Hz/ratioPhiNProtons.pdf}
  \includegraphics[width=0.30\textwidth]{figures/4TeV/offmom/comppm500Hz/ratioPhiNMuons.pdf}
  \includegraphics[width=0.30\textwidth]{figures/4TeV/offmom/comppm500Hz/ratioPhiEnAll.pdf}
  \includegraphics[width=0.30\textwidth]{figures/4TeV/offmom/comppm500Hz/ratioPhiEnProtons.pdf}
  \includegraphics[width=0.30\textwidth]{figures/4TeV/offmom/comppm500Hz/ratioPhiEnMuons.pdf}
  \includegraphics[width=0.30\textwidth]{figures/4TeV/offmom/comppm500Hz/ratioRadNAll.pdf}
  \includegraphics[width=0.30\textwidth]{figures/4TeV/offmom/comppm500Hz/ratioRadNProtons.pdf}
  \includegraphics[width=0.30\textwidth]{figures/4TeV/offmom/comppm500Hz/ratioRadNMuons.pdf}
  \includegraphics[width=0.30\textwidth]{figures/4TeV/offmom/comppm500Hz/ratioRadEnAll.pdf}
  \includegraphics[width=0.30\textwidth]{figures/4TeV/offmom/comppm500Hz/ratioRadEnProtons.pdf}
  \includegraphics[width=0.30\textwidth]{figures/4TeV/offmom/comppm500Hz/ratioRadEnMuons.pdf}
\end{center}
\vspace{-0.6cm}
\caption{4 TeV: Comparisons of particle distributions in the ``plus'' and ``minus'' cases showing all particles (left colum), protons (middle) and muons (right). 
The first two rows highlight multiplicity and energy in $\phi$, and the last two rows show the multiplicity and energy in $r$. Since the radii are not inclusive distributions (everything with $r~>~600~$cm is cut off) unlike the $\phi$-distributions the integral ratios - all being below 1 - indicate that the ``plus-case'' gives higher contributions to smaller radii which means the ``minus-case'' produces more energetic particles only for $r > 600~$cm per TCT interaction. 
  \label{compPM_phien}}
\end{figure}
\newpage


\begin{figure}%[!htb]
\centering
\includegraphics[width=0.49\textwidth]{figures/BH_run2/b2/RadNDist_BH_6500GeV_haloB2_20MeV.pdf}
\includegraphics[width=0.49\textwidth]{figures/BH_run2/b2/RadEnDist_BH_6500GeV_haloB2_20MeV.pdf}
\includegraphics[width=0.49\textwidth]{figures/BH_run2/b2/OrigYZMuons_BH_6500GeV_haloB2_20MeV.pdf}
\includegraphics[width=0.49\textwidth]{figures/BH_run2/b2/OrigXYMuons_BH_6500GeV_haloB2_20MeV.pdf}
 \caption{Betatron halo induced background at the interface plane from B2. 
  \label{dist6500GeVB22}}
\end{figure}

\begin{figure}%[!htb]
\begin{center}
  \includegraphics[width=0.411\textwidth]{figures/BH_run2/perTCThit/ratioEkinMuons.pdf}
  \includegraphics[width=0.411\textwidth]{figures/BH_run2/perTCThit/ratioPhiNMuons.pdf}
  \includegraphics[width=0.411\textwidth]{figures/BH_run2/perTCThit/ratioPhiEnMuons.pdf}
  \includegraphics[width=0.411\textwidth]{figures/BH_run2/perTCThit/ratioRadEnMuons.pdf}
\end{center}
\vspace{-0.6cm}
 \caption{Comparison of B1 and B2 betatron halo induced distributions per TCT hit for muons.
  \label{compBHB1B2run2}}
\end{figure}

\clearpage

\begin{figure}%[!htb]
\begin{center}
%  \includegraphics[width=0.49\textwidth]{figures/6500GeV/20MeV/Ekin_BG_6500GeV_flat_20MeV_bs.pdf}
%  \includegraphics[width=0.49\textwidth]{figures/6500GeV/20MeV/PhiEnDist_BG_6500GeV_flat_20MeV_bs.pdf}
  \includegraphics[width=0.49\textwidth]{figures/6500GeV/20MeV/RadNDist_BG_6500GeV_flat_20MeV_bs.pdf}
  \includegraphics[width=0.49\textwidth]{figures/6500GeV/20MeV/RadEnDist_BG_6500GeV_flat_20MeV_bs.pdf}
\end{center}
\vspace{-0.6cm}
 \caption{Characteristic beam-gas induced distributions at 6.5~TeV per BG interaction using the more realistic model of the beam size.
  \label{bg65002}}
\end{figure}

\begin{figure}
\begin{center}
  \includegraphics[width=0.75\textwidth]{figures/6500GeV/reweighted/cv81_OrigZAll_BG_6500GeV_flat_20MeV_bs.pdf}
  \includegraphics[width=0.75\textwidth]{figures/6500GeV/reweighted/cv81_OrigZProtons_BG_6500GeV_flat_20MeV_bs.pdf}
  \includegraphics[width=0.75\textwidth]{figures/6500GeV/reweighted/cv81_OrigZPhotons_BG_6500GeV_flat_20MeV_bs.pdf}
\end{center}
\vspace{-0.6cm}
 \caption{Origin of all particles (top), protons (middle) and photons (bottom) production along s, in black before and in gold after re-normalising to the 2015 pressure profile. 
  \label{fig:OrigZ6p52}}
\end{figure}


\begin{figure}
\begin{center}
  \includegraphics[width=0.45\textwidth]{figures/6500GeV/compBGflat/ratioEkinAll}
  \includegraphics[width=0.45\textwidth]{figures/6500GeV/compBGflat/ratioPhiEnAll}
  \includegraphics[width=0.45\textwidth]{figures/6500GeV/compBGflat/ratioRadNAll}
  \includegraphics[width=0.45\textwidth]{figures/6500GeV/compBGflat/ratioRadEnAll}
\end{center}
\vspace{-0.6cm}
 \caption{Beam-gas distributions at the interface plane before (flat) and after normalising to the pressure at 6.5~TeV (reweighted) for all particles. Note, the black dashed line is scaled up by a factor $10^7$.
  \label{fig:EkinPhiEn6p52}}
\end{figure}



\begin{figure}
\begin{center}
  \includegraphics[width=0.45\textwidth]{figures/6500GeV/compBGflat/ratioEkinProtons}
  \includegraphics[width=0.45\textwidth]{figures/6500GeV/compBGflat/ratioEkinNeutrons}
  \includegraphics[width=0.45\textwidth]{figures/6500GeV/compBGflat/ratioPhiEnProtons}
  \includegraphics[width=0.45\textwidth]{figures/6500GeV/compBGflat/ratioPhiEnNeutrons}
  \includegraphics[width=0.45\textwidth]{figures/6500GeV/compBGflat/ratioRadNProtons}
  \includegraphics[width=0.45\textwidth]{figures/6500GeV/compBGflat/ratioRadNNeutrons}
  \includegraphics[width=0.45\textwidth]{figures/6500GeV/compBGflat/ratioRadEnProtons}
  \includegraphics[width=0.45\textwidth]{figures/6500GeV/compBGflat/ratioRadEnNeutrons}
\end{center}
\vspace{-0.6cm}
 \caption{Beam-gas distributions at the interface plane before (flat) and after normalising to the pressure at 6.5~TeV (reweighted) for protons (left) and neutrons (right). The black dashed line is scaled up by a factor $10^7$.
  \label{fig:ProtNeut6p52}} 
\end{figure}

% ----------------------------------
\clearpage

