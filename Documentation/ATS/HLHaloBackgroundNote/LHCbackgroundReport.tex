\begin{filecontents}{pictexa.tex}
\documentclass{article}
\usepackage{epic,eepic,pspicture,mathptmx,times,graphpap}


\newcommand{\lumi}[1]{\ensuremath{{\mathcal{L}= 10^{#1}\,\mathrm{cm}^{-2}\,\mathrm{s}^{-1}}}}
\newcommand{\lumiapprox}[1]{\ensuremath{{\mathcal{L} \approx 10^{#1}\,\mathrm{cm}^{-2}\,\mathrm{s}^{-1}}}}
\newcommand{\lumif}[2]{\ensuremath{{\mathcal{L}= {#1} \times 10^{#2}\,\mathrm{cm}^{-2}\,\mathrm{s}^{-1}}}}
\newcommand{\ifb}{\mbox{fb$^{-1}$}}%  Inverse femtobarns.
\newcommand{\ipb}{\mbox{pb$^{-1}$}}%  Inverse picobarns.
\newcommand{\inb}{\mbox{nb$^{-1}$}}%  Inverse nanobarns.
\newcommand{\sph}{\mbox{$\mu$Sv/h}}%  mu Sv/h
\newcommand{\nseu}{$N_{\mathrm{SEU}}$}
\newcommand{\twosigmaret}{$2\sigma$-retracted}
\newcommand{\thnf}{\ensuremath{\Phi_{\mathrm{thn}}}}
\newcommand{\fluka}{\textsc{Fluka}}
\newcommand{\mypercent}{\%}

\begin{document}
\pagestyle{empty}
\begin{picture}(60,40)(-2,-2)
\setlength{\unitlength}{1mm}
%\graphpaper(0,0)(70,50)
\arrowlength{2mm}\linethickness{1pt}
\put(0,0){\Vector(60,0)}
\put(0,0){\Vector(0,40)}
\thicklines
\put(15,0){\Line(35,35)}
\thinlines
\dashline{3}(50,0)(50,35)
\dashline{3}(0,35)(50,35)
\dashline{2}(15,0)(15,35)
\put(15,0){\arc{19}{4.7124}{5.4978}}
\put(17.5,10.5){\ensuremath{\displaystyle\theta}}
\put(1,37){\emph{h}}
\put(51,2){\emph{n(h)}}
\end{picture}
\end{document}
\end{filecontents}
\begin{filecontents*}{pictexa.eps}
%!PS-Adobe-2.0 EPSF-2.0
%%Creator: dvips(k) 5.94b Copyright 2004 Radical Eye Software
%%Title: pictexa.dvi
%%BoundingBox: 148 627 322 745
%%DocumentFonts: Symbol Times-Italic
%%EndComments
%DVIPSWebPage: (www.radicaleye.com)
%DVIPSCommandLine: dvips -E pictexa -opictexa.eps
%DVIPSParameters: dpi=600, compressed
%DVIPSSource:  TeX output 2005.05.30:1757
%%BeginProcSet: texc.pro 0 0
%!
/TeXDict 300 dict def TeXDict begin/N{def}def/B{bind def}N/S{exch}N/X{S
N}B/A{dup}B/TR{translate}N/isls false N/vsize 11 72 mul N/hsize 8.5 72
mul N/landplus90{false}def/@rigin{isls{[0 landplus90{1 -1}{-1 1}ifelse 0
0 0]concat}if 72 Resolution div 72 VResolution div neg scale isls{
landplus90{VResolution 72 div vsize mul 0 exch}{Resolution -72 div hsize
mul 0}ifelse TR}if Resolution VResolution vsize -72 div 1 add mul TR[
matrix currentmatrix{A A round sub abs 0.00001 lt{round}if}forall round
exch round exch]setmatrix}N/@landscape{/isls true N}B/@manualfeed{
statusdict/manualfeed true put}B/@copies{/#copies X}B/FMat[1 0 0 -1 0 0]
N/FBB[0 0 0 0]N/nn 0 N/IEn 0 N/ctr 0 N/df-tail{/nn 8 dict N nn begin
/FontType 3 N/FontMatrix fntrx N/FontBBox FBB N string/base X array
/BitMaps X/BuildChar{CharBuilder}N/Encoding IEn N end A{/foo setfont}2
array copy cvx N load 0 nn put/ctr 0 N[}B/sf 0 N/df{/sf 1 N/fntrx FMat N
df-tail}B/dfs{div/sf X/fntrx[sf 0 0 sf neg 0 0]N df-tail}B/E{pop nn A
definefont setfont}B/Cw{Cd A length 5 sub get}B/Ch{Cd A length 4 sub get
}B/Cx{128 Cd A length 3 sub get sub}B/Cy{Cd A length 2 sub get 127 sub}
B/Cdx{Cd A length 1 sub get}B/Ci{Cd A type/stringtype ne{ctr get/ctr ctr
1 add N}if}B/id 0 N/rw 0 N/rc 0 N/gp 0 N/cp 0 N/G 0 N/CharBuilder{save 3
1 roll S A/base get 2 index get S/BitMaps get S get/Cd X pop/ctr 0 N Cdx
0 Cx Cy Ch sub Cx Cw add Cy setcachedevice Cw Ch true[1 0 0 -1 -.1 Cx
sub Cy .1 sub]/id Ci N/rw Cw 7 add 8 idiv string N/rc 0 N/gp 0 N/cp 0 N{
rc 0 ne{rc 1 sub/rc X rw}{G}ifelse}imagemask restore}B/G{{id gp get/gp
gp 1 add N A 18 mod S 18 idiv pl S get exec}loop}B/adv{cp add/cp X}B
/chg{rw cp id gp 4 index getinterval putinterval A gp add/gp X adv}B/nd{
/cp 0 N rw exit}B/lsh{rw cp 2 copy get A 0 eq{pop 1}{A 255 eq{pop 254}{
A A add 255 and S 1 and or}ifelse}ifelse put 1 adv}B/rsh{rw cp 2 copy
get A 0 eq{pop 128}{A 255 eq{pop 127}{A 2 idiv S 128 and or}ifelse}
ifelse put 1 adv}B/clr{rw cp 2 index string putinterval adv}B/set{rw cp
fillstr 0 4 index getinterval putinterval adv}B/fillstr 18 string 0 1 17
{2 copy 255 put pop}for N/pl[{adv 1 chg}{adv 1 chg nd}{1 add chg}{1 add
chg nd}{adv lsh}{adv lsh nd}{adv rsh}{adv rsh nd}{1 add adv}{/rc X nd}{
1 add set}{1 add clr}{adv 2 chg}{adv 2 chg nd}{pop nd}]A{bind pop}
forall N/D{/cc X A type/stringtype ne{]}if nn/base get cc ctr put nn
/BitMaps get S ctr S sf 1 ne{A A length 1 sub A 2 index S get sf div put
}if put/ctr ctr 1 add N}B/I{cc 1 add D}B/bop{userdict/bop-hook known{
bop-hook}if/SI save N @rigin 0 0 moveto/V matrix currentmatrix A 1 get A
mul exch 0 get A mul add .99 lt{/QV}{/RV}ifelse load def pop pop}N/eop{
SI restore userdict/eop-hook known{eop-hook}if showpage}N/@start{
userdict/start-hook known{start-hook}if pop/VResolution X/Resolution X
1000 div/DVImag X/IEn 256 array N 2 string 0 1 255{IEn S A 360 add 36 4
index cvrs cvn put}for pop 65781.76 div/vsize X 65781.76 div/hsize X}N
/p{show}N/RMat[1 0 0 -1 0 0]N/BDot 260 string N/Rx 0 N/Ry 0 N/V{}B/RV/v{
/Ry X/Rx X V}B statusdict begin/product where{pop false[(Display)(NeXT)
(LaserWriter 16/600)]{A length product length le{A length product exch 0
exch getinterval eq{pop true exit}if}{pop}ifelse}forall}{false}ifelse
end{{gsave TR -.1 .1 TR 1 1 scale Rx Ry false RMat{BDot}imagemask
grestore}}{{gsave TR -.1 .1 TR Rx Ry scale 1 1 false RMat{BDot}
imagemask grestore}}ifelse B/QV{gsave newpath transform round exch round
exch itransform moveto Rx 0 rlineto 0 Ry neg rlineto Rx neg 0 rlineto
fill grestore}B/a{moveto}B/delta 0 N/tail{A/delta X 0 rmoveto}B/M{S p
delta add tail}B/b{S p tail}B/c{-4 M}B/d{-3 M}B/e{-2 M}B/f{-1 M}B/g{0 M}
B/h{1 M}B/i{2 M}B/j{3 M}B/k{4 M}B/w{0 rmoveto}B/l{p -4 w}B/m{p -3 w}B/n{
p -2 w}B/o{p -1 w}B/q{p 1 w}B/r{p 2 w}B/s{p 3 w}B/t{p 4 w}B/x{0 S
rmoveto}B/y{3 2 roll p a}B/bos{/SS save N}B/eos{SS restore}B end

%%EndProcSet
%%BeginProcSet: pspicture.ps 0 0
%!
%%
%% Source File `pspicture.dtx'.
%% Copyright (C) 1992 1999 David Carlisle
%% This file may be distributed under the terms of the LPPL.
%% See 00readme.txt for details.
%%




/!BP{
  72 72.27 div dup scale
  }def
/!A{
  newpath
  0 0 moveto
  dup neg dup .4 mul rlineto
  .8 mul 0 exch  rlineto
  closepath
  fill
  } def
/!V{
  !BP
  /!X exch def
  /!y exch def
  /!x exch def
  newpath
  0 0 moveto
  !x 0 eq {0  !y 0 lt {!X neg}{!X} ifelse}
         {!x 0 lt {!X neg}{!X}ifelse  !X !y mul !x abs div} ifelse
  lineto
  setlinewidth  % @wholewidth
  currentpoint
  stroke
  translate
  !y !x atan
  rotate
  !A            % @arrowlength
  }def
/!L{
  !BP
  /!X exch def
  /!y exch def
  /!x exch def
  newpath
  0 0 moveto
  !x 0 eq {0  !y 0 lt {!X neg}{!X} ifelse}
         {!x 0 lt {!X neg}{!X}ifelse  !X !y mul !x abs div} ifelse
  lineto
  setlinewidth  % @wholewidth
  stroke
  }def
/!C{
  !BP
  0 0 3 2 roll
  2 div 0 360 arc
  setlinewidth  % @wholewidth
  stroke
  }def
/!D{
  !BP
  0 0 3 2 roll
  2 div 0 360 arc fill
  }def
/!O{
  !BP
  /!y exch 2 div def
  /!x exch 2 div def
  /!r exch !x !y
    2 copy gt {exch} if pop
    2 copy gt {exch} if pop
      def
  setlinewidth  % @wholewidth
  1 eq
  {newpath
   !x neg 0 moveto
   !x neg !y 0 !y !r arcto 4 {pop} repeat
   0 !y lineto
   stroke}if
  1 eq
  {newpath
   !x  0 moveto
   !x  !y 0 !y !r arcto 4 {pop} repeat
   0 !y lineto
   stroke}if
  1 eq
  {newpath
   !x neg 0 moveto
   !x neg !y neg 0 !y neg  !r arcto 4 {pop} repeat
   0 !y neg lineto
   stroke}if
  1 eq
  {newpath
   !x  0 moveto
   !x  !y neg 0 !y neg !r arcto 4 {pop} repeat
   0 !y neg lineto
   stroke}if
  }def
/!V2{
  !BP
  2 copy exch
  atan
  /a exch def
  2 copy
  newpath
  0 0 moveto
  lineto          % <x*unitlength> <y*unitlength>
  3 2 roll
  setlinewidth  % @wholewidth
  stroke
  translate       % <x*unitlength> <y*unitlength>
  a rotate
  !A                    % @arrowlength
  }def
/!L2{
  !BP
  newpath
  0 0 moveto
  lineto          % <x*unitlength> <y*unitlength>
  setlinewidth  % @wholewidth
  stroke
  }def
/!C2{
  !BP
  /!s exch def
  /!y exch def
  /!x exch def
  newpath
  0 0 moveto
  0 0
  !x 2 div !y 10 div !s mul add
  !y 2 div  !x 10 div  !s mul sub
  !x !y
  curveto
  setlinewidth  % @wholewidth
  stroke
  }def

%%EndProcSet
%%BeginProcSet: 8r.enc 0 0
% @@psencodingfile@{
%   author    = "S. Rahtz, P. MacKay, Alan Jeffrey, B. Horn, K. Berry,
%                W. Schmidt, P. Lehman",
%   version   = "20021105.19",
%   date      = "5 November 2002",
%   filename  = "8r.enc",
%   email     = "tex-fonts@@tug.org",
%   docstring = "This is the encoding vector for Type1 and TrueType
%                fonts to be used with TeX.  This file is also included
%                in the PSNFSS bundle."
% @}
% 
% The idea is to have all the characters normally included in Type 1 fonts
% available for typesetting. This is effectively the characters in Adobe
% Standard encoding, ISO Latin 1, Windows ANSI including the euro symbol,
% MacRoman, and some extra characters from Lucida.
% 
% Character code assignments were made as follows:
% 
% (1) the Windows ANSI characters are almost all in their Windows ANSI
% positions, because some Windows users cannot easily reencode the
% fonts, and it makes no difference on other systems. The only Windows
% ANSI characters not available are those that make no sense for
% typesetting -- rubout (127 decimal), nobreakspace (160), softhyphen
% (173). quotesingle and grave are moved just because it's such an
% irritation not having them in TeX positions.
% 
% (2) Remaining characters are assigned arbitrarily to the lower part
% of the range, avoiding 0, 10 and 13 in case we meet dumb software.
% 
% (3) Y&Y Lucida Bright includes some extra text characters; in the
% hopes that other PostScript fonts, perhaps created for public
% consumption, will include them, they are included starting at 0x12.
% These are /dotlessj /ff /ffi /ffl.
% 
% (4) hyphen appears twice for compatibility with both ASCII and Windows.
%
% (5) /Euro was assigned to 128, as in Windows ANSI.
%
% (6) Missing characters from MacRoman encoding incorporated in October
% 2002 as follows:
%
%     PostScript      MacRoman        TeXBase1
%     --------------  --------------  --------------
%     /notequal       173             0x16
%     /infinity       176             0x17
%     /lessequal      178             0x18
%     /greaterequal   179             0x19
%     /partialdiff    182             0x1A
%     /summation      183             0x1B
%     /product        184             0x1C
%     /pi             185             0x1D
%     /integral       186             0x81
%     /Omega          189             0x8D
%     /radical        195             0x8E
%     /approxequal    197             0x8F
%     /Delta          198             0x9D
%     /lozenge        215             0x9E
%
/TeXBase1Encoding [
% 0x00
 /.notdef /dotaccent /fi /fl
 /fraction /hungarumlaut /Lslash /lslash
 /ogonek /ring /.notdef /breve
 /minus /.notdef /Zcaron /zcaron
% 0x10
 /caron /dotlessi /dotlessj /ff
 /ffi /ffl /notequal /infinity
 /lessequal /greaterequal /partialdiff /summation
 /product /pi /grave /quotesingle
% 0x20
 /space /exclam /quotedbl /numbersign
 /dollar /percent /ampersand /quoteright
 /parenleft /parenright /asterisk /plus
 /comma /hyphen /period /slash
% 0x30
 /zero /one /two /three
 /four /five /six /seven
 /eight /nine /colon /semicolon
 /less /equal /greater /question
% 0x40
 /at /A /B /C
 /D /E /F /G
 /H /I /J /K
 /L /M /N /O
% 0x50
 /P /Q /R /S
 /T /U /V /W
 /X /Y /Z /bracketleft
 /backslash /bracketright /asciicircum /underscore
% 0x60
 /quoteleft /a /b /c
 /d /e /f /g
 /h /i /j /k
 /l /m /n /o
% 0x70
 /p /q /r /s
 /t /u /v /w
 /x /y /z /braceleft
 /bar /braceright /asciitilde /.notdef
% 0x80
 /Euro /integral /quotesinglbase /florin
 /quotedblbase /ellipsis /dagger /daggerdbl
 /circumflex /perthousand /Scaron /guilsinglleft
 /OE /Omega /radical /approxequal
% 0x90
 /.notdef /.notdef /.notdef /quotedblleft
 /quotedblright /bullet /endash /emdash
 /tilde /trademark /scaron /guilsinglright
 /oe /Delta /lozenge /Ydieresis
% 0xA0
 /.notdef /exclamdown /cent /sterling
 /currency /yen /brokenbar /section
 /dieresis /copyright /ordfeminine /guillemotleft
 /logicalnot /hyphen /registered /macron
% 0xD0
 /degree /plusminus /twosuperior /threesuperior
 /acute /mu /paragraph /periodcentered
 /cedilla /onesuperior /ordmasculine /guillemotright
 /onequarter /onehalf /threequarters /questiondown
% 0xC0
 /Agrave /Aacute /Acircumflex /Atilde
 /Adieresis /Aring /AE /Ccedilla
 /Egrave /Eacute /Ecircumflex /Edieresis
 /Igrave /Iacute /Icircumflex /Idieresis
% 0xD0
 /Eth /Ntilde /Ograve /Oacute
 /Ocircumflex /Otilde /Odieresis /multiply
 /Oslash /Ugrave /Uacute /Ucircumflex
 /Udieresis /Yacute /Thorn /germandbls
% 0xE0
 /agrave /aacute /acircumflex /atilde
 /adieresis /aring /ae /ccedilla
 /egrave /eacute /ecircumflex /edieresis
 /igrave /iacute /icircumflex /idieresis
% 0xF0
 /eth /ntilde /ograve /oacute
 /ocircumflex /otilde /odieresis /divide
 /oslash /ugrave /uacute /ucircumflex
 /udieresis /yacute /thorn /ydieresis
] def

%%EndProcSet
%%BeginProcSet: texps.pro 0 0
%!
TeXDict begin/rf{findfont dup length 1 add dict begin{1 index/FID ne 2
index/UniqueID ne and{def}{pop pop}ifelse}forall[1 index 0 6 -1 roll
exec 0 exch 5 -1 roll VResolution Resolution div mul neg 0 0]FontType 0
ne{/Metrics exch def dict begin Encoding{exch dup type/integertype ne{
pop pop 1 sub dup 0 le{pop}{[}ifelse}{FontMatrix 0 get div Metrics 0 get
div def}ifelse}forall Metrics/Metrics currentdict end def}{{1 index type
/nametype eq{exit}if exch pop}loop}ifelse[2 index currentdict end
definefont 3 -1 roll makefont/setfont cvx]cvx def}def/ObliqueSlant{dup
sin S cos div neg}B/SlantFont{4 index mul add}def/ExtendFont{3 -1 roll
mul exch}def/ReEncodeFont{CharStrings rcheck{/Encoding false def dup[
exch{dup CharStrings exch known not{pop/.notdef/Encoding true def}if}
forall Encoding{]exch pop}{cleartomark}ifelse}if/Encoding exch def}def
end

%%EndProcSet
%%BeginProcSet: special.pro 0 0
%!
TeXDict begin/SDict 200 dict N SDict begin/@SpecialDefaults{/hs 612 N
/vs 792 N/ho 0 N/vo 0 N/hsc 1 N/vsc 1 N/ang 0 N/CLIP 0 N/rwiSeen false N
/rhiSeen false N/letter{}N/note{}N/a4{}N/legal{}N}B/@scaleunit 100 N
/@hscale{@scaleunit div/hsc X}B/@vscale{@scaleunit div/vsc X}B/@hsize{
/hs X/CLIP 1 N}B/@vsize{/vs X/CLIP 1 N}B/@clip{/CLIP 2 N}B/@hoffset{/ho
X}B/@voffset{/vo X}B/@angle{/ang X}B/@rwi{10 div/rwi X/rwiSeen true N}B
/@rhi{10 div/rhi X/rhiSeen true N}B/@llx{/llx X}B/@lly{/lly X}B/@urx{
/urx X}B/@ury{/ury X}B/magscale true def end/@MacSetUp{userdict/md known
{userdict/md get type/dicttype eq{userdict begin md length 10 add md
maxlength ge{/md md dup length 20 add dict copy def}if end md begin
/letter{}N/note{}N/legal{}N/od{txpose 1 0 mtx defaultmatrix dtransform S
atan/pa X newpath clippath mark{transform{itransform moveto}}{transform{
itransform lineto}}{6 -2 roll transform 6 -2 roll transform 6 -2 roll
transform{itransform 6 2 roll itransform 6 2 roll itransform 6 2 roll
curveto}}{{closepath}}pathforall newpath counttomark array astore/gc xdf
pop ct 39 0 put 10 fz 0 fs 2 F/|______Courier fnt invertflag{PaintBlack}
if}N/txpose{pxs pys scale ppr aload pop por{noflips{pop S neg S TR pop 1
-1 scale}if xflip yflip and{pop S neg S TR 180 rotate 1 -1 scale ppr 3
get ppr 1 get neg sub neg ppr 2 get ppr 0 get neg sub neg TR}if xflip
yflip not and{pop S neg S TR pop 180 rotate ppr 3 get ppr 1 get neg sub
neg 0 TR}if yflip xflip not and{ppr 1 get neg ppr 0 get neg TR}if}{
noflips{TR pop pop 270 rotate 1 -1 scale}if xflip yflip and{TR pop pop
90 rotate 1 -1 scale ppr 3 get ppr 1 get neg sub neg ppr 2 get ppr 0 get
neg sub neg TR}if xflip yflip not and{TR pop pop 90 rotate ppr 3 get ppr
1 get neg sub neg 0 TR}if yflip xflip not and{TR pop pop 270 rotate ppr
2 get ppr 0 get neg sub neg 0 S TR}if}ifelse scaleby96{ppr aload pop 4
-1 roll add 2 div 3 1 roll add 2 div 2 copy TR .96 dup scale neg S neg S
TR}if}N/cp{pop pop showpage pm restore}N end}if}if}N/normalscale{
Resolution 72 div VResolution 72 div neg scale magscale{DVImag dup scale
}if 0 setgray}N/psfts{S 65781.76 div N}N/startTexFig{/psf$SavedState
save N userdict maxlength dict begin/magscale true def normalscale
currentpoint TR/psf$ury psfts/psf$urx psfts/psf$lly psfts/psf$llx psfts
/psf$y psfts/psf$x psfts currentpoint/psf$cy X/psf$cx X/psf$sx psf$x
psf$urx psf$llx sub div N/psf$sy psf$y psf$ury psf$lly sub div N psf$sx
psf$sy scale psf$cx psf$sx div psf$llx sub psf$cy psf$sy div psf$ury sub
TR/showpage{}N/erasepage{}N/setpagedevice{pop}N/copypage{}N/p 3 def
@MacSetUp}N/doclip{psf$llx psf$lly psf$urx psf$ury currentpoint 6 2 roll
newpath 4 copy 4 2 roll moveto 6 -1 roll S lineto S lineto S lineto
closepath clip newpath moveto}N/endTexFig{end psf$SavedState restore}N
/@beginspecial{SDict begin/SpecialSave save N gsave normalscale
currentpoint TR @SpecialDefaults count/ocount X/dcount countdictstack N}
N/@setspecial{CLIP 1 eq{newpath 0 0 moveto hs 0 rlineto 0 vs rlineto hs
neg 0 rlineto closepath clip}if ho vo TR hsc vsc scale ang rotate
rwiSeen{rwi urx llx sub div rhiSeen{rhi ury lly sub div}{dup}ifelse
scale llx neg lly neg TR}{rhiSeen{rhi ury lly sub div dup scale llx neg
lly neg TR}if}ifelse CLIP 2 eq{newpath llx lly moveto urx lly lineto urx
ury lineto llx ury lineto closepath clip}if/showpage{}N/erasepage{}N
/setpagedevice{pop}N/copypage{}N newpath}N/@endspecial{count ocount sub{
pop}repeat countdictstack dcount sub{end}repeat grestore SpecialSave
restore end}N/@defspecial{SDict begin}N/@fedspecial{end}B/li{lineto}B
/rl{rlineto}B/rc{rcurveto}B/np{/SaveX currentpoint/SaveY X N 1
setlinecap newpath}N/st{stroke SaveX SaveY moveto}N/fil{fill SaveX SaveY
moveto}N/ellipse{/endangle X/startangle X/yrad X/xrad X/savematrix
matrix currentmatrix N TR xrad yrad scale 0 0 1 startangle endangle arc
savematrix setmatrix}N end

%%EndProcSet
TeXDict begin 40258437 52099154 1000 600 600 (pictexa.dvi)
@start /Fa 145[42 5[42 62[28 28 40[{TeXBase1Encoding ReEncodeFont}4
83.022 /Times-Italic rf /Fb 142[43 113[{.167 SlantFont}1
83.022 /Symbol rf end
%%EndProlog
%%BeginSetup
%%Feature: *Resolution 600dpi
TeXDict begin
 end
%%EndSetup
TeXDict begin 1 0 bop 656 756 a @beginspecial @setspecial
5.69054 1.0 170.71564 0.0 !V2


@endspecial @beginspecial @setspecial
5.69054 1.0 0.0 113.81042 !V2
 
@endspecial 354
w @beginspecial @setspecial
0.79999 99.58412 99.58412 !L2
 
@endspecial 3 setlinewidth
np 1837 756 a 1837 687 li st 3 setlinewidth np 1837 604
a 1837 535 li st 3 setlinewidth np 1837 452 a 1837 383
li st 3 setlinewidth np 1837 301 a 1837 232 li st 3 setlinewidth
np 1837 149 a 1837 80 li st 3 setlinewidth np 1837 -2
a 1837 -71 li st 3 setlinewidth np 656 -71 a 725 -71
li st 3 setlinewidth np 815 -71 a 884 -71 li st 3 setlinewidth
np 973 -71 a 1042 -71 li st 3 setlinewidth np 1132 -71
a 1201 -71 li st 3 setlinewidth np 1291 -71 a 1360 -71
li st 3 setlinewidth np 1450 -71 a 1519 -71 li st 3 setlinewidth
np 1609 -71 a 1678 -71 li st 3 setlinewidth np 1767 -71
a 1836 -71 li st 3 setlinewidth np 1010 756 a 1010 713
li st 3 setlinewidth np 1010 658 a 1010 615 li st 3 setlinewidth
np 1010 560 a 1010 517 li st 3 setlinewidth np 1010 462
a 1010 419 li st 3 setlinewidth np 1010 364 a 1010 321
li st 3 setlinewidth np 1010 266 a 1010 223 li st 3 setlinewidth
np 1010 168 a 1010 125 li st 3 setlinewidth np 1010 70
a 1010 27 li st 3 setlinewidth np 1010 -28 a 1010 -71
li st 3 setlinewidth np 1010 756 224 270.00 315.00 arc
st 1069 507 a Fb(q)679 -119 y Fa(h)1861 708 y(n\(h\))p
eop end
%%Trailer

userdict /end-hook known{end-hook}if
%%EOF
\end{filecontents*}


\documentclass{accreport}
\def\docnumber{xxx}
\def\emailaddress{Regina.Kwee@rhul.ac.uk}

\pagestyle{fancy}

\begin{document}

\title{Machine-induced Background Simulation Studies for LHC Run~I, Run~II and HL-LHC}
\author{R.~Kwee-Hinzmann\thanks{Royal Holloway University of London}, G.~Bregliozzi, R.~Bruce, F.~Cerutti,~L.~S.~Esposito, S.~M.~Gibson, A.~Lechner, H.~Garcia Morales, C.~Yin Vallgren}% Other
                 %{On leave from another institute somewhere.}}
\institute{Institute name in English, Town, Country}
\maketitle

\begin{RepAbstract}
  The study of machine-induced background to the experiments is vital for several reasons. Too much background can be an issue for operation and the difficult part is to judge when exactly it is ``too much''. It is a complex topic as experiments are directly or indirectly affected by conditions all around the LHC ring e.g.~collimation settings and vacuum quality. A detailed study of background can also help understanding the machine better to identify potential issues and complemented by dedicated machine tests. Finally such a study is relevant for the experiments to analyse the characteristica of machine background to make sure not to count it into a new physics signal.

  This report summarises simulation results of three background sources, local beam-gas, beam-halo from the betatron collimation in IR7 and for the first time beam-halo from momentum collimation in IR3. The most dominant sources, betatron halo and beam-gas, have been systematically studied for LHC Run~I and Run~II cases, and several HL--LHC scenarios. We analyse the evolution of background of these scenarios and discuss rates expected for HL. Impacts on the tertiary collimators upstream of ATLAS and CMS are analysed, while particle distributions streaming into the experimental area were estimated for ATLAS only in detailed shower simulations in all of these scenarios.
\end{RepAbstract}
\begin{Keywords}
background, beam-gas, halo, SixTrack, FLUKA, HL-LHC, collimation, cleaning efficiency, IR3 leakage, IR7 leakage, off-momentum, ATLAS, CMS}
\end{Keywords}

\par \vspace{2cm} \par

\begin{center}
Presented at: \par
\par \vspace{2cm} \par
City, Country \par
Month, \the\year
\end{center}

\newpage
\pagenumbering{roman}
\tableofcontents

\newpage
\pagenumbering{arabic}
\section{Section heading}

A section title is styled as above, and first paragraphs after
headings are not indented. References appear in numerical
order~\cite{Raby1966,Dupont1961}. An itemized list looks like the following:
\begin{itemize}
\item the first item,
\item the second item.
\end{itemize}

You can also have an enumerated list:

\begin{enumerate}
\item the first item,
\item the second item.
\end{enumerate}

\subsection{Subsection heading}

As you see the first paragraph is not indented. References when part
      of the text use the term `Ref.', for example, see
      Ref.~\cite{Raby1966} and
      Refs.~\cite{Appleman1959,vanBerg1965,Bryant1985,Allen1977}.

Subsequent paragraphs are indented. See Table~\ref{tab:LET} for an
example of how to display a table.

\begin{table}[h]
\begin{center}
\caption{A simple table}
\label{tab:LET}
\begin{tabular}{p{6cm}cc}
\hline\hline
\textbf{Heading}             & \textbf{Result 1}
                                                & \textbf{Result 2}\\
\hline
200 kVp X-rays total$^{a}$   & 3.25             & 1.79 \\
200 kVp X-rays (primary)     & 2.60             & 1.48 \\
\hline\hline
\multicolumn{3}{l}{$^{a}$ \footnotesize Notes in tables appear as
                      this one here.}
\end{tabular}
\end{center}
\end{table}

\subsubsection{Subsubsection title}
\label{sec:sss}

Equation~(\ref{eq:a1}) is presented correctly:
\begin{equation}
n^k(h)= k h \frac{k}{32}~. \label{eq:a1}  
\end{equation}

This is how all equations should be formatted,
including~Eqs.~(2)--(10), 
%% With LaTeX use: Eqs,~(\ref{eq:a2})--(\ref{eq:a10}) syntax.
which are not shown\footnote{Footnotes are
to be used only when absolutely necessary.}.

\subsubsubsection{Subsubsubsection title}

Figure~\ref{tab:LET} is an example of how to display figures. Please
refer to figures as Figs.~2--4, etc.
%% With LaTeX use : Figs.~\ref{fig:a1}--\ref{fig:a4} syntax.
Section~\ref{sec:sss} is a cross-reference to a section. 
See the document describing the CERN reports~\cite{cernrep} for more
details on how to present figures, tables, equations, and much more
with \LaTeX. 

\begin{figure}[ht]
\begin{center}
\includegraphics[width=6cm]{pictexa}
\caption{Diagram of a straight line}
\label{fig:line}
\end{center}
\end{figure}

Examples of references and a bibliography follow the acknowledgements.

\section{Acknowledgements}

We wish to thank A.N. Colleague for enlightening comments on
the present topic.

\begin{thebibliography}{99}
\bibitem{Raby1966}
J.M. Raby, Biophysical aspects of radiation quality, International 
Atomic EnergyAgency, Technical Reports Series No. 58 (1966).
\bibitem{Dupont1961}
J.-P. Dupont, Proc. Int. Conf. on Radiation Hazards, Columbia, 
1960 (Academic Press Inc., New York, 1961), Vol. II, p. 396.
\bibitem{Appleman1959}
H. Appleman \emph{et al.}, J. \emph{Med. Biol.} \textbf{8} (1959) 911.
\bibitem{vanBerg1965}
E. van Berg, D. Johnson and J. Smith, \emph{Rad. Res.} \textbf{5} (1965) 
215.
\bibitem{Bryant1985}
P. Bryant and S. Newman (Eds.), The generation of high fields, 
CAS--ECFA--INFN Workshop, Frascati, 1984., ECFA 85/91, CERN 85/07 
(1985).
\bibitem{Allen1977}
M.A. Allen \emph{et al.}, \emph{IEEE Trans. Nucl. Sci.} \textbf{NS--24} (1977) 
1780.
\bibitem{cernrep}
DTP Section, Preparing contributions to CERN reports,
\url{http://cern.ch/DTP/cernrep.pdf}.
\end{thebibliography}

\section{Bibliography}


I.C. Percival and D. Richards, \emph{Introduction to Dynamics}
(Cambridge University Press, Cambridge, 1982).

\appendix
\section{Title of appendix}
\label{sec:app}

\subsection{Subsection title in appendix}

Inside an appendix the same level of headings (section, subsection,
etc.) as in the main text applies. Only the first number is replaced
by an uppercase letter.

\subsubsection{Subsubsection title in appendix}

Inside a subsubsection inside an appendix.

\subsubsubsection{Subsubsubsection title in appendix}

\end{document}
