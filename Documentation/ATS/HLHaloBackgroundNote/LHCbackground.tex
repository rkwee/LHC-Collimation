% This is cernatsreport.tex in text format, as of 17 September 2012
% adapted from abreport.tex in text format, as of 27 March 2003.

% This file gives you a basic template for writing an CERN ATS Sector report.

% This file MUST be studied in conjunction with:
%    LaTeX: A Document Preparation System  by Leslie Lamport
%    Addison-Wesley, Reading, Massachussetts,  .
%

%------------------------------------------------------------------------
% This is a comment since TeX ignores everything to the right of a '%'.
% This file is much longer than it need be only because there are lots
% of comments to explain it.

% Now comes the first LaTeX command in this file.

\documentclass{cernatsreport}    % Specifies the document style.
\usepackage{vmargin,times,graphicx,amsmath,amssymb,color} % ,draftcopy use draftcopy for experiments
\usepackage{verbatim} % to allow for verbatim and comment
\usepackage{lineno} % to allow for linenumbers in the draft version
\usepackage{tabularx}
\usepackage{multirow}
\usepackage{hyperref}
\newcommand{\lumi}[1]{\ensuremath{{\mathcal{L}= 10^{#1}\,\mathrm{cm}^{-2}\,\mathrm{s}^{-1}}}}
\newcommand{\lumiapprox}[1]{\ensuremath{{\mathcal{L} \approx 10^{#1}\,\mathrm{cm}^{-2}\,\mathrm{s}^{-1}}}}
\newcommand{\lumif}[2]{\ensuremath{{\mathcal{L}= {#1} \times 10^{#2}\,\mathrm{cm}^{-2}\,\mathrm{s}^{-1}}}}
\newcommand{\ifb}{\mbox{fb$^{-1}$}}%  Inverse femtobarns.
\newcommand{\ipb}{\mbox{pb$^{-1}$}}%  Inverse picobarns.
\newcommand{\inb}{\mbox{nb$^{-1}$}}%  Inverse nanobarns.
\newcommand{\sph}{\mbox{$\mu$Sv/h}}%  mu Sv/h
\newcommand{\nseu}{$N_{\mathrm{SEU}}$}
\newcommand{\twosigmaret}{$2\sigma$-retracted}
\newcommand{\thnf}{\ensuremath{\Phi_{\mathrm{thn}}}}
\newcommand{\fluka}{\textsc{Fluka}}
\newcommand{\mypercent}{\%}


	% Now follows the "preamble" with some declarations.
	% No text should be produced until after the \begin{document}

\documentlabel{CERN-ACC-2017-0025}       %Declares label for this document. Modify as required. 
%\email{\small{Regina.Kwee@cern.ch}}
\typist{Regina.Kwee@rhul.ac.uk}     % optional command to indicate who typed the report.
                    % This command was useful in the good old days.

\title{Machine-induced Background Simulation Studies 
  for LHC Run 1, Run 2 and HL-LHC}

\author{R.~Kwee-Hinzmann, G.~Bregliozzi, R.~Bruce, F.~Cerutti,~L.~S.~Esposito, \\S.~M.~Gibson, A.~Lechner, H.~Garcia Morales, C. Yin Vallgren}
%\author{R.~Kwee-Hinzmann\thanks{Royal Holloway University of London}, S.~M.~Gibson, H.~Garcia-Morales \\
%  R.~Bruce\thanks{CERN}, F.~Cerutti,~L.~S.~Esposito, A.~Lechner
%\author{Ren\'e Q. \AE\"o\c ca\ss\ Jr.\thanks{On leave of absence from


\date{June 13, 2017}



% Now we give the text of the abstract as another declaration; this is
% different from other LaTeX styles in which the abstract is given as
% the abstract environment.
\abstract{
  The study of machine-induced background to the experiments is vital for several reasons. Too much background can be an issue for operation and the difficult part is to judge when exactly ``too much'' is attained. It is a complex topic as experiments are directly or indirectly affected by conditions all around the LHC ring e.g.~collimation settings and vacuum quality. A detailed study of background can also help understanding the machine better to identify potential issues and complemented by dedicated machine tests. Finally, such a study is relevant for the experiments to analyse the characteristics of machine background to make sure not to count it into a new physics signal.

  This report summarises simulation results of three background sources, local beam-gas, beam-halo from the betatron collimation in IR7, and for the first time beam-halo from momentum collimation in IR3. Two of the most dominant sources, local beam-gas and betatron halo, have been systematically studied for LHC Run~1 and Run~2 cases, and several HL--LHC scenarios. We analyse the evolution of background of these scenarios and discuss rates expected for HL-LHC. Impacts on the tertiary collimators upstream of ATLAS and CMS are analysed, while particle distributions streaming into the experimental area were estimated for ATLAS only in detailed shower simulations in all of these scenarios.


% The following sort of thing can be conveniently included
% as part of the abstract.
% Don't leave any blank lines within the abstract declaration.
%
  %% \begin{center}\textsl{Run~I to HL--LHC background \\
  %%     simulation studies}
  %%         \end{center}
         }  %%% this is the end of the abstract declaration

% N.B. Nothing has been printed so far, we've only made declarations.


\begin{document}           % End of preamble and beginning of text.


\maketitle                 % Produces the title page.

% Some or all of the following may be useful in a long report
%     \tableofcontents
%     \listoffigures
%     \listoftables
%     \newpage
\tableofcontents
\newpage
\newpage

% uncomment to also show section header
%\addtocontents{toc}{\protect\setcounter{tocdepth}{1}}
%-------------------------------------------------------------------------------------------
\section{Introduction}


The LHC has finished its first run and is half way through its second run of operation. Proton-proton collisions collected by the two large experiments ATLAS and CMS have led to the long thought discovery of the Higgs particle~\cite{Aad20121,Chatrchyan201230}. To extend the discovery potential of the LHC, an upgrade is planned with even higher luminosities, HL-LHC~\cite{hl-lhc-prel-design}, and it is planned to take up major constructions in 2024.

In this report particle losses and background sources to experiments are discussed in all of these scenarios ranging from 4 to 6.5 TeV in LHC to 7~TeV in HL-LHC using different collimator settings and beam optics. Simulations have been performed for two of the most important sources of beam-induced background using new, improved techniques. Any interaction of beam particles with anything upstream the interaction point (IP), causing shower towards the experiment, can contribute to background.

Depending on how often signals from such interactions appear they are judged either critical or managable for operation. Often it is hard to precisely predict the rate as the it depends on machine conditions and settings, like the vacuum quality and settings of the collimators, which define the leakage to the experimental areas. Certainly, also very basic beam properties are crucial for background, e.g.~the beam life time directly correlates with the rates observable at the experiments and the lifetime is not easy to control or predict. There were also luminosity-dependent phenomena observed. In such a complex machine like the LHC with various filling schemes and different beam optics to meet various physics purposes at increasing beam energies, there is a good chance to encounter background phenomena which have not been well known from other machines, like in ATLAS where ``after-glow" or ``ghost-charge"~\cite{ATLAS_JINST_13} were observed. That is why the experiments, even if the background is judged to be not critical for operation in the current run, are motivated to study their origin and characteristics. Certain sources may become an issue in the future when machine parameters change. As experiments would like to analyse rare interaction processes, also background that does not appear at a high rate is of high interest. This enables them to study e.g.~how many high energy particles reach which part of the detector. This is vital for them to substract the background from signals that may mean a new particle. 

One can differentiate between several background sources depending on how they are created. Once typical characteristic distributions are known, it is a matter of normalisation which accounts for the specific running conditions to determine the rate. In this report, we present both, the study of shapes of background distributions at a virtual machine-detector interface plane and highlight rate estimates of two of the most important sources close by that can create background at the experiments. We discuss three sources here.

A well-known source is if beam protons hit residual gas molecules inelastically upstream and in the vicinity of the experiment producing a shower of secondaries and unwanted signals in the detector. The rate of such beam-gas events is tightly connected to the vacuum quality and thus has to be permanently monitored by both, the experiments and machine operators.

Another source of steady losses in the machine are halo protons diffusing out to high transverse amplitudes (they form a halo of the beam). Eventually they get lost before they reach one of the interaction points (IPs). In order to avoid losses in cold areas (where they can cause magnets to quench when depositing too much energy in the supraconductiong coils) an entire cleaning insertion in IR7 is dedicated to these unavoidable steady losses~\cite{LHCDesignRep,assmann05chamonix}. Halo particles escaping IR7 are even further reduced by tertiary collimators (TCTs) close to the experiments, each IR where the bunches cross, is equipped with such collimators. Anything that leaks from this system can produce a background shower for the experiment when interacting with the collimator material. 

A very similar system is installed in IR3 to clean off-momentum particles. These particles, when they deviate slightly from the nominal beam energy, will be bent slightly differently and run on a different orbit given by the dispersion function. The leakage out of IR3 impacts as well on the TCTs in the experimental IRs and contribute to background at the experiments. They had never been simulated as background before, we report here for the first time on their characteristics and how they compare to the other two background sources of local beam-gas and betatron halo. 

Shower distributions are evaluated at an artificial interface plane between the machine and detector, which is defined at 22.6~m from IP1. This is the main output of the studies presented in this report, and can be used as input for further background studies by the experiments within the detectors. We either highlight general particle distributions or focus on muons as experiments are usually interested in these particles. The IR1 layout of the beamline is identical to the one in IR5, but there are other differences, like the path length from cleaning insertions to IP1, the crossing plane is chosen horizontally in IR5 and vertically in IR1 (but the crossing angle in IR1 and IR5 are the same). Therefore, we study also collimator losses in IR5.

We also present details of several scenarios simulated with the final HL-LHC collimation layout. We investigate how much the shapes of particle distributions at the interface plane differ from the present LHC and what can be expected in terms of rates. As the stored beam energy increases from 360~MJ to 675~MJ, all scenarios of beam losses are more critical and one has to review dominant sources. This report highlights the evolution of background sources including future trends for the upgrade of the LHC.

%In Sect.~\ref{simSetup} we described the simulation techniques used in Sect.~\ref{run1run2} for the results for Run~I and Run~II as well as for scenarios in HL--LHC in Sect.~\ref{hllhcResults}. We discuss the evolution of background sources in Sect.~\ref{evolut} and give conclusions in Sect.~\ref{last}. 

\section{Simulation Techniques and Physics Scenarios\label{simSetup}}

The simulation of background in the LHC can be very demanding in terms of CPU as detailed shower simulations have to be performed along several hundred meters or many millions of primaries have to be tracked to get sufficient particles leaking to collimators in IR1 or IR5 which by definition is many orders of magnitudes lower than protons hitting the primary collimators. Two simulation codes were used as main tools: SixTrack for particle tracking around the ring and \fluka~for particle showering. 

The layout of nominal LHC is shown in Fig.~\ref{nominalLHC_layout} for intersertion region 1 (IR1) with the interaction point 1 (IP1) at s~=~0. By design, the left and the right side are symmetric for incoming and outgoing beams. For our purposes, shower evaluation from different background sources, it is sufficient to use only half of the layout for the incoming beam. 

%Note, that usually the distributions are indicated per interaction, for beam-halo (BH) simulations distributions are shown per hit in the tertiary collimators and for beam-gas (BG) it is given per proton-nitrogen interaction.

\subsection{Beam-Halo simulations}
The simulation is performed in two steps: First, we use SixTrack~\cite{SixTrackRef} for particle tracking in order to obtain the leakage onto the TCTs in the IRs. To be CPU efficient, only a halo proton distribution is tracked not the full beam with its core that is usually many orders of magnitude more dense. Tracking is performed through a magnetic field lattice, usually prepared using MadX, and the LHC collimatiors of which only the jaws are modeled. We also assume a perfect machine, neglecting any uneven jaw surfaces or edges\footnote{Previous studies REFERENCE have shown if one includes machine errors, the results differ by a factor of about 2.}.

The beam halo is usually simulated in horizontal (h) and vertical (v) distributions, these are flat in one plane and Gaussian in the other (for more information see also~\cite{chiarasThesis}). When a collimator is hit, a built-in, recently updated Monte-Carlo model~\cite{claudiasThesis} decides on the physics process. Protons continue in the lattice until they dissociate in an inelastic interaction with the collimator material or (in a post-processing step) are lost on the aperture. As a result, loss locations around the ring can be identified and protons absorbed by the TCTs serve in second step as seeds in \fluka~\cite{flukaRef1,flukaRef2}~to evaluate particle showers streaming towards the detector. Particle interaction and transport are calculated within matter for a user-defined geometry. It uses modern physics models and experimental data as available by the Particle Data Group~\cite{pdgRef}. The geometry was built up to 546~m of the right side of IR1 for the present LHC machine, for HL-LHC scenarios the geometry reaches up the tertiary collimators of around 215~m including the long straight section of IR1 (LSS1).

When halo protons interact with the jaw material of the TCTH or TCTV shower particles are created and stream towards the experiment. Every particle passing an imaginary (x,y)-plane at 22.6~m away from the IP, essentially between the triplet magnets and the TAS, is recorded and written out. The left plot of Fig.~\ref{tctHits} shows part of the \fluka~geometry and in green the hits in an horizontal collimator TCTH.4L1 with absorbed protons from an horizontal and vertical halo distribution. On the right of Fig.~\ref{tctHits} one can see how deep these positions are with respect to the jaw surface shown for the collimator pair of IR1. This depth distribution is shown here as an example and discussed in more detail together with the other simulation cases in Sec.~\ref{evolut}.

\begin{figure}%[!htb]
\begin{center}
\includegraphics[width=0.9\textwidth]{figures/IR1_layout_runII.pdf}
\end{center}
\vspace{-0.6cm}
 \caption{Machine layout for the nominal LHC (as in Run I and II) of the left side of IR1 with IP1 at s = 0 for the incoming beam. Highlighted are the tertiary collimators (TCT4) at around -147~m, TCLs are debris collimators for the outgoing beam.
  \label{nominalLHC_layout}}
\end{figure}


%% \begin{figure}
%% \begin{center}
%% \includegraphics[width=0.48\textwidth]{figures/phasespace_d2}
%% \includegraphics[width=0.48\textwidth]{figures/realspace_d2}
%% \end{center}
%% \caption{Initial halo distribution of type 2 in phase-space (left) and real space (right) is generally used in horizontal halo simulations with SixTrack.
%% \label{haloExamples}}
%% \end{figure}


\begin{figure}%[!htb]
\begin{center}
\includegraphics[width=0.9\textwidth]{figures/IR1_interfaceplane.pdf}
\end{center}
\vspace{-0.6cm}
 \caption{x-z cut of \fluka~geometry zoomed into the interface region between detector and machine. A virtual plane at 22.6 m is used to evaluate particle showers. Each color defines a different material.
  \label{flukaGeo_nominal}}
\end{figure}


\begin{figure}%[!htb]
\begin{center}
  \includegraphics[width=0.4\textwidth]{figures/6500GeV/xz_6500GeV_b1_TCT4.pdf}
  \includegraphics[width=0.4\textwidth]{figures/6500GeV/inelposition_sum_HALOB1.pdf}
\end{center}
\vspace{-0.6cm}
 \caption{View in the (x,z)-plane of hits in TCTH from the Run~II halo simulation case zoomed into the inner collimator parts. The hits are all contained within the collimator jaws and were obtained by SixTrack and transformed to the positions of the TCT in the \fluka~coordinates.
  \label{tctHits}}
\end{figure}


\subsection{Beam-Gas simulations}
Beam-gas interactions were simulated for Run I and Run II cases with \fluka~using a detailed geometry up to the arc until 546~m away from the IP. %, but with different magnetic fields for a 4 TeV and a 6.5 TeV proton beam and the respective optics.
The difference for the Run I and Run II geometry is merely in the generation of the magnetic fields for 4 TeV and 6.5 TeV respectively, the method of sampling beam-gas is the same. In \fluka, protons were forced to undergo an inelastic interaction with residual gas molecules on positions on the ideal orbit. A new technique used here includes the variation of the transverse beam size which in particular is large inside the triplet when the beam is squeezed final focussing. 
The sampling positions were read into \fluka~and like for the beam-halo simulations each shower particle reaching the interface plane at 22.6~m is used to dump information about fluxes towards the experiment. 

The normalisation to a real beam-gas rate can either be done at run-time or in a subsequent step if one keeps the information of the initial z-coordinate of the interacting proton per particle at the interface plane. The advantage of the second method is that distributions can be easily re-normalised to another pressure profile. More details are given directly per simulation case in Sect.\ref{run1run2}. For HL, the studies were already shown in Ref.~\cite{ipac2014_rkh} and they are still the most recent of this kind.% The normalisation with the intensity and pressure profile was done as follows: A representative fill of the run was selected and gauge data is used in the pressure simulation in order to interpolate in between the measurement points and also estimate the contribution per molecule type using the VAcuum Simulation COde, VASCO. The result file can then be re-weighted with the pressure in a subsequent step, thus the unweighted results correspond to a flat pressure. 

\subsection{New simulation techniques}
Previous studies as in Ref.~\cite{nimPaperRod} used methods relaying on approximations of either of the beam or on its trajectory. One approximation is that the transverse beam size was neglected. In particular, just before the triplet the beamsize is very large and addtional interactions in that location could contribute to shower particles at the interface plane. Another improvement in the simulations is the inclusion of the crossing-angle which was present right from start-up of the LHC.

\subsubsection{Setup to include the beam size}
To test the effect of the beam size, two cases were simulated in \fluka, with exactly the same setup for the 2012 Run I scenario, see Table~\ref{paramsRun12}, but using a different input file for positions at which beam-gas interactions are sampled.

The new input file was created by dumping positions of the trajectory with different starting positions. The ideal orbit goes through (0,0) in (x,y) at the IP. Assuming a gaussian distribution of the beam particles one can produce matched phase space coordinates in the transverse plane, as shown in Fig.~\ref{ip1_gauss}, at the IP where the optical functions are $\alpha = 0$, $\beta = \beta^* = 60$~cm. Using a normalised coordinate system, the phase space coordinates were calculated as in Eq.~\ref{eq1} and used as initial seeds in \fluka~to create the trajectory. 1000 trajectories were created and randomly 10 sampling positions per longitudinal coordinate were chosen. The final input file to \fluka~is visualised in Fig.~\ref{BGASflukaInp}. One would expect the beam size to be largest after the D1 at the entrance of the triplet in Fig.~\ref{ip1_gauss}~(a). Looking back to Fig.~\ref{nominalLHC_layout} we identify first a squeeze in x (red bottom box), then in y (purple top box) and then in x again and indeed this is also what we see if the beamsize is calculated theoretically using the MadX $\beta-$functions\footnote{given by $\sigma_{x,y} = \sqrt{\epsilon_{geo} \cdot \beta_{x,y}}$ with $\epsilon_{\textrm{geo}} = \frac{ \epsilon_{\textrm{n}}}{\gamma_{\textrm{rel}}}$} as shown in Fig.~\ref{twissfileBS}.

\begin{equation} \label{eq1}
  \begin{split}
x = & \, \sqrt{\beta \epsilon} \cdot X \\
x' = & \sqrt{\frac{\epsilon_{geo}}{\beta}} \, \big( X' - \alpha X \big)
  \end{split}
\end{equation}

with $\epsilon$ being the geometric emittance, $\alpha, \beta$ and $\gamma$ the usual twiss parameters from the definition of the emittance as conservative in $\epsilon = \gamma x^2 + \beta x'^2 + 2 \alpha x x'$, and $X$ and $X'$ satisfying the circle equation, $X^2 + X'^2 = 1$. 


\begin{figure}%[!htb]
\begin{center}
\includegraphics[width=0.9\textwidth]{figures/IP1_gauss.pdf}
\includegraphics[width=0.9\textwidth]{figures/twiss_gauss.pdf}
\end{center}
%% \begin{picture} (0.,0.)
%% \setlength{\unitlength}{1.0cm}
%% \small{
%%     \put ( 4.,7.35){(a)}
%%     \put ( 12.4,7.35){(b)}
%%     \put ( 4.,1.){(c)}
%%     \put ( 12.4,1.){(d)}}
%% \end{picture}
\vspace{-0.6cm}
 \caption{Matched phase space coordinates at IP1 in x and y (top) and at an example position (bottom, here TCTH). The rings indicate in $\sigma$ the gaussian distribution.
  \label{ip1_gauss}}
\end{figure}


\begin{figure}[!htb]
\begin{center}
\includegraphics[width=0.44\textwidth]{figures/inputFluka6500GeV_yBGAS.pdf}
\includegraphics[width=0.44\textwidth]{figures/xBGAS10z1.pdf}
\end{center}
\begin{picture} (0.,0.)
\setlength{\unitlength}{1.0cm}
\small{
    \put ( 4.,1.){(a)}
    \put ( 12.4,1.){(b)}
}
\end{picture}
\vspace{-0.6cm}
 \caption{Positions as sampled in \fluka~with variations of the transverse beam size vertically (a) shown for full range until the arc and horizontally (b) shown for the first 85~m from the IP without arc.
  \label{BGASflukaInp}}
\end{figure}

%% \begin{figure}[!htb]
%%   \begin{center}
%%     \includegraphics[width=0.442\textwidth]{figures/6500GeV/x_MQXA_3R1}
%%     \includegraphics[width=0.442\textwidth]{figures/6500GeV/y_MQXA_3R1}
%% %    \includegraphics[width=0.32\textwidth]{figures/6500GeV/yp_MQXA_3R1}
%% \end{center}
%% \vspace{-0.6cm}
%%  \caption{Comparison of the calculated beam size by MadX (left corner) and gaussian fit parameters (right corner) of input sample for the angle $x$ and $y$ at an example position inside the triplet.
%%   \label{bgFitCheck}}
%% \end{figure}

%% %Muons, Protons
A global view on particle distributions in $\phi$ shows that the former peaks, in particular when looking at Fig.~\ref{bsRatioPhiAll} in the vertical plane at $\pm\frac{\pi}{2}$ is ``washed out'' but else no major differences can be observed. 

The distributions per species, muons and protons, are shown in Fig.~\ref{bsRatioPhiMP} and one can observe clearly in the proton distributions that the shoulders are a little wider and instead the peak is not as high as when no beam size was taken into account. For muons one can conclude it does not change anything significantly.

Having a closer look to proton distributions we investigate if an increase is visible where the beam size is indeed large just downstram the interface plane, from 22.6 to 59~m. Comparing the shapes per z-region of all particles and protons only one can see most of the energy comes from protons as visible from Fig.~\ref{bsZ2}. A direct comparison of the shapes in Fig.~\ref{bsZ} reveals all differences are almost entirely due to protons.

\begin{figure}%[!htb]
\begin{center}
  \includegraphics[width=0.85\textwidth]{figures/twiss_b1_sigma_IR1Right_4TeV.pdf}
\end{center}
\vspace{-0.6cm}
 \caption{Beam sizes in IR1 in horizontal and vertical plane. The red lines indicate the longitudinal s-sections: from 22.6~m to 59~m it contains triplet, 59~m to 153~m is where the beampipes are split and the D1 sits, in 153--269~m is the D2 and goes up to the end of the LSS1 and at 269~m the arc starts and is shown up to 550~m.
  \label{twissfileBS}}
\end{figure}

\subsubsection{Simulations with crossing angle}
The motivation to introduce a crossing angle in the machine is to avoid parasitic interactions of the beam while they travel in the same beam pipe in the interaction region. A small crossing angle allows for a quasi head-on collision of two bunches while other bunches are kept separated. The amount of the crossing angle is given by beam-beam effects which one wants to suppress and is trade-off between maximising luminosity and keeping the beam stable. The plane in which the angle is introduced is chosen such that one can compensate partially another long-range beam-beam effect resulting in either a positive or negative tune shift. While in IR1 the crossing angle is in the vertical plane, it is in the horizontal plane in IR5.

In all simulations from 4~TeV onwards, it was possible in \fluka~to consider such a crossing angle.

\subsection{Off-momentum particles simulations}

%% \begin{equation}
%%   \delta = \frac{\Delta \mathrm{p}}{\mathrm{p}} = 1.6 \, \,10^{-3}
%% \end{equation}


\subsection{Run I and Run II simulation cases}
For Run I and II, real physics configurations of the LHC were used for most of the $pp$ physics runs in 2012 and 2015. At 4~TeV in 2012, the optics were for a $\beta^* = 60$ cm and TCT collimator settings in IR1/IR5 TCTs were set to 9~$\sigma$~\cite{parametersRun1}. For Run II, the optics changed to $\beta^* = 80$~cm which was used in the machine throughout 2015 for proton-proton collisions and IR1/IR5 collimators were set to 13.7~$\sigma$. For both runs, a vertical crossing angle of $290~\mu$m was taken into account in the simulations. More simulation and run parameters can be found in Tab.~\ref{paramsRun12}. 

\begin{table}
   \centering
   \caption{Run I (2012) and Run II (2015) simulation parameters.}
   \begin{tabular}{l||c|c}
       \hline
       beam energy & 4 TeV & 6.5~TeV \\
       $\beta^*$ optics  & 60~cm &  80~cm \\
       bunch intensity & 1.4$\times 10^{11}$ protons &  1.2$\times 10^{11}$ protons\\
       number of bunches & 1380 & 2748\\
       bunch spacing & 50~ns & 25~ns\\
       half-crossing angle IP1~/~5 & 145~$\mu$rad & 145~$\mu$rad \\
       TCP.IR7~/~TCSG.IR7~/~TCT.IR1 & 4.3~/~6.3~/~9.0~$\sigma$ & 5.5~/~8.0~/~13.7~$\sigma$ \\
       \hline
   \end{tabular}
   \label{paramsRun12}
\end{table}


\subsection{HL-LHC simulation cases}

Several cases were simulated in order to characterise the cleaning efficiency for baseline settings of HL-LHC and variants and to estimate quantitatively any advantages in terms of background with different collimator layouts. An alternative set of collimator openings, so called \twosigmaret~settings, was prepared based on experience with the machine and is listed in Tab.~\ref{HLcollSettings}. In particular, the aim was to quantify the effect of additional tertiary collimators, TCT5s, for incoming beams (B1 and B2). Inelastic interactions with beam protons are forced in \fluka~at initial conditions given by SixTrack on the TCT4s and (when included) TCT5s. These interactions generate a particle flux towards the experiment. All shower particles are recorded at the machine-detector interface plane at 22.6~m from the IP using in \fluka~a production and transportation cut-off at 20 MeV. Section~\ref{hllhcResults} is dedicated to all the HL simulation details.

\begin{figure}%[!htb]
\begin{center}
\includegraphics[width=0.9\textwidth]{figures/IR1_layout_HL.pdf}
\end{center}
\vspace{-0.6cm}
 \caption{Machine layout for the HL-LHC for the incoming beam in IR1 with IP1 at s = 0. Highlighted are the horizontal and vertical tertiary collimators (TCT4s) at around -131~m, the new pair of tertiaries TCT5s at around -213~m.
  \label{hllhc_layout}}
\end{figure}


 \begin{table}[hbt]
   \centering
   \caption{HL half-gap collimator settings calculated for a normalised emittance of $\epsilon_{\mathrm{n}}$ of 3.5~$\mu$m. Full and updated settings can be found in~\cite{collSettRef}. When included, the TCT5s had the same settings as the TCT4s.}

   \begin{tabular}{l|c|c}
       \hline
       collimators &        nominal settings & $2\sigma$-retracted settings\\
                   &         [$\sigma$] &  [$\sigma$]\\
       \hline
       TCP3 & 12 (now 15) & 15 \\
       TCSG3 & 15.6 (now 18)& 18 \\
       TCP7 & 6 & 5.7 \\
       TCSG7 & 7 & 7.7 \\
       TCT4 IR1/5 & 8.3 & 10.5 \\
       \hline
   \end{tabular}
   \label{HLcollSettings}
\end{table}

\begin{table}[!hbt]
   \centering
   \caption{Simulation parameters for HL-LHC normalisation for $\beta^* =$15~cm ATS optics.}
   \begin{tabular}{l|c}
       \hline
       beam energy & 7 TeV \\
       bunch intensity & 2.2$\times 10^{11}$ protons\\
       bunch spacing & 25~ns \\
       number of bunches & 2736 \\
       \hline
   \end{tabular}
   \label{hlscenario}
\end{table}

\section{Simulation Results for LHC's Run I and Run II}

The two background sources were investigated for several run scenarios. During Run I, we had two beam energies, 3.5 TeV in 2010--11 and 4~TeV in 2012\footnote{A detailed overview of parameters in Run 1 can be found in~\cite{ParametersRun1}} While the 3.5~TeV background data has been presented in detail in~\cite{nimPaperRod}, we focus on further developments made since that paper and use these improvements for 4 TeV and Run II simulation cases. Beam-halo simulations were improved compared to~\cite{nimPaperRod} by considering a crossing angle in the simulation as it has been present in the machine. Another improvement concerns beam-gas simulations for which the transverse beam size has been taken into account. 
\subsection{Sensitivity Studies}
We investigate how the additional improvements, i.e.~the inclusion of the crossing angle and beam size, effect previous simulations.

\subsubsection{Crossing Angle}
The motivation to introduce a crossing angle in the machine is to avoid parasitic interactions of the beam while they travel in the same beam pipe in the interaction region. A small crossing angle allows for a quasi head-on collision of two bunches while other bunches are kept separated. The amount of the crossing angle is given by other beam-beam effects which one wants to suppress and is trade-off between maximising luminosity and keeping the beam stable. The plane in which the angle is introduced is chosen such that one can compensate partially another long-range beam-beam effect resulting in either a positive or negative tune shift. While in IR1 the crossing angle is in the vertical plane, it is in the horizontal plane in IR5.

We can study the crossing angle effect on background in IR1 using the 3.5~TeV halo simulation data of~\cite{nimPaperRod} which did not have any crossing angle included and the 4 TeV halo data with a crossing angle of 290~$\mu$rad. For this comparison, we plot the distributions at the interface plane per TCT hit. 


\begin{figure}[!htb]
\begin{center}
\includegraphics[width=0.495\textwidth]{figures/}

\end{center}
\begin{picture} (0.,0.)
\setlength{\unitlength}{1.0cm}
\small{
    \put ( 4.,1.){(a)}
    \put ( 12.4,1.){(b)}
}
\end{picture}
\vspace{-0.6cm}
 \caption{$\phi$ distribution of .
  \label{compTCT5INOUT}}
\end{figure}



\subsubsection{Beam Size}

\subsection{Run I: 4 TeV Beam-Halo}
\subsection{Run I: 4 TeV Beam-Gas}

\subsection{Run II: 6.5 TeV Beam-Halo}
\subsection{Run II: 6.5 TeV Beam-Gas}

\section{Simulation Results for HL--LHC\label{hllhcResults}}

%Several cases were simulated in order to characterise the cleaning efficiency for baseline settings of HL-LHC, deploying different collimator layouts (in IR1/5 TCT4s only and TCT4s + TCT5s) and alternative collimator settings, so called \twosigmaret~settings to quantify the effect of the TCT5s for incoming beams (B1 and B2). Inelastic interactions with beam protons are forced in FLUKA at initial conditions given by SixTrack on the TCT4s and (when included) TCT5s. These interactions generate a particle flux towards the experiment. All shower particles are recorded at the machine-detector interface plane at 22.6~m from the IP using in FLUKA a production and transportation cut-off at 20 MeV.

This section describes the several case studies for HL, focussing on beam-halo. As mentioned earlier, the HL beam-gas simulations as described in Ref.~\cite{ipac2014_rkh} are still the most recent of this kind.

\subsection{Beam-Halo}

The aim was to study loss locations for different collimator layouts and settings and evaluate quantitatively on halo background in IR1. The simulation technique as described in Sec.~\ref{simSetup} was performed in detail for the more realistic collimator settings scenario, so-called \twosigmaret~settings, and all cases are listed in Tab.~\ref{hlscenario}. The baseline collimator layout contains the TCT4s and TCT5s, although the position may not be fixed to the meter precise for the moment due to lack of sufficient space at the positions as shown in Fig.~\ref{hllhc_layout}. 

Two different HL optics (version HLLHCv1.0\footnote{The FLUKA geometry contained layout updates of v1.1, but v1.0 collimator settings were used assuming the uncertainties from the version differences are negligable.}) were studied:
The current baseline of ATS optics for $\beta^{*}$ of 15~cm for round beams is used in the simulations.
Another ATS optics scenario was also studied with the purpose of validating a new HL collimation layout. To maintain flexibility in low-$\beta^*$ reach, e.g~in the case of crab cavity failures in the experimental IRs, flat beam optics were developed. While in round optics, $\beta^*$ is 15~cm in IP1/5 in the horizontal and vertical plane, it is for flat beams 7.5~cm in the horizontal plane and 30~cm in the vertical plane at IP1 (vice versa for IP5)~\cite{opticsWebRef}. 


\begin{table}%[hbt]
   \centering
   \caption{Beam-halo simulation cases in SixTrack (HLLHC v1.0).}\vskip2mm
   \begin{tabular}{|l|l|l|l|}
       \hline
       collimator settings & TCT5s & beam halo & optics \\
       \hline\hline
       updated nominal  & out & h+v B1 & round \\
       updated nominal  & in & h+v B1 & round \\\hline
       \twosigmaret & out & h+v B1 & round \\ 
       \twosigmaret & in  & h+v B1 & round \\ 
       \twosigmaret & out & h+v B2 & round \\
       \twosigmaret & in  & h+v B2 & round \\ \hline
       \twosigmaret & out  & h+v B1 & flat \\
       \twosigmaret & in  & h+v B1 & flat \\ 
%       \twosigmaret & in  & h+v B1 & sround \\ 
%       \twosigmaret & in  & h+v B1 & sflat \\ 

       \hline

   \end{tabular}
   \label{hlscenario}
\end{table}

\subsubsection{Losses at IR1/IR5 tertiary collimators in HL--LHC}

A zoom into loss locations in the experimental IRs of ATLAS and CMS are shown in Fig.~\ref{IR15_roundB1_nomSett} for nominal collimator settings. The similar set of zooms are shown and discussed for the \twosigmaret~settings in the Appendix~\ref{lossmapszooms}. The beam direction is from left to right. The two black bars at upstream of the IP are the losses on the pair of tertiary collimators (TCT4s and TCT5s), while the black bars downstream are losses on TCLs, debris collimators. These losses are normalised to total number of lost particles. One can see, IR5 losses are generally lower than in IR1, an expected feature known from Run I and II. A more direct comparison of the losses is made in Fig.~\ref{compTCT5INOUT}. The losses are normalised to the number of simulated primary per simulation case, then the sum per collimator is shown, i.e. as example the bin of a specific TCTH is set to $\big(\frac{\mathrm{hits\,on\,TCTH}}{\#\mathrm{primary}}\big)_{\mathrm{h}} + \big(\frac{\mathrm{hits\,on\,TCTH}}{\#\mathrm{primary}}\big)_{\mathrm{v}}$. The simulation cases shown are in Fig.~\ref{compTCT5INOUT}~(a) for round, and Fig.~\ref{compTCT5INOUT}~(b) for flat beams.

\begin{figure} [!htb]
\begin{center}

\includegraphics[width=0.48\textwidth]{figures/lossmaps/coll_loss_H5_HL_nomSett_hHalo_b1_IR1}
\includegraphics[width=0.48\textwidth]{figures/lossmaps/coll_loss_H5_HL_nomSett_vHalo_b1_IR1}
\includegraphics[width=0.48\textwidth]{figures/lossmaps/coll_loss_H5_HL_nomSett_hHalo_b1_IR5}
\includegraphics[width=0.48\textwidth]{figures/lossmaps/coll_loss_H5_HL_nomSett_vHalo_b1_IR5}
\end{center}
\vspace{-0.3cm}
 \caption{Zoom into IR1 (top) and IR5 (bottom, IP5 is at 133,300~m) when the TCT5s were inserted using round optics. Horizontal beam 1 is on the left, vertical beam 1 on the right.
  \label{IR15_roundB1_nomSett}}
\end{figure}



\begin{figure}[!htb]
\begin{center}
\includegraphics[width=0.495\textwidth]{figures/inelposition_sum_tct5otrd.pdf}
\includegraphics[width=0.495\textwidth]{figures/inelposition_sum_tcotrdb2.pdf}

\end{center}
\vspace{-0.6cm}
 \caption{Positions of inelastic interactions as given by SixTrack within the collimator jaws.
  \label{inelHLtct5in}}
\end{figure}


\begin{figure}[!htb]
\begin{center}
\includegraphics[width=0.495\textwidth]{figures/inelposition_sum_tct5inrd.pdf}
\includegraphics[width=0.495\textwidth]{figures/inelposition_sum_tcinrdb2.pdf}
\end{center}
\vspace{-0.6cm}
 \caption{Positions of inelastic interactions as given by SixTrack within the collimator jaws.
  \label{inelHLtct5in}}
\end{figure}



\begin{figure}[!htb]
\begin{center}
\includegraphics[width=0.495\textwidth]{figures/compTCT5LINOUT_roundthin_B1_IR1IR5}
\includegraphics[width=0.495\textwidth]{figures/compTCT5LINOUT_flatthin_B1_IR1IR5}
\end{center}
\begin{picture} (0.,0.)
\setlength{\unitlength}{1.0cm}
\small{
    \put ( 4.,1.){(a)}
    \put ( 12.4,1.){(b)}
}
\end{picture}
\vspace{-0.6cm}
 \caption{Summary of hits load of the single TCTs for the cases TCT5 are out and in for round (a, $\beta^*_{\textrm{x,y}}$ is 15~cm) and flat (b, $\beta^*_{\textrm{x/y}}$ = 7.5~cm, $\beta^*_{\textrm{y/x}}$ = 30~cm) beam optics.
  \label{compTCT5INOUT}}
\end{figure}

\begin{figure}[tbh]
    \centering
    \includegraphics[width=0.5\textwidth]{figures/TUPTY067f3}
    \vspace{-0.5cm}
    \caption{Comparison of losses on TCTs for round and flat beam optics.}
    \label{compOptics}
\end{figure}

\subsubsection{Comparison of round and flat beam optics}
We focus on the different loads of TCT4s and TCT5s in IR1 and IR5 and are interested in how the TCT hits are shared amongst them. Both cases, having TCT5s in and out, are compared for B1 round beam optics in IR1 and IR5 in Fig.~\ref{compTCT5INOUT}~(a). One can observe as expected that the TCT5s take over a large fraction of halo protons when they are included. In return, nearly a factor 5 less load on the TCT4s in IR1 can be expected. However, considering the losses of TCTH.4 from Fig.~\ref{compTCT5INOUT}~(a), there are a factor 4 less hits when TCT5s are in IR1, and about a factor 6 in IR5. This gain will be decreased by contributions from TCT5s. Even more significant is the difference in IR5 with about 25 times less losses at the TCT4s. We also find that B1 losses are smaller in IR5 than in IR1. This can be expected for B1 since the halo protons that leak through the cleaning system of IR7 have a much shorter distance to travel to IR1 than to IR5. We also note that a slight increase of about 8~\% of intercepted losses is found when both TCT4s and TCT5s are deployed. Since more particles are intercepted, more shower particles can be created which also affects the background level. %Detailed shower simulations with FLUKA provide a first estimate of that background reduction.


\subsubsection{Effect of TCT5s on halo background in IR1}

%We studied also how the TCT load changes when the optics change from round to flat for the case that the TCT5s are included. The result is shown in Fig.~\ref{compOptics} for the separate TCTs in IR1/5. In the figure, one can see per IR, that the number of hits are very similar for round and flat beam except for TCTV.4 in IR1 where a flat B1 would create about a factor 4 more hits. When summing the hits of TCT4s and TCT5s, the number of TCT hits are very similar and so possibly is also the background level in IR1 and potentially even slightly better in IR5.

% ------------------------------------------------------------------------------------------
% 

\begin{figure}
\begin{center}
\includegraphics[width=0.43\textwidth]{figures/HL/tct5inrd/Ekin_BH_HL_tct5inrdB1_20MeV.pdf}
\includegraphics[width=0.43\textwidth]{figures/HL/tct5inrd/PhiEnDist_BH_HL_tct5inrdB1_20MeV.pdf}
\includegraphics[width=0.43\textwidth]{figures/HL/tct5inrd/RadNDist_BH_HL_tct5inrdB1_20MeV.pdf}
\includegraphics[width=0.43\textwidth]{figures/HL/tct5inrd/RadEnDist_BH_HL_tct5inrdB1_20MeV.pdf}
\end{center}
%% \begin{picture} (0.,0.)
%% \setlength{\unitlength}{1.0cm}
%% \small{
%%     \put ( 4.,7.35){(a)}
%%     \put ( 12.4,7.35){(b)}
%%     \put ( 4.,1.){(c)}
%%     \put ( 12.4,1.){(d)}
%% }
%% \end{picture}
\vspace{-0.6cm}
 \caption{Particle distribution at the interface plane.}
  \label{tct5inrdb1retr}
\end{figure}


\begin{figure}
\begin{center}
\includegraphics[width=0.495\textwidth]{figures/OrigYZMuons_BH_HL_tct5otrdB1_20MeV}
\includegraphics[width=0.495\textwidth]{figures/OrigYZMuonsE100_BH_HL_tct5otrdB1_20MeV}
\includegraphics[width=0.495\textwidth]{figures/OrigYZMuons_BH_HL_tct5inrdB1_20MeV}
\includegraphics[width=0.495\textwidth]{figures/OrigYZMuonsE100_BH_HL_tct5inrdB1_20MeV}
\end{center}
\begin{picture} (0.,0.)
\setlength{\unitlength}{1.0cm}
\small{
    \put ( 4.,7.35){(a)}
    \put ( 12.4,7.35){(b)}
    \put ( 4.,1.){(c)}
    \put ( 12.4,1.){(d)}
}
\end{picture}
\vspace{-0.6cm}
 \caption{Origin of muons for all energies (a,c) and for an energy above 100~GeV (b,d) in the y-z plane with TCT5 out (a,b) and TCT5 in (c,d).
  \label{OrigMuonE}}
\end{figure}

% ------------------------------------------------------------------------------------------
% comparisons
In a further simulation step, we evaluate the actual change in particle flux at the interface plane in IR1 using \fluka. Although the load on TCT4s is reduced by about a factor 4 for round B1 optics, shower particles created at TCT5s contribute to halo-induced background as well. We show in Fig.~\ref{tct5inrdb1retr} the case \twosigmaret~settings and TCT5s included. We compare to the case with the same collimator settings but TCT4s only in Fig.~\ref{compTCT5inoutEkin} the energy spectra for all particles and muons only reaching the interface location. Their transversal radial distribution is shown in Fig.~\ref{compRadN}. One can see in the ratio of the top plot of Fig.~\ref{Ekin} that all particles will be reduced except those with an energy reaching the beam energy when TCT5s are also installed. The integral ratio of the total number of particles indicates that close to 2 times less particles reach the interface plane when the TCT5s are in. The bottom plot of Fig.~\ref{Ekin} shows the number of muons with an energy of 100~GeV decreases, however the number of higher energy muons will rather increase by about 20~\% and even more. The lower bins of Fig.~\ref{compRadN} represent the space close to the beampipe. They will be up to a factor 3 less populated (almost entirely due to photons and electrons, not shown) if the TCT5s are in. The bottom plot of Fig.~\ref{compRadN} shows that muon distributions and their ratios feature a very similar shape.

\begin{figure}
\begin{center}
\includegraphics[width=0.43\textwidth]{figures/HL/compINOUTB1_retracted/perTCThit/ratioEkinAll}
\includegraphics[width=0.43\textwidth]{figures/HL/compINOUTB1_retracted/perTCThit/ratioEkinMuons}
\end{center}
\begin{picture} (0.,0.)
\setlength{\unitlength}{1.0cm}
\small{
    \put ( 4.,1.){(a)}
    \put ( 12.4,1.){(b)}
}
\end{picture}
\vspace{-0.6cm}
 \caption{Energy distribution for all particles (a) and muons (b) at the interface plane.
  \label{compTCT5inoutEkin}}
\end{figure}



\begin{figure}
\begin{center}
\includegraphics[width=0.43\textwidth]{figures/HL/compINOUTB1_retracted/perTCThit/ratioPhiNAll}
\includegraphics[width=0.43\textwidth]{figures/HL/compINOUTB1_retracted/perTCThit/ratioPhiNMuons}
\includegraphics[width=0.43\textwidth]{figures/HL/compINOUTB1_retracted/perTCThit/ratioPhiEnAll}
\includegraphics[width=0.43\textwidth]{figures/HL/compINOUTB1_retracted/perTCThit/ratioPhiEnMuons}
\end{center}
%% \begin{picture} (0.,0.)
%% \setlength{\unitlength}{1.0cm}
%% \small{
%%     \put ( 4.,1.){(a)}
%%     \put ( 12.4,1.){(b)}
%% }
%% \end{picture}
\vspace{-0.6cm}
 \caption{Azimuthal distribution of all particles and muons (top) and their energy (bottom).
  \label{compPhi}}
\end{figure}

\begin{figure}
\begin{center}
\includegraphics[width=0.43\textwidth]{figures/HL/compINOUTB1_retracted/perTCThit/ratioRadNAll}
\includegraphics[width=0.43\textwidth]{figures/HL/compINOUTB1_retracted/perTCThit/ratioRadNMuons}
\includegraphics[width=0.43\textwidth]{figures/HL/compINOUTB1_retracted/perTCThit/ratioRadEnAll}
\includegraphics[width=0.43\textwidth]{figures/HL/compINOUTB1_retracted/perTCThit/ratioRadEnMuons}
\end{center}
%% \begin{picture} (0.,0.)
%% \setlength{\unitlength}{1.0cm}
%% \small{
%%     \put ( 4.05,8.95){(a)}
%%     \put ( 12.45,8.95){(b)}
%%     \put ( 4.05,1.){(c)}
%%     \put ( 12.45,1.){(d)}
%% }
%% \end{picture}
\vspace{-0.6cm}
 \caption{Tranverse radial distribution of all particles and muons (top) and their energy (bottom).
  \label{compRad}}
\end{figure}

\subsubsection{Comparing layout for B2}
\begin{figure}
\begin{center}
\includegraphics[width=0.43\textwidth]{figures/HL/compINOUTB2_retracted/perTCThit/ratioEkinAll}
\includegraphics[width=0.43\textwidth]{figures/HL/compINOUTB2_retracted/perTCThit/ratioEkinMuons}
\end{center}
\begin{picture} (0.,0.)
\setlength{\unitlength}{1.0cm}
\small{
    \put ( 4.,1.){(a)}
    \put ( 12.4,1.){(b)}
}
\end{picture}
\vspace{-0.6cm}
 \caption{Energy distribution for all particles (a) and muons (b) at the interface plane.
  \label{Ekin}}
\end{figure}



\begin{figure}
\begin{center}
\includegraphics[width=0.43\textwidth]{figures/HL/compINOUTB2_retracted/perTCThit/ratioPhiNAll}
\includegraphics[width=0.43\textwidth]{figures/HL/compINOUTB2_retracted/perTCThit/ratioPhiNMuons}
\includegraphics[width=0.43\textwidth]{figures/HL/compINOUTB2_retracted/perTCThit/ratioPhiEnAll}
\includegraphics[width=0.43\textwidth]{figures/HL/compINOUTB2_retracted/perTCThit/ratioPhiEnMuons}
\end{center}
%% \begin{picture} (0.,0.)
%% \setlength{\unitlength}{1.0cm}
%% \small{
%%     \put ( 4.,1.){(a)}
%%     \put ( 12.4,1.){(b)}
%% }
%% \end{picture}
\vspace{-0.6cm}
 \caption{Azimuthal distribution of all particles and muons (top) and their energy (bottom).
  \label{compPhi}}
\end{figure}

\begin{figure}
\begin{center}
\includegraphics[width=0.43\textwidth]{figures/HL/compINOUTB2_retracted/perTCThit/ratioRadNAll}
\includegraphics[width=0.43\textwidth]{figures/HL/compINOUTB2_retracted/perTCThit/ratioRadNMuons}
\includegraphics[width=0.43\textwidth]{figures/HL/compINOUTB2_retracted/perTCThit/ratioRadEnAll}
\includegraphics[width=0.43\textwidth]{figures/HL/compINOUTB2_retracted/perTCThit/ratioRadEnMuons}
\end{center}
%% \begin{picture} (0.,0.)
%% \setlength{\unitlength}{1.0cm}
%% \small{
%%     \put ( 4.05,8.95){(a)}
%%     \put ( 12.45,8.95){(b)}
%%     \put ( 4.05,1.){(c)}
%%     \put ( 12.45,1.){(d)}
%% }
%% \end{picture}
\vspace{-0.6cm}
 \caption{Tranverse radial distribution of all particles and muons (top) and their energy (bottom).
  \label{compRad}}
\end{figure}



\subsubsection{Comparison of background with different collimator settings}


\begin{figure}
\begin{center}
\includegraphics[width=0.43\textwidth]{figures/HL/compNomRetrCollSett/perTCThit/ratioEkinAll.pdf}
\includegraphics[width=0.43\textwidth]{figures/HL/compNomRetrCollSett/perTCThit/ratioPhiEnAll.pdf}
\includegraphics[width=0.43\textwidth]{figures/HL/compNomRetrCollSett/perTCThit/ratioRadNAll.pdf}
\includegraphics[width=0.43\textwidth]{figures/HL/compNomRetrCollSett/perTCThit/ratioRadEnAll.pdf}
\end{center}
%% \begin{picture} (0.,0.)
%% \setlength{\unitlength}{1.0cm}
%% \small{
%%     \put ( 4.05,8.95){(a)}
%%     \put ( 12.45,8.95){(b)}
%%     \put ( 4.05,1.){(c)}
%%     \put ( 12.45,1.){(d)}
%% }
%% \end{picture}
\vspace{-0.6cm}
 \caption{Comparison background showers with nominal and \twosigmaret~collimator settings (see Tab.~\ref{HLcollSettings}).
  \label{compNomRetrSett}}
\end{figure}

\subsubsection{Comparison of B1 and B2 induced showers with TCT5s in}


\begin{figure}
\begin{center}
\includegraphics[width=0.43\textwidth]{figures/HL/compINB1B2/perTCThit/ratioEkinAll.pdf}
\includegraphics[width=0.43\textwidth]{figures/HL/compINB1B2/perTCThit/ratioPhiEnAll.pdf}
\includegraphics[width=0.43\textwidth]{figures/HL/compINB1B2/perTCThit/ratioRadNAll.pdf}
\includegraphics[width=0.43\textwidth]{figures/HL/compINB1B2/perTCThit/ratioRadEnAll.pdf}
\end{center}
%% \begin{picture} (0.,0.)
%% \setlength{\unitlength}{1.0cm}
%% \small{
%%     \put ( 4.05,8.95){(a)}
%%     \put ( 12.45,8.95){(b)}
%%     \put ( 4.05,1.){(c)}
%%     \put ( 12.45,1.){(d)}
%% }
%% \end{picture}
\vspace{-0.6cm}
 \caption{Comparison B1 and B2 induced background showers with \twosigmaret~collimator settings.
  \label{compINB1B2}}
\end{figure}

\newpage
\section{Evolution of Background in LHC and Comparison to HL--LHC\label{evolut}}

While several sources of background types had been studied in the past, we focus here on the evolution of the most dominant ones, local beam-gas and beam-halo, in IR1. IR5 usually is less prone to halo background as simulation results and operation show.




\subsection{Comparison of Run I background sources}
We investigate how the different sources of background compare to each other. We show them per interaction for the Run I case in Fig.~\ref{fig:compAllBKG_perInt1} and normalised to expected rates for stable beam operation in Fig~\ref{compAllBKG4TeV_rates}.

\begin{figure}
\begin{center}
  \includegraphics[width=0.42\textwidth]{figures/4TeV/compAllBKG/EkinAll.pdf}
  \includegraphics[width=0.42\textwidth]{figures/4TeV/compAllBKG/PhiEnAll.pdf}
  \includegraphics[width=0.42\textwidth]{figures/4TeV/compAllBKG/EkinMuons.pdf}
  \includegraphics[width=0.42\textwidth]{figures/4TeV/compAllBKG/PhiEnMuons.pdf}
  \includegraphics[width=0.42\textwidth]{figures/4TeV/compAllBKG/EkinProtons.pdf}
  \includegraphics[width=0.42\textwidth]{figures/4TeV/compAllBKG/PhiEnProtons.pdf}
  \includegraphics[width=0.42\textwidth]{figures/4TeV/compAllBKG/EkinPhotons.pdf}
  \includegraphics[width=0.42\textwidth]{figures/4TeV/compAllBKG/PhiEnPhotons.pdf}
\end{center}
\vspace{-0.6cm}
 \caption{Comparison of all background sources at 4 TeV normalised per interaction showing energy spectrum and energy in $\phi$.
  \label{fig:compAllBKG_perInt1}}
\end{figure}

%% \begin{figure}
%% \begin{center}
%%   \includegraphics[width=0.42\textwidth]{figures/4TeV/compAllBKG/RadNAll.pdf}
%%   \includegraphics[width=0.42\textwidth]{figures/4TeV/compAllBKG/RadEnAll.pdf}
%%   \includegraphics[width=0.42\textwidth]{figures/4TeV/compAllBKG/RadNMuons.pdf}
%%   \includegraphics[width=0.42\textwidth]{figures/4TeV/compAllBKG/RadEnMuons.pdf}
%%   \includegraphics[width=0.42\textwidth]{figures/4TeV/compAllBKG/RadNProtons.pdf}
%%   \includegraphics[width=0.42\textwidth]{figures/4TeV/compAllBKG/RadEnProtons.pdf}
%%   \includegraphics[width=0.42\textwidth]{figures/4TeV/compAllBKG/RadNPhotons.pdf}
%%   \includegraphics[width=0.42\textwidth]{figures/4TeV/compAllBKG/RadEnPhotons.pdf}
%% \end{center}
%% \vspace{-0.6cm}
%%  \caption{Comparison of all background sources at 4 TeV normalised per interaction showing radial distributions and energy in $r$.
%%   \label{fig:compAllBKG_perInt2}}
%% \end{figure}


\begin{figure}
\begin{center}
  \includegraphics[width=0.42\textwidth]{figures/4TeV/reweighted/cv78_EkinAll.pdf}
  \includegraphics[width=0.42\textwidth]{figures/4TeV/reweighted/cv78_PhiEnAll.pdf}
  \includegraphics[width=0.42\textwidth]{figures/4TeV/reweighted/cv78_EkinMuons.pdf}
  \includegraphics[width=0.42\textwidth]{figures/4TeV/reweighted/cv78_PhiEnMuons.pdf}
  \includegraphics[width=0.42\textwidth]{figures/4TeV/reweighted/cv78_EkinProtons.pdf}
  \includegraphics[width=0.42\textwidth]{figures/4TeV/reweighted/cv78_PhiEnProtons.pdf}
  \includegraphics[width=0.42\textwidth]{figures/4TeV/reweighted/cv78_EkinPhotons.pdf}
  \includegraphics[width=0.42\textwidth]{figures/4TeV/reweighted/cv78_PhiEnPhotons.pdf}
\end{center}
\vspace{-0.6cm}
 \caption{comparison of all background sources at 4 TeV normalised to a rate.
  \label{compAllBKG4TeV_rates}}
\end{figure}

%% \begin{figure}
%% \begin{center}
%%   \includegraphics[width=0.42\textwidth]{figures/4TeV/reweighted/cv78_RadNAll.pdf}
%%   \includegraphics[width=0.42\textwidth]{figures/4TeV/reweighted/cv78_RadEnAll.pdf}
%%   \includegraphics[width=0.42\textwidth]{figures/4TeV/reweighted/cv78_RadNMuons.pdf}
%%   \includegraphics[width=0.42\textwidth]{figures/4TeV/reweighted/cv78_RadEnMuons.pdf}
%%   \includegraphics[width=0.42\textwidth]{figures/4TeV/reweighted/cv78_RadNProtons.pdf}
%%   \includegraphics[width=0.42\textwidth]{figures/4TeV/reweighted/cv78_RadEnProtons.pdf}
%%   \includegraphics[width=0.42\textwidth]{figures/4TeV/reweighted/cv78_RadNPhotons.pdf}
%%   \includegraphics[width=0.42\textwidth]{figures/4TeV/reweighted/cv78_RadEnPhotons.pdf}
%% \end{center}
%% \vspace{-0.6cm}
%%  \caption{comparison of all background sources at 4 TeV normalised to a rate.
%%   \label{compAllBKG4TeV_rates}}
%% \end{figure}

% --------------------------------------------------------------------------------------------
\subsection{Comparison of Run II background sources}
\begin{figure}
\begin{center}
  \includegraphics[width=0.42\textwidth]{figures/6500GeV/reweighted/cv78_EkinAll.pdf}
  \includegraphics[width=0.42\textwidth]{figures/6500GeV/reweighted/cv78_PhiEnAll.pdf}
  \includegraphics[width=0.42\textwidth]{figures/6500GeV/reweighted/cv78_EkinMuons.pdf}
  \includegraphics[width=0.42\textwidth]{figures/6500GeV/reweighted/cv78_PhiEnMuons.pdf}
  \includegraphics[width=0.42\textwidth]{figures/6500GeV/reweighted/cv78_EkinProtons.pdf}
  \includegraphics[width=0.42\textwidth]{figures/6500GeV/reweighted/cv78_PhiEnProtons.pdf}
 \includegraphics[width=0.42\textwidth]{figures/6500GeV/reweighted/cv78_EkinPhotons.pdf}
 \includegraphics[width=0.42\textwidth]{figures/6500GeV/reweighted/cv78_PhiEnPhotons.pdf}
\end{center}
\vspace{-0.6cm}
 \caption{comparison of all background sources at 6.5~TeV.
  \label{compAllBKG_6.5}}
\end{figure}

\begin{figure}
\begin{center}
  \includegraphics[width=0.8\textwidth]{figures/cv87_allenergies_OrigZAll.pdf}
  \includegraphics[width=0.8\textwidth]{figures/cv87_allenergies_OrigZMuon.pdf}
\end{center}
\vspace{-0.6cm}
 \caption{
  \label{fig:OrigZMuonAllEn}} 
\end{figure}

% --------------------------------------------------------------------------------------------
\subsection{Beam-gas rate comparisons in Run I and Run II}
\begin{figure}[!htb]
\centering
\includegraphics[width=0.45\textwidth]{figures/compBGreweighted/ratioEkinAll.pdf}
\includegraphics[width=0.45\textwidth]{figures/compBGreweighted/ratioEkinMuons.pdf}
\includegraphics[width=0.45\textwidth]{figures/compBGreweighted/ratioPhiEnAll.pdf}
\includegraphics[width=0.45\textwidth]{figures/compBGreweighted/ratioPhiEnMuons.pdf}
\caption{Reweighted beam-gas distributions in the 2012 Run I and 2015 Run II scenario for all particles and muons showing the energy spectrum (top) and the azimuthal distribution (bottom).
  \label{fig:compBGreweighted1}}
\end{figure}

\begin{figure}%[!htb]
\centering
\includegraphics[width=0.45\textwidth]{figures/compBGreweighted/ratioRadNAll.pdf}
\includegraphics[width=0.45\textwidth]{figures/compBGreweighted/ratioRadNMuons.pdf}
\includegraphics[width=0.45\textwidth]{figures/compBGreweighted/ratioRadEnAll.pdf}
\includegraphics[width=0.45\textwidth]{figures/compBGreweighted/ratioRadEnMuons.pdf}
\caption{Reweighted beam-gas distributions in the 2012 Run I and 2015 Run II scenario for all particles and muons showing radial positions and energy in $r$.
  \label{fig:compBGreweighted2}}
\end{figure}

%% \subsection{Comparison of halo hit distributions}


%% \begin{figure}
%% \begin{center}
%% \includegraphics[width=0.495\textwidth]{figures/inelposition_sum_impacts_real_HL_TCT5IN_nomColl_haloB1.pdf}
%% %%\includegraphics[width=0.495\textwidth]{figures/inelposition_sum_impacts_real_tct5inb1_crabs.pdf}
%% \end{center}
%%  \caption{Depth distribution of TCT hits when using the nominal collimator settings (from the LHC Design report).
%%   \label{inelHLtct5inNomCrab}}
%% \end{figure}



\subsection{Comparison of different scenarios}



\begin{figure}
  \centering
    \includegraphics[width=0.495\textwidth]{figures/XYNMuons_BG_6500GeV_flat_20GeV_bs.pdf}
%    \includegraphics[width=0.495\textwidth]{figures/BH_run2/b2/XYNMuons_BH_6500GeV_haloB2_20MeV.pdf}
    \includegraphics[width=0.495\textwidth]{figures/XYNMuons_BH_6500GeV_haloB1_20MeV.pdf}
  \caption{Spatial distribution of muons of all energies in an beam-gas (left) and beam-halo (right) scenario for a 6.5~TeV beam in Run II 2015.
    \label{fig:XYNMuons}}
\end{figure}

\subsubsection{2012 4 TeV vs. 2015 6.5~TeV}
\begin{figure}
\begin{center}
  \includegraphics[width=0.49\textwidth]{figures/compBHB1_4TeV_vs_6p5TeV/normalised/ratioEkinAll.pdf}
%  \includegraphics[width=0.49\textwidth]{figures/compBHB1_4TeV_vs_6p5TeV/normalised/ratioEkinMuons.pdf}
  \includegraphics[width=0.49\textwidth]{figures/compBHB1_4TeV_vs_6p5TeV/normalised/ratioPhiEnAll.pdf}
  \includegraphics[width=0.49\textwidth]{figures/compBHB1_4TeV_vs_6p5TeV/normalised/ratioPhiEnMuons.pdf}
%  \includegraphics[width=0.49\textwidth]{figures/compBHB1_4TeV_vs_6p5TeV/normalised/ratioRadEnAll.pdf}
  \includegraphics[width=0.49\textwidth]{figures/compBHB1_4TeV_vs_6p5TeV/normalised/ratioRadEnMuons.pdf}
\end{center}
\vspace{-0.6cm}
 \caption{Comparison of halo induced background at 4 and 6.5~TeV in the azimuthal distributions of all particles at the interface plane (top) and high-energy muons and protons (bottom) and their energy.
  \label{compBHB1run1run2}}
\end{figure}

\begin{figure}%[!htb]
\begin{center}
  \includegraphics[width=0.49\textwidth]{figures/compBHB2_4TeV_vs_6p5TeV/normalised/ratioEkinAll.pdf}
  \includegraphics[width=0.49\textwidth]{figures/compBHB2_4TeV_vs_6p5TeV/normalised/ratioPhiEnAll.pdf}
  \includegraphics[width=0.49\textwidth]{figures/compBHB2_4TeV_vs_6p5TeV/normalised/ratioPhiEnMuons.pdf}
  \includegraphics[width=0.49\textwidth]{figures/compBHB2_4TeV_vs_6p5TeV/normalised/ratioRadEnMuons.pdf}
\end{center}
\vspace{-0.6cm}
 \caption{Comparison of halo induced background at 4 and 6.5~TeV in the azimuthal distributions of all particles at the interface plane (top) and high-energy muons and protons (bottom) and their energy.
  \label{compBHB2run1run2}}
\end{figure}



\begin{figure}%[!htb]
\begin{center}
  \includegraphics[width=0.49\textwidth]{figures/compBG_4TeV_vs_6.5TeV/ratioEkinAll.pdf}
  \includegraphics[width=0.49\textwidth]{figures/compBG_4TeV_vs_6.5TeV/ratioPhiEnAll.pdf}
  \includegraphics[width=0.49\textwidth]{figures/compBG_4TeV_vs_6.5TeV/ratioPhiNMuE100.pdf}
  \includegraphics[width=0.49\textwidth]{figures/compBG_4TeV_vs_6.5TeV/ratioPhiEnMuE100.pdf}
\end{center}
\vspace{-0.6cm}
 \caption{Comparison BG interactions in the azimuthal distributions of all particles at the interface plane (top) and high-energy muons (bottom) and their energy at 4 and 6.5 TeV.
  \label{compBGrun1run2}}
\end{figure}


\subsection{Run II vs HL--LHC}

\begin{figure}
\begin{center}
  \includegraphics[width=0.42\textwidth]{figures/HLRunII/cv78_EkinAll.pdf}
  \includegraphics[width=0.42\textwidth]{figures/HLRunII/cv78_PhiEnAll.pdf}
  \includegraphics[width=0.42\textwidth]{figures/HLRunII/cv78_EkinMuons.pdf}
  \includegraphics[width=0.42\textwidth]{figures/HLRunII/cv78_PhiEnMuons.pdf}
  \includegraphics[width=0.42\textwidth]{figures/HLRunII/cv78_EkinProtons.pdf}
  \includegraphics[width=0.42\textwidth]{figures/HLRunII/cv78_PhiEnProtons.pdf}
  \includegraphics[width=0.42\textwidth]{figures/HLRunII/cv78_EkinPhotons.pdf}
  \includegraphics[width=0.42\textwidth]{figures/HLRunII/cv78_PhiEnPhotons.pdf}
\end{center}
\vspace{-0.6cm}
 \caption{Comparison of beam-gas (BG) in Run II and beam-halo in HL using the baseline layout (TCT5s in, \twosigmaret~settings) and round beam optics.
  \label{fig:hlrun2}}
\end{figure}

\section{Conclusion and outlook}

We investigated in detail and systematically two sources of background, beam-gas and beam-halo, the most relevant sources for the experiments. The study was carried out for IR1 and both beams in general. Although there is a layout symmetry for incoming beams in one IR, both beams have a different ``history'' which will result in different backround patterns. 



%-------------------------------------------------------------------------------------------
\section*{Acknowledgments}
The authors would like to acknowledge the forum of the LHC Background Study Group (LBS), organised by Helmut Burkhardt and Reyes A.~Fernandez, where most of the work of this presented here was discussed.

%-------------------------------------------------------------------------------------------
\clearpage
\appendix
\section{Full tables of SixTrack TCP-to-TCT conversion factors}
\subsection*{IR7 to TCT conversion factors in betatron halo cleaning simulations\label{leakageFactorsIR7}}

SixTrack simulation results for 2012 Run~I, 2015~Run~II and HL-LHC are shown in Tab.~\ref{tab:leakageFactorsIR7}, and for 2016 Run~II in Tab.~\ref{2016leakageFactorsIR7}.
\begin{table}
   \centering
   \caption{Intercepted protons on IR7 TCPs (\textsc{TCP.D6, TCP.C6, TCP.B6}), IR1/5 and IR7 leakage to IR1/IR5 TCTs using Eq.~\ref{eq3}. The TCT settings are also indicated in the heading of each simulation case. All results are from SixTrack betatron cleaning simulations.}

   \begin{tabular}{c|cc|cc}

       & hB1 & vB1 & hB2 & vB2\\ \hline       
       & \multicolumn{4}{c}{4 TeV 2012, 9~$\sigma$} \\   %\cline{2-5}%\hline
       TCPs in IR7 & 50807535 & 53036514 & 49207325 & 46222723 \\
       TCTs in IR1 & 622 & 930 & 1179 & 967 \\
       ICTs in IR5 & 958 & 626 & 1893 & 135 \\ %
       IR7 to IR1  & \multicolumn{2}{c|}{1.5 $\times 10^{-5}$} & \multicolumn{2}{c}{2.4 $\times 10^{-5}$ } \\
       IR7 to IR5  & \multicolumn{2}{c|}{1.5 $\times 10^{-5}$} & \multicolumn{2}{c}{2.6 $\times 10^{-5}$ } \\
       \hline
       & \multicolumn{4}{c}{6.5 TeV 2015, 13.7~$\sigma$} \\      
       total losses & 62515929 & 62692523 & 50890652 & 63119778 \\
       %     peak loss in IR7 & 5.18402e+07 & 
       TCPs in IR7 & 53731448 & 52806720 & 43692659 & 52962459 \\
       TCTs in IR1 & 739 & 585 & 779 & 773 \\
       TCTs in IR5 & 346 & 408 & 302 & 106 \\% \cline{2-5}
       IR7 to IR1 &  \multicolumn{2}{c|}{1.2 $\times 10^{-5}$} &  \multicolumn{2}{c}{1.6 $\times 10^{-5}$ } \\
       IR7 to IR5 &  \multicolumn{2}{c|}{7.1 $\times 10^{-6}$} &  \multicolumn{2}{c}{4.5 $\times 10^{-6}$ } \\
       \hline       

       & \multicolumn{4}{c}{HL nominal settings, TCT5s in, round beam, 8.3~$\sigma$}  \\ %see lhc_mib/HL1.0/factors
       
       TCPs in IR7 & 52836357 & 50278617 & & \\
       TCTs in IR1 & 32557 &15813 & & \\
       TCTs in IR5 & 7500 & 3154   & & \\
       IR7 to IR1  &  \multicolumn{2}{c|}{ 4.7 $\times 10^{-4}$ }& &   \\ 
       IR7 to IR5  &  \multicolumn{2}{c|}{ 1.0 $\times 10^{-4}$} & &  \\ 
       \hline
       & \multicolumn{4}{c}{ HL \twosigmaret~settings, TCT5s in, round beam, 10.5~$\sigma$ }  \\
       TCPs in IR7 & 54532193 & 52154816 & 40401333 & 53199970 \\
       TCTs in IR1 & 9712 & 3366 & 9948 &  12028\\
       TCTs in IR5 & 1506 & 1122 & 473  & 95 \\
       IR7 to IR1  &  \multicolumn{2}{c|}{ 1.2 $\times 10^{-4}$ } &  \multicolumn{2}{c} { 2.36 $\times 10^{-4}$} \\
       IR7 to IR5 & \multicolumn{2}{c|}{ 2.5 $\times 10^{-5}$} & \multicolumn{2}{c} {6.8 $\times 10^{-6}$ } \\
       \hline
       & \multicolumn{4}{c}{ HL \twosigmaret~settings, TCT4s only, round beam, 10.5~$\sigma$ }  \\
       TCPs in IR7 & 54609869 & 52175081 & 40392116 & 53157089 \\
       TCTs in IR1 & 9024 & 3071 & 9936 & 11898 \\
       TCTs in IR5 & 1408 & 1024 & 368 & 70 \\
       IR7 to IR1  &  \multicolumn{2}{c|}{ 1.1 $\times 10^{-4}$ } &  \multicolumn{2}{c} { 2.35 $\times 10^{-4}$} \\
       IR7 to IR5  &  \multicolumn{2}{c|}{ 2.7 $\times 10^{-5}$ } &  \multicolumn{2}{c} { 5.2 $\times 10^{-6}$} \\
       \hline
       & \multicolumn{4}{c}{ HL \twosigmaret~settings, TCT5s in, flat beam, 10.5~$\sigma$ }  \\
       TCPs in IR7 & 35831038 & 5052849 &  &  \\
       TCTs in IR1 & 7814 & 306 & & \\
       TCTs in IR5 & 660 & 75 & \\
       IR7 to IR1  & \multicolumn{2}{c|}{1.4 $\times 10^{-4}$ } &  \\
       IR7 to IR5 & \multicolumn{2}{c|}{1.7 $\times 10^{-5}$} & \\
       \hline
       & \multicolumn{4}{c}{ HL \twosigmaret~settings, TCT4s only, flat beam, 10.5~$\sigma$ }  \\
       TCPs in IR7 & 14320971 & 50517818 &  &  \\
       TCTs in IR1 & 3035 & 3035 & & \\
       TCTs in IR5 & 338 & 824 & \\
       IR7 to IR1  & \multicolumn{2}{c|}{1.4 $\times 10^{-4}$ } &  \\
       IR7 to IR5 & \multicolumn{2}{c|}{2.0 $\times 10^{-5}$} & \\

       %% HL retracted settings, flat beam  & B1 \\       
       %% IR1 & 1.39 $\times 10^{-4}$ & \\ % B1 : .5*(7814.0/35831038.0 + 306.0/5052849.0)
       %% IR5 & 1.66 $\times 10^{-5}$ & \\ % B1 : .5*(660.0/35831038.0 + 75.0/5052849.0)
       \hline
   \end{tabular}
   \label{tab:leakageFactorsIR7}
\end{table}


\begin{table}
   \centering
   \caption{Intercepted protons on IR7 TCPs (\textsc{TCP.D6, TCP.C6, TCP.B6}), IR1/5 and IR7 leakage to IR1/IR5 TCTs using Eq.~\ref{eq3}. In the last line the leakage was calculated wrt total losses in the simulations. All results are from SixTrack betatron cleaning simulations.}
   \begin{tabular}{c|cc|cc}
     \hline

       & \multicolumn{4}{c}{6.5 TeV 2016, 9~$\sigma$} \\      
     & hB1 & vB1 & hB2 & vB2\\ \hline       

       total losses & 19062803 & 18407758 & 19082325 & 18404556 \\       
       TCPs in IR7 & 16245854 & 15577729 & 16268025 & 15583550 \\
       TCTs in IR1 & 8346 & 5083 & 2138 & 1641 \\
       TCTs in IR5 & 2563 & 3709 & 2500 & 970 \\% \cline{2-5}
       IR7 to IR1 &  \multicolumn{2}{c|}{4.2 $\times 10^{-4}$} &  \multicolumn{2}{c}{1.2 $\times 10^{-4}$ } \\
       IR7 to IR5 &  \multicolumn{2}{c|}{2.0 $\times 10^{-4}$} &  \multicolumn{2}{c}{1.1 $\times 10^{-4}$ } \\
       IR1 leakage/all losses &  \multicolumn{2}{c|}{3.6 $\times 10^{-4}$} &  \multicolumn{2}{c}{1.0 $\times 10^{-4}$ } \\

       \hline
   \end{tabular}
   \label{2016leakageFactorsIR7}
\end{table}


\subsection*{IR3 to TCT conversion factors in off-momentum cleaning simulations}
SixTrack simulation results for 2012 Run~I and 2015~Run~II are shown in Tab.~\ref{tab:IR3leakageFactors}.
\begin{table}[!h]
   \centering
   \caption{Leakage from IR3 primary collimator to IR1/IR5 TCTs from off-momentum cleaning simulations. The numbers in brackets show the losses per TCT, TCTH + TCTV. }

   \begin{tabular}{l|c|c}
       \hline
       4 TeV, + 500~Hz  & B1 & B2\\
       TCP.IR3  & 6279727 & 995698  \\
       TCTs IR1 & 6 (6 + 0) & 11919 (695 + 11224) \\
       TCTs IR5 & 28304 (5387 + 22917) & 16 (14 + 2) \\
       IR3 to IR1 & 9.5 10$^{-7}$ & 0.012 \\
       IR3 to IR5 & 0.0045 & 1.6 10$^{-5}$ \\
       \hline
       4 TeV, -~500~Hz  & B1 & B2\\
       TCP.IR3  & 1853387 & 3501844 \\
       TCTs IR1  & 93 (37 + 56) & 22278 (4851+17427) \\
       TCTs IR5  &  4735 (1746 + 2989) & 23 (22 + 1) \\
       IR3 to IR1 & 5.0 10$^{-5}$ & 0.0063 \\
       IR3 to IR5 & 0.0025 & 6.2 10$^{-6}$ \\
       \hline

       6.5 TeV, + 500~Hz  & B1 & B2\\
       TCP.IR3  & 8055019 & 5648325  \\
       TCTs IR1 & 3 (3 + 0) & 11951 (5607 + 6344) \\
       TCTs IR5 & 17963 (8711 + 9252) & 9 (9 + 0)\\
       IR3 to IR1 & 3.7 10$^{-7}$ & 0.0021 \\
       IR3 to IR5 & 0.0022 & 1.5 10$^{-6}$ \\
       \hline
       6.5 TeV, - 500~Hz  & B1 & B2\\
       TCP.IR3  &  3389385 &  4965539 \\
       TCTs IR1  &  1 (1 + 0) &  12263 (6238 + 6025) \\
       TCTs IR5  &  7642 (4337 + 3305) &  5 (4 + 1)\\
       IR3 to IR1 &  2.9 10$^{-7}$ & 0.0025 \\
       IR3 to IR5 &  0.0022 &  1.0 $10^{-6}$ \\
       \hline

   \end{tabular}
   \label{tab:IR3leakageFactors}
\end{table}

 % leakageTables
\newpage
\section{Additional plots for Run I and II simulations\label{run1run2app}}
\clearpage


\begin{figure}[!htb]
\begin{center}

\includegraphics[width=0.49\textwidth]{figures/4TeV/haloB1_20MeV/RadNDist_BH_4TeV_B1_20MeV.pdf}
\includegraphics[width=0.49\textwidth]{figures/4TeV/haloB1_20MeV/RadEnDist_BH_4TeV_B1_20MeV.pdf}

\end{center}
\vspace{-0.6cm}
 \caption{Halo Beam 1 distributions at the interface plane. 
  \label{dist4TeVB12}}
\end{figure}

\begin{figure}%[!htb]
\begin{center}
%\includegraphics[width=0.4\textwidth]{figures/4TeV/compB1B2/perTCThit/ratioEkinMuons.pdf}
%\includegraphics[width=0.4\textwidth]{figures/4TeV/compB1B2/perTCThit/ratioPhiEnAll.pdf}
\includegraphics[width=0.4\textwidth]{figures/4TeV/compB1B2/perTCThit/ratioPhiEnMuons.pdf}
\includegraphics[width=0.4\textwidth]{figures/4TeV/compB1B2/perTCThit/ratioPhiEnProtons.pdf}
\end{center}
\vspace{-0.6cm}
\caption{Comparison of B1 and B2 halo shower distributions at the interface plane. The numbers in the ratio plot is the ratio of both integrals of the top distributions to indicate by how much nominator or denominator is larger. The error bars indicate statistical uncertainties.
  \label{comp4TeVB1B2}}
\end{figure}


\begin{figure}%[!htb]
\begin{center}
\includegraphics[width=0.49\textwidth]{figures/4TeV/bs_20MeV/RadNDist_BG_4TeV_20MeV_bs.pdf}
\includegraphics[width=0.49\textwidth]{figures/4TeV/bs_20MeV/RadEnDist_BG_4TeV_20MeV_bs.pdf}
\end{center}
\vspace{-0.6cm}
 \caption{Beam-gas induced background shown per BG interaction (flat pressure).
  \label{dist4TeVBGbs2}}
\end{figure}

% ------------------------------------
% comp Run1 

\begin{figure}
\begin{center}
  \includegraphics[width=0.41\textwidth]{figures/4TeV/compBG_3p5_vs_4TeV/perBGint_bs/ratioPhiEnAll.pdf}
  \includegraphics[width=0.41\textwidth]{figures/4TeV/compBG_3p5_vs_4TeV/perBGint_bs/ratioPhiEnPhotons.pdf}
  \includegraphics[width=0.41\textwidth]{figures/4TeV/compBG_3p5_vs_4TeV/perBGint_bs/ratioPhiEnMuons.pdf}
  \includegraphics[width=0.41\textwidth]{figures/4TeV/compBG_3p5_vs_4TeV/perBGint_bs/ratioPhiEnMuE100.pdf}
\end{center}
\vspace{-0.6cm}
 \caption{Run I beam-gas data: Same conclusions as from Fig.~\ref{xingCompBG}. Clear effect of crossing angle as also when comparing to the beam-gas data with beamsize (bs) (top) and no influence on muons (middle). Since the TCTs were at 11.8~$\sigma$ at 3.5~TeV and 9~$\sigma$ comparison of BG from before the TCTs may tell also about the influence of TCT settings on beam-gas. The bottom plots show there is no significant influence.
  \label{xingCompBG2}}
\end{figure}

% ------------------------------------
% crossing angle comparison
\begin{figure}
  \begin{center}
    \includegraphics[width=0.41\textwidth]{figures/4TeV/compB2_3p5vs4TeV/ratioPhiEnAll.pdf}
    \includegraphics[width=0.41\textwidth]{figures/4TeV/compB2_3p5vs4TeV/ratioPhiEnPhotons.pdf}
    %% \includegraphics[width=0.41\textwidth]{figures/4TeV/compB2_3p5vs4TeV/ratioPhiEnAll.pdf}
    %% \includegraphics[width=0.41\textwidth]{figures/4TeV/compB2_3p5vs4TeV/ratioPhiEnPhotons.pdf}
    %% \includegraphics[width=0.41\textwidth]{figures/4TeV/compB2_3p5vs4TeV/ratioPhiEnMuMinus.pdf}
    %% \includegraphics[width=0.41\textwidth]{figures/4TeV/compB2_3p5vs4TeV/ratioPhiEnMuPlus.pdf}
  \end{center}
  \vspace{-0.6cm}
  \caption{B2 beam halo data without a clear sign of a crossing angle effect visible.
    \label{xingCompBHB2}}
\end{figure}

\begin{figure}%[!htb]
\begin{center}
  \includegraphics[width=0.411\textwidth]{figures/4TeV/beamsizeRatio/ratioPhiNProtonsE100.pdf}
  \includegraphics[width=0.411\textwidth]{figures/4TeV/beamsizeRatio/ratioPhiEnProtons.pdf}
  \includegraphics[width=0.411\textwidth]{figures/4TeV/beamsizeRatio/ratioPhiNMuonsE100.pdf}
  \includegraphics[width=0.411\textwidth]{figures/4TeV/beamsizeRatio/ratioPhiEnMuons.pdf}
\end{center}
\vspace{-0.6cm}
 \caption{Effect of beam size in 4 TeV beam-gas: azimuthal distributions of multiplicity (left) of high energy protons (top) and muons (bottom) and all proton and muon energies. 
  \label{bsRatioPhiMP}}
\end{figure}

\begin{figure}%[!htb]
\begin{center}
  \includegraphics[width=0.24\textwidth]{figures/4TeV/beamsizeRatio/ratioPhiEnPrZ1.pdf}
  \includegraphics[width=0.24\textwidth]{figures/4TeV/beamsizeRatio/ratioPhiEnPrZ2.pdf}
  \includegraphics[width=0.24\textwidth]{figures/4TeV/beamsizeRatio/ratioPhiEnPrZ3.pdf}
  \includegraphics[width=0.24\textwidth]{figures/4TeV/beamsizeRatio/ratioPhiEnPrZ4.pdf}
\end{center}
\vspace{-0.6cm}
 \caption{Azimuthal energy distributions of protons in different s-regions.
  \label{bsZPr}}
\end{figure}

\begin{figure}%[!htb]
\begin{center}
  \includegraphics[width=0.411\textwidth]{figures/4TeV/beamsizeRatio/ratioEkinAll.pdf}
  \includegraphics[width=0.411\textwidth]{figures/4TeV/beamsizeRatio/ratioEkinMuons.pdf}
  \includegraphics[width=0.411\textwidth]{figures/4TeV/beamsizeRatio/ratioEkinNeutrons.pdf}
  \includegraphics[width=0.411\textwidth]{figures/4TeV/beamsizeRatio/ratioEkinProtons.pdf}
\end{center}
\vspace{-0.6cm}
 \caption{Effect of beam size in 4 TeV beam-gas: kinetic energy of particles.
  \label{bsRatioEkin}}
\end{figure}

\begin{figure}%[!htb]
\begin{center}
  \includegraphics[width=0.411\textwidth]{figures/4TeV/beamsizeRatio/ratioRadNAll.pdf}
  \includegraphics[width=0.411\textwidth]{figures/4TeV/beamsizeRatio/ratioRadNMuons.pdf}
  \includegraphics[width=0.411\textwidth]{figures/4TeV/beamsizeRatio/ratioRadNNeutrons.pdf}
  \includegraphics[width=0.411\textwidth]{figures/4TeV/beamsizeRatio/ratioRadNProtons.pdf}
\end{center}
\vspace{-0.6cm}
 \caption{Effect of beam size in 4 TeV beam-gas: radius for different particle types.
  \label{bsRatioRadN}}
\end{figure}

\begin{figure}%[!htb]
\begin{center}
  \includegraphics[width=0.411\textwidth]{figures/4TeV/beamsizeRatio/ratioRadEnAll.pdf}
  \includegraphics[width=0.411\textwidth]{figures/4TeV/beamsizeRatio/ratioRadEnMuons.pdf}
  \includegraphics[width=0.411\textwidth]{figures/4TeV/beamsizeRatio/ratioRadEnNeutrons.pdf}
  \includegraphics[width=0.411\textwidth]{figures/4TeV/beamsizeRatio/ratioRadEnProtons.pdf}
\end{center}
\vspace{-0.6cm}
 \caption{Effect of beam size in 4 TeV beam-gas: energy in $r$ of different particle types.
  \label{bsRatioRadEn}}
\end{figure}


\newpage

% -----------------------------------------------------------------------------------------------------
% beamgas before and after reweightening


%% \begin{figure}
%%   \centering
%%   \includegraphics[width=0.45\textwidth]{figures/4TeV/compBGflat/ratioEkinAll.pdf}
%%   \includegraphics[width=0.45\textwidth]{figures/4TeV/compBGflat/ratioEkinMuons.pdf}
%%   \includegraphics[width=0.45\textwidth]{figures/4TeV/compBGflat/ratioPhiEnAll.pdf}
%%   \includegraphics[width=0.45\textwidth]{figures/4TeV/compBGflat/ratioPhiEnMuons.pdf}
%%   \caption{Same as Fig.~\ref{fig:cv81EkinPhiEn4TeV} but with ratio at the bottom. Note the color change.
%%     \label{fig:cv16EkinPhiEn4TeV}}
%% \end{figure}
   


\begin{figure}
  
\begin{center}
  \includegraphics[width=0.75\textwidth]{figures/4TeV/reweighted/cv81_OrigZMuon_BG_4TeV_20MeV_bs}
  \includegraphics[width=0.75\textwidth]{figures/4TeV/reweighted/cv81_OrigZPhotons_BG_4TeV_20MeV_bs}
  \includegraphics[width=0.75\textwidth]{figures/4TeV/reweighted/cv81_OrigZProtons_BG_4TeV_20MeV_bs}
\end{center}
\vspace{-0.6cm}
 \caption{Origin of muon (top), photon (middle) and proton (bottom) production along s, in black before and in pink after re-normalising to the 2012 pressure profile. 
  \label{fig:OrigZ4TeV2}}
\end{figure}

\begin{figure}
\begin{center}
   \includegraphics[width=0.45\textwidth]{figures/4TeV/compBGflat/ratioEkinAll}
  %% \includegraphics[width=0.45\textwidth]{figures/4TeV/compBGflat/ratioEkinMuons}
   \includegraphics[width=0.45\textwidth]{figures/4TeV/compBGflat/ratioPhiEnAll}
  %% \includegraphics[width=0.45\textwidth]{figures/4TeV/compBGflat/ratioPhiEnMuons}
  \includegraphics[width=0.45\textwidth]{figures/4TeV/compBGflat/ratioRadNAll}
%%  \includegraphics[width=0.45\textwidth]{figures/4TeV/compBGflat/ratioRadNMuons}
  \includegraphics[width=0.45\textwidth]{figures/4TeV/compBGflat/ratioRadEnAll}
%%  \includegraphics[width=0.45\textwidth]{figures/4TeV/compBGflat/ratioRadEnMuons}
\end{center}
\vspace{-0.6cm}
 \caption{Same type of distributions as in Fig.~\ref{fig:cv81EkinPhiEn4TeV}, showing then here for all particles. They follow the same trend of muons. the transverse radius $r$ (top) and the energy in $r$ (bottom). 
  \label{fig:cv81EkinPhiEn4TeV2}} 
\end{figure}


\begin{figure}
\begin{center}
  \includegraphics[width=0.45\textwidth]{figures/4TeV/compBGflat/ratioEkinProtons}
  \includegraphics[width=0.45\textwidth]{figures/4TeV/compBGflat/ratioEkinNeutrons}
  \includegraphics[width=0.45\textwidth]{figures/4TeV/compBGflat/ratioPhiEnProtons}
  \includegraphics[width=0.45\textwidth]{figures/4TeV/compBGflat/ratioPhiEnNeutrons}
  \includegraphics[width=0.45\textwidth]{figures/4TeV/compBGflat/ratioRadNProtons}
  \includegraphics[width=0.45\textwidth]{figures/4TeV/compBGflat/ratioRadNNeutrons}
  \includegraphics[width=0.45\textwidth]{figures/4TeV/compBGflat/ratioRadEnProtons}
  \includegraphics[width=0.45\textwidth]{figures/4TeV/compBGflat/ratioRadEnNeutrons}
\end{center}
\vspace{-0.6cm}
 \caption{Same observables as in Fig.~\ref{fig:cv81EkinPhiEn4TeV} but for protons (left) and neutrons (right): energy spectrum, energy in $\phi$, transverse radius $r$, and energy in $r$ before reweightening (black) and reweighted to pressure profile. The black curve are scaled up to better compare the shapes. Very similar shapes before and after reweightening.
   \label{fig:cv81ProtNeut4TeV}}
\end{figure}

% -----------------------------------------------------------------------------------------------------
% offmomentum


\begin{figure}
  \begin{center}
    \includegraphics[width=0.49\textwidth]{figures/4TeV/offmom/20MeV/Ekin_offplus500Hz_4TeV_B2_20MeV.pdf}
    \includegraphics[width=0.49\textwidth]{figures/4TeV/offmom/20MeV/PhiEnDist_offplus500Hz_4TeV_B2_20MeV.pdf}
  \includegraphics[width=0.49\textwidth]{figures/4TeV/offmom/20MeV/RadNDist_offplus500Hz_4TeV_B2_20MeV.pdf}
  \includegraphics[width=0.49\textwidth]{figures/4TeV/offmom/20MeV/RadEnDist_offplus500Hz_4TeV_B2_20MeV.pdf}

\end{center}
\vspace{-0.6cm}
 \caption{Off-momentum induced particle distributions.
  \label{offmom4TeV2}}
\end{figure}




\begin{figure}%[!htb]
\begin{center}
  \includegraphics[width=0.30\textwidth]{figures/4TeV/offmom/comppm500Hz/ratioEkinMuons.pdf}
  \includegraphics[width=0.30\textwidth]{figures/4TeV/offmom/comppm500Hz/ratioEkinProtons.pdf}
  \includegraphics[width=0.30\textwidth]{figures/4TeV/offmom/comppm500Hz/ratioEkinNeutrons.pdf}
  \includegraphics[width=0.30\textwidth]{figures/4TeV/offmom/comppm500Hz/ratioEkinPhotons.pdf}
  \includegraphics[width=0.30\textwidth]{figures/4TeV/offmom/comppm500Hz/ratioEkinElecPosi.pdf}
\end{center}
\vspace{-0.6cm}
 \caption{Comparison of IR3 cleaning induced showers generated with a positive and negative frequency shift.
  \label{compPM_ekin}}
\end{figure}

\begin{figure}%[!htb]
\begin{center}
  \includegraphics[width=0.30\textwidth]{figures/4TeV/offmom/comppm500Hz/ratioPhiNAll.pdf}
  \includegraphics[width=0.30\textwidth]{figures/4TeV/offmom/comppm500Hz/ratioPhiNProtons.pdf}
  \includegraphics[width=0.30\textwidth]{figures/4TeV/offmom/comppm500Hz/ratioPhiNMuons.pdf}

  \includegraphics[width=0.30\textwidth]{figures/4TeV/offmom/comppm500Hz/ratioPhiEnProtons.pdf}
  \includegraphics[width=0.30\textwidth]{figures/4TeV/offmom/comppm500Hz/ratioPhiEnMuons.pdf}
  \includegraphics[width=0.30\textwidth]{figures/4TeV/offmom/comppm500Hz/ratioRadNAll.pdf}
  \includegraphics[width=0.30\textwidth]{figures/4TeV/offmom/comppm500Hz/ratioRadNProtons.pdf}
  \includegraphics[width=0.30\textwidth]{figures/4TeV/offmom/comppm500Hz/ratioRadNMuons.pdf}
  \includegraphics[width=0.30\textwidth]{figures/4TeV/offmom/comppm500Hz/ratioRadEnAll.pdf}
  \includegraphics[width=0.30\textwidth]{figures/4TeV/offmom/comppm500Hz/ratioRadEnProtons.pdf}
  \includegraphics[width=0.30\textwidth]{figures/4TeV/offmom/comppm500Hz/ratioRadEnMuons.pdf}
\end{center}
\vspace{-0.6cm}
\caption{Comparisons of particle distributions in the ``plus'' and ``minus'' cases showing all particles (left colum), protons (middle) and muons (right). 
The first two rows highlight multiplicity and energy in $\phi$, and the last two rows shows also multiplicity and energy in $r$. Since the radii are not inclusive distributions (everything with $r~>~600~$cm is cut off) unlike the $\phi$-distributions the integral ratios - all being below 1 - indicate that the ``plus-case'' gives higher contributions to smaller radii or differently said, the ``minus-case'' produces more energetic particles only for $r > 600~$cm. 
  \label{compPM_phien}}
\end{figure}
\newpage


\begin{figure}%[!htb]
\centering
\includegraphics[width=0.49\textwidth]{figures/BH_run2/b2/RadNDist_BH_6500GeV_haloB2_20MeV.pdf}
\includegraphics[width=0.49\textwidth]{figures/BH_run2/b2/RadEnDist_BH_6500GeV_haloB2_20MeV.pdf}
\includegraphics[width=0.49\textwidth]{figures/BH_run2/b2/OrigYZMuons_BH_6500GeV_haloB2_20MeV.pdf}
\includegraphics[width=0.49\textwidth]{figures/BH_run2/b2/OrigXYMuons_BH_6500GeV_haloB2_20MeV.pdf}
 \caption{B2 halo induced background at the interface plane. 
  \label{dist6500GeVB22}}
\end{figure}

\begin{figure}%[!htb]
\begin{center}
  \includegraphics[width=0.411\textwidth]{figures/BH_run2/perTCThit/ratioEkinMuons.pdf}
  \includegraphics[width=0.411\textwidth]{figures/BH_run2/perTCThit/ratioPhiNMuons.pdf}
  \includegraphics[width=0.411\textwidth]{figures/BH_run2/perTCThit/ratioPhiEnMuons.pdf}
  \includegraphics[width=0.411\textwidth]{figures/BH_run2/perTCThit/ratioRadEnMuons.pdf}
\end{center}
\vspace{-0.6cm}
 \caption{Comparison of B1/B2 halo induced distributions per TCT hit for muons.
  \label{compBHB1B2run2}}
\end{figure}

\clearpage

\begin{figure}%[!htb]
\begin{center}
%  \includegraphics[width=0.49\textwidth]{figures/6500GeV/20MeV/Ekin_BG_6500GeV_flat_20MeV_bs.pdf}
%  \includegraphics[width=0.49\textwidth]{figures/6500GeV/20MeV/PhiEnDist_BG_6500GeV_flat_20MeV_bs.pdf}
  \includegraphics[width=0.49\textwidth]{figures/6500GeV/20MeV/RadNDist_BG_6500GeV_flat_20MeV_bs.pdf}
  \includegraphics[width=0.49\textwidth]{figures/6500GeV/20MeV/RadEnDist_BG_6500GeV_flat_20MeV_bs.pdf}
\end{center}
\vspace{-0.6cm}
 \caption{Characteristic beam-gas induced distributions at 6.5~TeV per BG interaction using the more realistic model of the beam size.
  \label{bg65002}}
\end{figure}

\begin{figure}
\begin{center}
  \includegraphics[width=0.75\textwidth]{figures/6500GeV/reweighted/cv81_OrigZAll_BG_6500GeV_flat_20MeV_bs.pdf}
  \includegraphics[width=0.75\textwidth]{figures/6500GeV/reweighted/cv81_OrigZProtons_BG_6500GeV_flat_20MeV_bs.pdf}
  \includegraphics[width=0.75\textwidth]{figures/6500GeV/reweighted/cv81_OrigZPhotons_BG_6500GeV_flat_20MeV_bs.pdf}
\end{center}
\vspace{-0.6cm}
 \caption{Origin of all particles (top), protons (middle) and photons (bottom) production along s, in black before and in gold after re-normalising to the 2015 pressure profile. 
  \label{fig:OrigZ6p52}}
\end{figure}


\begin{figure}
\begin{center}
  \includegraphics[width=0.45\textwidth]{figures/6500GeV/compBGflat/ratioEkinAll}
  \includegraphics[width=0.45\textwidth]{figures/6500GeV/compBGflat/ratioPhiEnAll}
  \includegraphics[width=0.45\textwidth]{figures/6500GeV/compBGflat/ratioRadNAll}
  \includegraphics[width=0.45\textwidth]{figures/6500GeV/compBGflat/ratioRadEnAll}
\end{center}
\vspace{-0.6cm}
 \caption{The distributions 
  \label{fig:EkinPhiEn6p52}}
\end{figure}



\begin{figure}
\begin{center}
  \includegraphics[width=0.45\textwidth]{figures/6500GeV/compBGflat/ratioEkinProtons}
  \includegraphics[width=0.45\textwidth]{figures/6500GeV/compBGflat/ratioEkinNeutrons}
  \includegraphics[width=0.45\textwidth]{figures/6500GeV/compBGflat/ratioPhiEnProtons}
  \includegraphics[width=0.45\textwidth]{figures/6500GeV/compBGflat/ratioPhiEnNeutrons}
  \includegraphics[width=0.45\textwidth]{figures/6500GeV/compBGflat/ratioRadNProtons}
  \includegraphics[width=0.45\textwidth]{figures/6500GeV/compBGflat/ratioRadNNeutrons}
  \includegraphics[width=0.45\textwidth]{figures/6500GeV/compBGflat/ratioRadEnProtons}
  \includegraphics[width=0.45\textwidth]{figures/6500GeV/compBGflat/ratioRadEnNeutrons}
\end{center}
\vspace{-0.6cm}
 \caption{The distributions 
  \label{fig:ProtNeut6p52}} 
\end{figure}

% ----------------------------------
\clearpage

 % run1run2app
\clearpage
\clearpage
\begin{figure}
\begin{center}
  \includegraphics[width=0.42\textwidth]{figures/4TeV/compAllBKG/EkinAll.pdf}
  \includegraphics[width=0.42\textwidth]{figures/4TeV/compAllBKG/PhiEnAll.pdf}
%  \includegraphics[width=0.42\textwidth]{figures/4TeV/compAllBKG/EkinMuons.pdf}
%  \includegraphics[width=0.42\textwidth]{figures/4TeV/compAllBKG/PhiEnMuons.pdf}
  \includegraphics[width=0.42\textwidth]{figures/4TeV/compAllBKG/EkinProtons.pdf}
  \includegraphics[width=0.42\textwidth]{figures/4TeV/compAllBKG/PhiEnProtons.pdf}
  \includegraphics[width=0.42\textwidth]{figures/4TeV/compAllBKG/EkinPhotons.pdf}
  \includegraphics[width=0.42\textwidth]{figures/4TeV/compAllBKG/PhiEnPhotons.pdf}
\end{center}
\vspace{-0.6cm}
 \caption{Comparison of all background sources at 4 TeV normalised per interaction showing energy spectrum and energy in $\phi$.
  \label{fig:compAllBKG_perInt1}}
\end{figure}

\begin{figure}
\begin{center}
  \includegraphics[width=0.42\textwidth]{figures/4TeV/compAllBKG/RadNAll.pdf}
  \includegraphics[width=0.42\textwidth]{figures/4TeV/compAllBKG/RadEnAll.pdf}
  \includegraphics[width=0.42\textwidth]{figures/4TeV/compAllBKG/RadNMuons.pdf}
  \includegraphics[width=0.42\textwidth]{figures/4TeV/compAllBKG/RadEnMuons.pdf}
  \includegraphics[width=0.42\textwidth]{figures/4TeV/compAllBKG/RadNProtons.pdf}
  \includegraphics[width=0.42\textwidth]{figures/4TeV/compAllBKG/RadEnProtons.pdf}
  \includegraphics[width=0.42\textwidth]{figures/4TeV/compAllBKG/RadNPhotons.pdf}
  \includegraphics[width=0.42\textwidth]{figures/4TeV/compAllBKG/RadEnPhotons.pdf}
\end{center}
\vspace{-0.6cm}
 \caption{Comparison of all background sources at 4 TeV normalised per interaction showing radial distributions and energy in $r$.
  \label{fig:compAllBKG_perInt2}}
\end{figure}

\begin{figure}
\begin{center}
  \includegraphics[width=0.42\textwidth]{figures/4TeV/reweighted/app/cv78_EkinAll.pdf}
  \includegraphics[width=0.42\textwidth]{figures/4TeV/reweighted/app/cv78_PhiEnAll.pdf}
%  \includegraphics[width=0.42\textwidth]{figures/4TeV/reweighted/app/cv78_EkinMuons.pdf}
%  \includegraphics[width=0.42\textwidth]{figures/4TeV/reweighted/app/cv78_PhiEnMuons.pdf}
  \includegraphics[width=0.42\textwidth]{figures/4TeV/reweighted/app/cv78_EkinProtons.pdf}
  \includegraphics[width=0.42\textwidth]{figures/4TeV/reweighted/app/cv78_PhiEnProtons.pdf}
  \includegraphics[width=0.42\textwidth]{figures/4TeV/reweighted/app/cv78_EkinPhotons.pdf}
  \includegraphics[width=0.42\textwidth]{figures/4TeV/reweighted/app/cv78_PhiEnPhotons.pdf}
\end{center}
\vspace{-0.6cm}
 \caption{Comparison of background sources at 4 TeV at the interface plane normalised to a rate (halo is betatron halo).
  \label{compAllBKG4TeV_rates}}
\end{figure}

\begin{figure}
\begin{center}
  \includegraphics[width=0.42\textwidth]{figures/4TeV/reweighted/app/cv78_RadNAll.pdf}
  \includegraphics[width=0.42\textwidth]{figures/4TeV/reweighted/app/cv78_RadEnAll.pdf}
  \includegraphics[width=0.42\textwidth]{figures/4TeV/reweighted/app/cv78_RadNMuons.pdf}
  \includegraphics[width=0.42\textwidth]{figures/4TeV/reweighted/app/cv78_RadEnMuons.pdf}
  \includegraphics[width=0.42\textwidth]{figures/4TeV/reweighted/app/cv78_RadNProtons.pdf}
  \includegraphics[width=0.42\textwidth]{figures/4TeV/reweighted/app/cv78_RadEnProtons.pdf}
  \includegraphics[width=0.42\textwidth]{figures/4TeV/reweighted/app/cv78_RadNPhotons.pdf}
  \includegraphics[width=0.42\textwidth]{figures/4TeV/reweighted/app/cv78_RadEnPhotons.pdf}
\end{center}
\vspace{-0.6cm}
 \caption{Comparison of all background sources at 4 TeV at the inteface plane normalised to a rate (halo is betatron halo).
  \label{compAllBKG4TeV_rates2}}
\end{figure}




\begin{figure}
\begin{center}
  \includegraphics[width=0.42\textwidth]{figures/6500GeV/reweighted/app/cv78_EkinAll.pdf}
  \includegraphics[width=0.42\textwidth]{figures/6500GeV/reweighted/app/cv78_PhiEnAll.pdf}
%  \includegraphics[width=0.42\textwidth]{figures/6500GeV/reweighted/app/cv78_EkinMuons.pdf}
 % \includegraphics[width=0.42\textwidth]{figures/6500GeV/reweighted/app/cv78_PhiEnMuons.pdf}
  \includegraphics[width=0.42\textwidth]{figures/6500GeV/reweighted/app/cv78_EkinProtons.pdf}
  \includegraphics[width=0.42\textwidth]{figures/6500GeV/reweighted/app/cv78_PhiEnProtons.pdf}
 \includegraphics[width=0.42\textwidth]{figures/6500GeV/reweighted/app/cv78_EkinPhotons.pdf}
 \includegraphics[width=0.42\textwidth]{figures/6500GeV/reweighted/app/cv78_PhiEnPhotons.pdf}
\end{center}
\vspace{-0.6cm}
 \caption{Comparison of background sources at 6.5~TeV normalised to typical rates in Run~2 2015.
  \label{compAllBKG_6.52}}
\end{figure}




\begin{figure}
\begin{center}
  \includegraphics[width=0.492\textwidth]{figures/6500GeV/20MeV/PhiEnMu_BG_6500GeV_flat_20GeV_bs.pdf}
  \includegraphics[width=0.492\textwidth]{figures/BH_run2/b1/PhiEnMu_BH_6500GeV_haloB1_20MeV.pdf}      
\end{center}
\vspace{-0.6cm}
 \caption{Run~2 2015: Muons sorted by radius in beam-gas (left) and betatron halo of B1 (right) at the inteface plane.
  \label{fig:PhiEnMu}}
\end{figure}



\begin{figure}
\begin{center}
  \includegraphics[width=0.492\textwidth]{figures/6500GeV/20MeV/PhiEnMuPM_BG_6500GeV_flat_20GeV_bs.pdf}
    \includegraphics[width=0.492\textwidth]{figures/BH_run2/b1/PhiEnMuPM_BH_6500GeV_haloB1_20MeV.pdf}            
\end{center}
\vspace{-0.6cm}
 \caption{Run~2 2015: Energy of muons separated by their charge in beam-gas (left) and betatron halo (right) induced showers at the interface plane. A clear effect of the combination dipole is visible. 
  \label{fig:PhiEnMuPM}}
\end{figure}


\begin{figure}
\begin{center}
  \includegraphics[width=0.492\textwidth]{figures/6500GeV/20MeV/RadNMuons_BG_6500GeV_flat_20GeV_bs.pdf}
  \includegraphics[width=0.492\textwidth]{figures/BH_run2/b1/RadNMuons_BH_6500GeV_haloB1_20MeV.pdf}      
\end{center}
\vspace{-0.6cm}
 \caption{Run~2 2015: Radii of muons of different energies in beam-gas (left) and betatron halo (right) simulations.
  \label{fig:PhiEnMuComp}}
\end{figure}

% -------------- HL

\begin{figure}
\begin{center}
\includegraphics[width=0.495\textwidth]{figures/HL/tct5inrd/RadNDist_BH_HL_tct5inrdB2_20MeV.pdf}
\includegraphics[width=0.495\textwidth]{figures/HL/tct5inrd/RadEnDist_BH_HL_tct5inrdB2_20MeV.pdf}
\end{center}
\vspace{-0.6cm}
 \caption{B2 betatron halo induced particle distributions at the interface plane for the HL-LHC scenario in baseline configuration (TCT5s in, ATS round beam optics, \twosigmaret~settings).}
  \label{tct5inrdb2retr2}
\end{figure}



\begin{figure}
\centering
\includegraphics[width=0.4\textwidth]{figures/HL/compINOUTB1_retracted/perTCThit/ratioEkinAll}
\includegraphics[width=0.4\textwidth]{figures/HL/compINOUTB1_retracted/perTCThit/ratioEkinMuons}
\includegraphics[width=0.4\textwidth]{figures/HL/compINOUTB1_retracted/perTCThit/ratioPhiNAll}
\includegraphics[width=0.4\textwidth]{figures/HL/compINOUTB1_retracted/perTCThit/ratioPhiNMuons}
\includegraphics[width=0.4\textwidth]{figures/HL/compINOUTB1_retracted/perTCThit/ratioPhiEnAll}
\includegraphics[width=0.4\textwidth]{figures/HL/compINOUTB1_retracted/perTCThit/ratioPhiEnMuons}
 \caption{HL-LHC: Distributions of all particles (left) and muons (right) and their energy in the two cases, TCT4s only and TCT5s in, for B1.
  \label{fig:compInOutB1_perTCThit}}
\end{figure}




\begin{figure}
\begin{center}
\includegraphics[width=0.4\textwidth]{figures/HL/compINOUTB2_retracted/perTCThit/ratioEkinAll}
\includegraphics[width=0.4\textwidth]{figures/HL/compINOUTB2_retracted/perTCThit/ratioEkinMuons}
\includegraphics[width=0.4\textwidth]{figures/HL/compINOUTB2_retracted/perTCThit/ratioPhiNAll}
\includegraphics[width=0.4\textwidth]{figures/HL/compINOUTB2_retracted/perTCThit/ratioPhiNMuons}
\includegraphics[width=0.4\textwidth]{figures/HL/compINOUTB2_retracted/perTCThit/ratioPhiEnAll}
\includegraphics[width=0.4\textwidth]{figures/HL/compINOUTB2_retracted/perTCThit/ratioPhiEnMuons}
\end{center}
\vspace{-0.6cm}
 \caption{HL-LHC: Azimuthal distribution of all particles and muons (top) and their energy (bottom).
  \label{fig:compInOutB2}}
\end{figure}


\begin{figure}
\centering
\includegraphics[width=0.43\textwidth]{figures/HL/compNomRetrCollSett/normalised/ratioEkinAll.pdf}
\includegraphics[width=0.43\textwidth]{figures/HL/compNomRetrCollSett/normalised/ratioPhiEnAll.pdf}
 \caption{HL-LHC: Comparison of betatron halo-induced background showers with nominal design and baseline \twosigmaret~collimator settings. Overall halo reduction of a factor 2 to 3 can be obtained considering also the IR7 to IR1 conversion factors\label{fig:compNomRetrSett2}.}
\end{figure}


% --------- comparison of Run~1 to HL

\begin{figure}
  \begin{center}
    \includegraphics[width=0.495\textwidth]{figures/4TeV/bs_20MeV/XYNCharZoom_BG_4TeV_20MeV_bs.pdf}
    \includegraphics[width=0.495\textwidth]{figures/4TeV/bs_20MeV/XYNNeutronsE10_BG_4TeV_20MeV_bs.pdf}
    \includegraphics[width=0.495\textwidth]{figures/4TeV/haloB1_20MeV/XYNNeutronsE10_BH_4TeV_B1_20MeV.pdf}
    \includegraphics[width=0.495\textwidth]{figures/4TeV/haloB2_20MeV/XYNNeutronsE10_BH_4TeV_B2_20MeV.pdf}
    \includegraphics[width=0.495\textwidth]{figures/4TeV/haloB1_20MeV/XYNPhotonsZoom_BH_4TeV_B1_20MeV.pdf}
    \includegraphics[width=0.495\textwidth]{figures/4TeV/haloB2_20MeV/XYNPhotonsZoom_BH_4TeV_B2_20MeV.pdf}
    \includegraphics[width=0.495\textwidth]{figures/4TeV/haloB1_20MeV/XYNCharZoom_BH_4TeV_B1_20MeV.pdf}
    \includegraphics[width=0.495\textwidth]{figures/4TeV/haloB2_20MeV/XYNCharZoom_BH_4TeV_B2_20MeV.pdf}
    
    %% \includegraphics[width=0.495\textwidth]{figures/HL/tct5otrd/XYNNeutronsE10_BH_HL_tct5otrdB2_20MeV.pdf}
    %% \includegraphics[width=0.495\textwidth]{figures/HL/tct5inrd/XYNNeutronsE10_BH_HL_tct5inrdB1_20MeV.pdf}
    %% \includegraphics[width=0.495\textwidth]{figures/HL/tct5inrd/XYNCharZoom_BH_HL_tct5inrdB1_20MeV.pdf}
    %% \includegraphics[width=0.495\textwidth]{figures/HL/tct5inrd/XYNElecPosi_BH_HL_tct5inrdB1_20MeV.pdf}
    %% \includegraphics[width=0.495\textwidth]{figures/HL/tct5inrd/XYNProtonsE10_BH_HL_tct5inrdB1_20MeV.pdf}
    %% \includegraphics[width=0.495\textwidth]{figures/HL/tct5inrd/XYNPhotonsE10_BH_HL_tct5inrdB1_20MeV.pdf}
\end{center}
\vspace{-0.6cm}
 \caption{\fluka~particle distributions in the (xy)-plane at the interface plane normalised to 1/cm$^{2}$/TCT hit for various scenarios (see plot title), for charged particles (top left), and others (see description top right corner).
  \label{fig:XYNPho}}
\end{figure}

\begin{figure}
  \begin{center}
%    \includegraphics[width=0.495\textwidth]{figures/BH_run2/b1/XYNCharZoom_BH_6500GeV_haloB1_20MeV.pdf}
%    \includegraphics[width=0.495\textwidth]{figures/BH_run2/b2/XYNCharZoom_BH_6500GeV_haloB2_20MeV.pdf}  
    %\includegraphics[width=0.49\textwidth]{figures/4TeV/bs_20MeV/XYNCharZoom_BG_4TeV_20MeV_bs.pdf}
    %\includegraphics[width=0.49\textwidth]{figures/4TeV/bs_20MeV/XYNNeutronsE10_BG_4TeV_20MeV_bs.pdf}
    %\includegraphics[width=0.495\textwidth]{figures/4TeV/haloB1_20MeV/XYNCharZoom_BH_4TeV_B1_20MeV.pdf}
    \includegraphics[width=0.495\textwidth]{figures/HL/tct5inrd/XYNCharZoom_BH_HL_tct5inrdB1_20MeV.pdf}
    \includegraphics[width=0.495\textwidth]{figures/HL/tct5otrd/XYNCharZoom_BH_HL_tct5otrdB1_20MeV.pdf} 
    \includegraphics[width=0.495\textwidth]{figures/HL/tct5otrd/XYNNeutronsE10_BH_HL_tct5otrdB1_20MeV.pdf}
    \includegraphics[width=0.495\textwidth]{figures/HL/tct5inrd/XYNNeutronsE10_BH_HL_tct5inrdB1_20MeV.pdf}
    \includegraphics[width=0.495\textwidth]{figures/HL/tct5otrd/XYNPhotonsE10_BH_HL_tct5otrdB1_20MeV.pdf}
    \includegraphics[width=0.495\textwidth]{figures/HL/tct5inrd/XYNPhotonsE10_BH_HL_tct5inrdB1_20MeV.pdf}
    \includegraphics[width=0.495\textwidth]{figures/HL/tct5otrd/XYNPhotonsE10_BH_HL_tct5otrdB2_20MeV.pdf}
    \includegraphics[width=0.495\textwidth]{figures/HL/tct5inrd/XYNPhotonsE10_BH_HL_tct5inrdB2_20MeV.pdf}

%    \includegraphics[width=0.495\textwidth]{figures/HL/tct5inrd/XYNNeutronsE10_BH_HL_tct5inrdB2_20MeV.pdf}
\end{center}
\vspace{-0.6cm}
 \caption{\fluka~particle distributions in the (xy)-plane at the interface plane normalised to 1/cm$^{2}$/TCT hit for various scenarios (see plot title), for charged particles (top left and bottom), and others (see description top right corner).
  \label{fig:XYN}}
\end{figure}

 % run2 and HL
\newpage
\section{Additional plots for Sect.~\ref{evolut} \label{evolutApp}}
\begin{figure}
\begin{center}
  \includegraphics[width=0.492\textwidth]{figures/6500GeV/20MeV/PhiEnMu_BG_6500GeV_flat_20GeV_bs.pdf}
  \includegraphics[width=0.492\textwidth]{figures/BH_run2/b1/PhiEnMu_BH_6500GeV_haloB1_20MeV.pdf}      
\end{center}
\vspace{-0.6cm}
 \caption{Muons sorted by radius.
  \label{fig:PhiEnMu}}
\end{figure}



\begin{figure}
\begin{center}
  \includegraphics[width=0.492\textwidth]{figures/6500GeV/20MeV/PhiEnMuPM_BG_6500GeV_flat_20GeV_bs.pdf}
    \includegraphics[width=0.492\textwidth]{figures/BH_run2/b1/PhiEnMuPM_BH_6500GeV_haloB1_20MeV.pdf}            
\end{center}
\vspace{-0.6cm}
 \caption{Muons separated by charge. A clear effect of the combination dipole is visible. 
  \label{fig:PhiEnMuPM}}
\end{figure}



\begin{figure}
\begin{center}
  \includegraphics[width=0.42\textwidth]{figures/4TeV/compAllBKG/EkinAll.pdf}
  \includegraphics[width=0.42\textwidth]{figures/4TeV/compAllBKG/PhiEnAll.pdf}
%  \includegraphics[width=0.42\textwidth]{figures/4TeV/compAllBKG/EkinMuons.pdf}
%  \includegraphics[width=0.42\textwidth]{figures/4TeV/compAllBKG/PhiEnMuons.pdf}
  \includegraphics[width=0.42\textwidth]{figures/4TeV/compAllBKG/EkinProtons.pdf}
  \includegraphics[width=0.42\textwidth]{figures/4TeV/compAllBKG/PhiEnProtons.pdf}
  \includegraphics[width=0.42\textwidth]{figures/4TeV/compAllBKG/EkinPhotons.pdf}
  \includegraphics[width=0.42\textwidth]{figures/4TeV/compAllBKG/PhiEnPhotons.pdf}
\end{center}
\vspace{-0.6cm}
 \caption{Comparison of all background sources at 4 TeV normalised per interaction showing energy spectrum and energy in $\phi$.
  \label{fig:compAllBKG_perInt1}}
\end{figure}

\begin{figure}
\begin{center}
  \includegraphics[width=0.42\textwidth]{figures/4TeV/compAllBKG/RadNAll.pdf}
  \includegraphics[width=0.42\textwidth]{figures/4TeV/compAllBKG/RadEnAll.pdf}
  \includegraphics[width=0.42\textwidth]{figures/4TeV/compAllBKG/RadNMuons.pdf}
  \includegraphics[width=0.42\textwidth]{figures/4TeV/compAllBKG/RadEnMuons.pdf}
  \includegraphics[width=0.42\textwidth]{figures/4TeV/compAllBKG/RadNProtons.pdf}
  \includegraphics[width=0.42\textwidth]{figures/4TeV/compAllBKG/RadEnProtons.pdf}
  \includegraphics[width=0.42\textwidth]{figures/4TeV/compAllBKG/RadNPhotons.pdf}
  \includegraphics[width=0.42\textwidth]{figures/4TeV/compAllBKG/RadEnPhotons.pdf}
\end{center}
\vspace{-0.6cm}
 \caption{Comparison of all background sources at 4 TeV normalised per interaction showing radial distributions and energy in $r$.
  \label{fig:compAllBKG_perInt2}}
\end{figure}

\begin{figure}
\begin{center}
  \includegraphics[width=0.42\textwidth]{figures/4TeV/reweighted/cv78_EkinAll.pdf}
  \includegraphics[width=0.42\textwidth]{figures/4TeV/reweighted/cv78_PhiEnAll.pdf}
%  \includegraphics[width=0.42\textwidth]{figures/4TeV/reweighted/cv78_EkinMuons.pdf}
%  \includegraphics[width=0.42\textwidth]{figures/4TeV/reweighted/cv78_PhiEnMuons.pdf}
  \includegraphics[width=0.42\textwidth]{figures/4TeV/reweighted/cv78_EkinProtons.pdf}
  \includegraphics[width=0.42\textwidth]{figures/4TeV/reweighted/cv78_PhiEnProtons.pdf}
  \includegraphics[width=0.42\textwidth]{figures/4TeV/reweighted/cv78_EkinPhotons.pdf}
  \includegraphics[width=0.42\textwidth]{figures/4TeV/reweighted/cv78_PhiEnPhotons.pdf}
\end{center}
\vspace{-0.6cm}
 \caption{comparison of all background sources at 4 TeV normalised to a rate.
  \label{compAllBKG4TeV_rates}}
\end{figure}

\begin{figure}
\begin{center}
  \includegraphics[width=0.42\textwidth]{figures/4TeV/reweighted/cv78_RadNAll.pdf}
  \includegraphics[width=0.42\textwidth]{figures/4TeV/reweighted/cv78_RadEnAll.pdf}
  \includegraphics[width=0.42\textwidth]{figures/4TeV/reweighted/cv78_RadNMuons.pdf}
  \includegraphics[width=0.42\textwidth]{figures/4TeV/reweighted/cv78_RadEnMuons.pdf}
  \includegraphics[width=0.42\textwidth]{figures/4TeV/reweighted/cv78_RadNProtons.pdf}
  \includegraphics[width=0.42\textwidth]{figures/4TeV/reweighted/cv78_RadEnProtons.pdf}
  \includegraphics[width=0.42\textwidth]{figures/4TeV/reweighted/cv78_RadNPhotons.pdf}
  \includegraphics[width=0.42\textwidth]{figures/4TeV/reweighted/cv78_RadEnPhotons.pdf}
\end{center}
\vspace{-0.6cm}
 \caption{comparison of all background sources at 4 TeV normalised to a rate.
  \label{compAllBKG4TeV_rates2}}
\end{figure}

\begin{figure}
\begin{center}
  \includegraphics[width=0.8\textwidth]{figures/cv87_allenergies_OrigZAll.pdf}
%  \includegraphics[width=0.8\textwidth]{figures/cv87_allenergies_OrigZMuon.pdf}
\end{center}
\vspace{-0.6cm}
 \caption{
  \label{fig:OrigZMuonAllEn2}} 
\end{figure}



\begin{figure}
\begin{center}
  \includegraphics[width=0.42\textwidth]{figures/6500GeV/reweighted/cv78_EkinAll.pdf}
  \includegraphics[width=0.42\textwidth]{figures/6500GeV/reweighted/cv78_PhiEnAll.pdf}
%  \includegraphics[width=0.42\textwidth]{figures/6500GeV/reweighted/cv78_EkinMuons.pdf}
 % \includegraphics[width=0.42\textwidth]{figures/6500GeV/reweighted/cv78_PhiEnMuons.pdf}
  \includegraphics[width=0.42\textwidth]{figures/6500GeV/reweighted/cv78_EkinProtons.pdf}
  \includegraphics[width=0.42\textwidth]{figures/6500GeV/reweighted/cv78_PhiEnProtons.pdf}
 \includegraphics[width=0.42\textwidth]{figures/6500GeV/reweighted/cv78_EkinPhotons.pdf}
 \includegraphics[width=0.42\textwidth]{figures/6500GeV/reweighted/cv78_PhiEnPhotons.pdf}
\end{center}
\vspace{-0.6cm}
 \caption{comparison of all background sources at 6.5~TeV.
  \label{compAllBKG_6.52}}
\end{figure}

\begin{figure}%[!htb]
\centering
\includegraphics[width=0.45\textwidth]{figures/compBGreweighted/ratioEkinAll.pdf}
%\includegraphics[width=0.45\textwidth]{figures/compBGreweighted/ratioEkinMuons.pdf}
\includegraphics[width=0.45\textwidth]{figures/compBGreweighted/ratioPhiEnAll.pdf}
%\includegraphics[width=0.45\textwidth]{figures/compBGreweighted/ratioPhiEnMuons.pdf}
\caption{Reweighted beam-gas distributions in the 2012 Run I and 2015 Run II scenario for all particles and muons showing the energy spectrum (top) and the azimuthal distribution (bottom).
  \label{fig:compBGreweighted12}}
\end{figure}




\begin{figure}%[!htb]
\centering
\includegraphics[width=0.45\textwidth]{figures/compBGreweighted/ratioRadNAll.pdf}
\includegraphics[width=0.45\textwidth]{figures/compBGreweighted/ratioRadNMuons.pdf}
\includegraphics[width=0.45\textwidth]{figures/compBGreweighted/ratioRadEnAll.pdf}
\includegraphics[width=0.45\textwidth]{figures/compBGreweighted/ratioRadEnMuons.pdf}
\caption{Reweighted beam-gas distributions in the 2012 Run I and 2015 Run II scenario for all particles and muons showing radial positions and energy in $r$.
  \label{fig:compBGreweighted2}}
\end{figure}



\begin{figure}
  \begin{center}
  \includegraphics[width=0.4\textwidth]{figures/compBHB1_4TeV_vs_6p5TeV/perTCThit/ratioEkinMuons.pdf}
  \includegraphics[width=0.4\textwidth]{figures/compBHB1_4TeV_vs_6p5TeV/perTCThit/ratioPhiEnMuons.pdf}
  \includegraphics[width=0.41\textwidth]{figures/compBHB2_4TeV_vs_6p5TeV/perTCThit/ratioEkinMuons.pdf}
  \includegraphics[width=0.41\textwidth]{figures/compBHB2_4TeV_vs_6p5TeV/perTCThit/ratioPhiEnMuons.pdf}
\end{center}
\vspace{-0.6cm}
 \caption{Similar to Fig.~\ref{fig:compBHrun1run2} but shown per TCT interactions for B1 (top) and B2 (bottom).
  \label{fig:compBHrun1run2PerTCT}}
\end{figure}



\begin{figure}%[!htb]
\begin{center}
  \includegraphics[width=0.49\textwidth]{figures/compBHB1_4TeV_vs_6p5TeV/normalised/ratioEkinAll.pdf}
  \includegraphics[width=0.49\textwidth]{figures/compBHB1_4TeV_vs_6p5TeV/normalised/ratioPhiEnAll.pdf}
  \includegraphics[width=0.49\textwidth]{figures/compBHB2_4TeV_vs_6p5TeV/normalised/ratioEkinAll.pdf}
  \includegraphics[width=0.49\textwidth]{figures/compBHB2_4TeV_vs_6p5TeV/normalised/ratioPhiEnAll.pdf}
\end{center}
\vspace{-0.6cm}
 \caption{Comparison of halo induced background at 4 and 6.5~TeV of properties of all particles at the interface plane for B1 (top) and B2 (bottom).
  \label{compBHrun1run22}}
\end{figure}



\begin{figure}
\begin{center}
  \includegraphics[width=0.42\textwidth]{figures/HLRunII/cv78_EkinAll.pdf}
  \includegraphics[width=0.42\textwidth]{figures/HLRunII/cv78_PhiEnAll.pdf}
%  \includegraphics[width=0.42\textwidth]{figures/HLRunII/cv78_EkinMuons.pdf}
%  \includegraphics[width=0.42\textwidth]{figures/HLRunII/cv78_PhiEnMuons.pdf}
  \includegraphics[width=0.42\textwidth]{figures/HLRunII/cv78_EkinProtons.pdf}
  \includegraphics[width=0.42\textwidth]{figures/HLRunII/cv78_PhiEnProtons.pdf}
  \includegraphics[width=0.42\textwidth]{figures/HLRunII/cv78_EkinPhotons.pdf}
  \includegraphics[width=0.42\textwidth]{figures/HLRunII/cv78_PhiEnPhotons.pdf}
\end{center}
\vspace{-0.6cm}
 \caption{Comparison of beam-gas (BG) in Run II and beam-halo in HL using the baseline layout (TCT5s in, \twosigmaret~settings) and round beam optics.
  \label{fig:hlrun22}}
\end{figure}



\begin{figure}
  \centering
    %\includegraphics[width=0.495\textwidth]{figures/XYNMuons_BG_4TeV_20MeV_bs.pdf}
    \includegraphics[width=0.495\textwidth]{figures/XYNMuons_BG_6500GeV_flat_20GeV_bs.pdf}  
    %\includegraphics[width=0.495\textwidth]{figures/XYNMuons_BH_4TeV_B1_20MeV.pdf}
    \includegraphics[width=0.495\textwidth]{figures/XYNMuons_BH_6500GeV_haloB1_20MeV.pdf}
  \caption{Spatial distribution of muons in an beam-gas (left) and beam-halo (right) scenario for a 6.5~TeV beam in Run II 2015. 
    \label{fig:XYNMuons2}}
\end{figure}


\clearpage
\begin{figure}
\begin{center}
  \includegraphics[width=0.492\textwidth]{figures/6500GeV/20MeV/RadNMuons_BG_6500GeV_flat_20GeV_bs.pdf}
  \includegraphics[width=0.492\textwidth]{figures/BH_run2/b1/RadNMuons_BH_6500GeV_haloB1_20MeV.pdf}      
\end{center}
\vspace{-0.6cm}
 \caption{Radii of muons of different energies.
  \label{fig:PhiEnMuComp}}
\end{figure}

\begin{figure}
\begin{center}
  \includegraphics[width=0.492\textwidth]{figures/HLRunII/cv78_EkinAll.pdf}
  \includegraphics[width=0.492\textwidth]{figures/HLRunII/cv78_PhiEnAll.pdf}
  \includegraphics[width=0.492\textwidth]{figures/HLRunII/cv78_RadNAll.pdf}
  \includegraphics[width=0.492\textwidth]{figures/HLRunII/cv78_RadEnAll.pdf}      
\end{center}
\vspace{-0.6cm}
 \caption{Comparison of properties of all particles in Run~II and HL.
  \label{fig:compHLRun2All}}
\end{figure}
 % evolutApp
\newpage
\section{Appendix A}

\subsection{Loss map compendium}

% ---------------------------------------------------------------------------------------------------------------------
% fullring round/flat

\begin{figure}
\begin{center}
\vskip-12mm
\includegraphics[width=0.92\textwidth]{figures/coll_loss_H5_HL_TCT5LOUT_relaxColl_hHaloB1_roundthin_fullring}
\includegraphics[width=0.92\textwidth]{figures/coll_loss_H5_HL_TCT5LOUT_relaxColl_vHaloB1_roundthin_fullring}
\end{center}
\vspace{-0.3cm}
 \caption{Loss maps for horizontal (top) and vertical (bottom) B1 halo using round optics with TCT4s only.
  \label{fullring_roundB1_TCT5LOUT }}
\end{figure}

\begin{figure}
\begin{center}
\vskip-12mm
\includegraphics[width=0.92\textwidth]{figures/coll_loss_H5_HL_TCT5IN_relaxColl_hHaloB1_roundthin_fullring}
\includegraphics[width=0.92\textwidth]{figures/coll_loss_H5_HL_TCT5IN_relaxColl_vHaloB1_roundthin_fullring}
\end{center}
\vspace{-0.3cm}
 \caption{Loss maps for horizontal (top) and vertical (bottom) B1 halo using round optics with TCT4s and TCT5s.
  \label{fullring_roundB1_TCT5IN}}
\end{figure}


\begin{figure}
\begin{center}
\vskip-12mm
\includegraphics[width=0.92\textwidth]{figures/coll_loss_H5_HL_TCT5LOUT_relaxColl_hHaloB1_flatthin_fullring}
\includegraphics[width=0.92\textwidth]{figures/coll_loss_H5_HL_TCT5LOUT_relaxColl_vHaloB1_flatthin_fullring}
\end{center}
\vspace{-0.3cm}
 \caption{Loss maps for horizontal (top) and vertical (bottom) B1 halo using flat optics with TCT4s only.
  \label{fullring_flatB1_TCT5LOUT}}
\end{figure}

\begin{figure}
\begin{center}
\vskip-12mm
\includegraphics[width=0.92\textwidth]{figures/coll_loss_H5_HL_TCT5IN_relaxColl_hHaloB1_flatthin_fullring}
\includegraphics[width=0.92\textwidth]{figures/coll_loss_H5_HL_TCT5IN_relaxColl_vHaloB1_flatthin_fullring}
\end{center}
\vspace{-0.3cm}
 \caption{Loss maps for horizontal (top) and vertical (bottom) B1 halo using flat optics with TCT4s and TCT5s.
  \label{fullring_flatB1_TCT5IN}}
\end{figure}

% ---------------------------------------------------------------------------------------------------------------------
% zooms

\begin{figure}
\begin{center}
\vskip-12mm
\includegraphics[width=0.48\textwidth]{figures/coll_loss_H5_HL_TCT5LOUT_relaxColl_hHaloB1_roundthin_IR1}
\includegraphics[width=0.48\textwidth]{figures/coll_loss_H5_HL_TCT5LOUT_relaxColl_vHaloB1_roundthin_IR1}
\includegraphics[width=0.48\textwidth]{figures/coll_loss_H5_HL_TCT5IN_relaxColl_hHaloB1_roundthin_IR1}
\includegraphics[width=0.48\textwidth]{figures/coll_loss_H5_HL_TCT5IN_relaxColl_vHaloB1_roundthin_IR1}
\end{center}
\begin{picture} (0.,0.)
\setlength{\unitlength}{1.0cm}
\small{
    \put ( 4.,7.35){(a)}
    \put ( 12.4,7.35){(b)}
    \put ( 4.,1.){(c)}
    \put ( 12.4,1.){(d)}
}
\end{picture}
\vspace{-0.3cm}
 \caption{Zoom into IR1 for horizontal (top) and vertical (bottom) B1 halo using round optics with TCT4s only and both.
  \label{IR1_roundB1_TCT5LOUT }}
\end{figure}

\begin{figure}
\begin{center}
\vskip-12mm
\includegraphics[width=0.48\textwidth]{figures/coll_loss_H5_HL_TCT5LOUT_relaxColl_hHaloB1_roundthin_IR5}
\includegraphics[width=0.48\textwidth]{figures/coll_loss_H5_HL_TCT5LOUT_relaxColl_vHaloB1_roundthin_IR5}
\includegraphics[width=0.48\textwidth]{figures/coll_loss_H5_HL_TCT5IN_relaxColl_hHaloB1_roundthin_IR5}
\includegraphics[width=0.48\textwidth]{figures/coll_loss_H5_HL_TCT5IN_relaxColl_vHaloB1_roundthin_IR5}
\end{center}
\vspace{-0.3cm}
 \caption{Zoom into IR5 for horizontal (top) and vertical (bottom) B1 halo using round optics with TCT4s only.
  \label{IR5_roundB1_TCT5LOUT }}
\end{figure}

\begin{figure}
\begin{center}
\vskip-12mm
\includegraphics[width=0.48\textwidth]{figures/coll_loss_H5_HL_TCT5LOUT_relaxColl_hHaloB1_flatthin_IR1}
\includegraphics[width=0.48\textwidth]{figures/coll_loss_H5_HL_TCT5LOUT_relaxColl_vHaloB1_flatthin_IR1}
\includegraphics[width=0.48\textwidth]{figures/coll_loss_H5_HL_TCT5IN_relaxColl_hHaloB1_flatthin_IR1}
\includegraphics[width=0.48\textwidth]{figures/coll_loss_H5_HL_TCT5IN_relaxColl_vHaloB1_flatthin_IR1}
\end{center}
\begin{picture} (0.,0.)
\setlength{\unitlength}{1.0cm}
\small{
    \put ( 4.,7.35){(a)}
    \put ( 12.4,7.35){(b)}
    \put ( 4.,1.){(c)}
    \put ( 12.4,1.){(d)}
}
\end{picture}
\vspace{-0.3cm}
 \caption{Zoom into IR1 for horizontal (top) and vertical (bottom) B1 halo using flat optics with TCT4s and TCT5s.
  \label{IR1_flatB1_TCT5IN}}
\end{figure}


\begin{figure}
\begin{center}
\vskip-12mm
\includegraphics[width=0.48\textwidth]{figures/coll_loss_H5_HL_TCT5LOUT_relaxColl_hHaloB1_flatthin_IR5}
\includegraphics[width=0.48\textwidth]{figures/coll_loss_H5_HL_TCT5LOUT_relaxColl_vHaloB1_flatthin_IR5}
\includegraphics[width=0.48\textwidth]{figures/coll_loss_H5_HL_TCT5IN_relaxColl_hHaloB1_flatthin_IR5}
\includegraphics[width=0.48\textwidth]{figures/coll_loss_H5_HL_TCT5IN_relaxColl_vHaloB1_flatthin_IR5}
\end{center}
\begin{picture} (0.,0.)
\setlength{\unitlength}{1.0cm}
\small{
    \put ( 4.,7.35){(a)}
    \put ( 12.4,7.35){(b)}
    \put ( 4.,1.){(c)}
    \put ( 12.4,1.){(d)}
}
\end{picture}
\vspace{-0.3cm}
 \caption{Zoom into IR5 for horizontal (top) and vertical (bottom) B1 halo using flat optics with TCT4s only.
  \label{IR5_flatB1_TCT5LOUT}}
\end{figure}

% ---------------------------------------------------------------------------------------------------------------------
% s-optics

\begin{figure}
\begin{center}
\vskip-12mm
\includegraphics[width=0.92\textwidth]{figures/coll_loss_H5_HL_TCT5IN_relaxColl_hHaloB1_sroundthin_fullring}
\includegraphics[width=0.92\textwidth]{figures/coll_loss_H5_HL_TCT5IN_relaxColl_hHaloB1_sflatthin_fullring}
\end{center}
\vspace{-0.3cm}
 \caption{Loss maps for horizontal B1 halo using s-round (top) and s-flat optics with TCT4s and TCT5s.
  \label{fullring_sB1_TCT5IN}}
\end{figure}
 % HLapp: loss maps
% ---------------------------------------------------------------------------------------------------------------------

\newpage

        \bibliographystyle{ieeetr-new}
        \bibliography{bibliography}

\end{document}
