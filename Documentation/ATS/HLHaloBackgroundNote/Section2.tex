\section{Conclusion and outlook}

We investigated in detail and systematically two sources of background, beam-gas and beam-halo, the most relevant sources for the experiments. The study was carried out for IR1 and both beams in general. Although there is a layout symmetry for incoming and outgoing beams in IR1 and IR5, both beams have a different ``history'' which will result in different backround patterns. One can see from Tab.~\ref{leakageFactorsIR7}, B2 gives systematically higher contributions than B1 in IR1 in all simulated cases from that table. Although IR5 leakages from IR7 can be significantly smaller than the leakage to IR1, this is not necessarily generally the case: it is true for 2015 Run~II and the HL reference case (TCT5s in, round beam optics, \twosigmaret~settings), but not for the Run~I case at 4~TeV when the leakage is on a similar level for B1 and slightly higher for B2 ($2.6 \times 10^{-5}$). Tracking simulations with the main settings for proton physics in 2016 were recently performed and one can see from in Tab.~\ref{2016leakageFactorsIR7} that the different optics can cause B2 to contribute only about a third of B1 despite the shorter path length freshly cleaned from IR7. These are examples that show that a more detailed analysis e.g.~considering the phase advance of the collimators are needed to understand leakages to the experimental areas.




While in Run I at 3.5~TeV and still up to 2012 at 4~TeV beam operation background induced from local beam-gas interactions has been the dominating source it can actually change in HL, where rates become similar depending. 
Implications for the machine.
Measures which help to improve the beam life time help also in terms of reduction of halo background like machine conditioning. 
