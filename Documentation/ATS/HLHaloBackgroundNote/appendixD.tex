\section{Additional plots for Section~\ref{evolut} \label{evolutApp}}
\begin{figure}[!htb]
\begin{center}
  \includegraphics[width=0.8\textwidth]{figures/cv87_allenergies_OrigZAll.pdf}
%  \includegraphics[width=0.8\textwidth]{figures/cv87_allenergies_OrigZMuon.pdf}
\end{center}
\vspace{-0.6cm}
 \caption{Similar as Fig.~\ref{fig:OrigZMuon}, but for all particles. The spike in the 4~TeV curve at s~$\approx$~525~m is due to local photon production.
  \label{fig:OrigZAll}} 
\end{figure}





\begin{figure}
\centering
  \includegraphics[width=0.49\textwidth]{figures/compBHB1_4TeV_vs_6p5TeV/normalised/ratioEkinAll.pdf}
%  \includegraphics[width=0.49\textwidth]{figures/compBHB1_4TeV_vs_6p5TeV/normalised/ratioEkinMuons.pdf}
  \includegraphics[width=0.49\textwidth]{figures/compBHB1_4TeV_vs_6p5TeV/normalised/ratioPhiNAll.pdf}
  \includegraphics[width=0.49\textwidth]{figures/compBHB1_4TeV_vs_6p5TeV/normalised/ratioPhiEnAll.pdf}
%  \includegraphics[width=0.49\textwidth]{figures/compBHB1_4TeV_vs_6p5TeV/normalised/ratioPhiEnMuons.pdf}
  \includegraphics[width=0.49\textwidth]{figures/compBHB1_4TeV_vs_6p5TeV/normalised/ratioRadEnAll.pdf}
\vspace{-0.6cm}
 \caption{Comparison of betatron halo induced background at 4 and 6.5~TeV in the azimuthal distributions of all particles at the interface plane showing the energy spectra (top left), multiplicities in $\phi$ (top right), energy in $\phi$ (bottom left) and energy in $r$ (bottom right).
  \label{compBHB1run1run22}}
\end{figure}









\begin{figure}%[!htb]
\centering
\includegraphics[width=0.45\textwidth]{figures/compBGreweighted/ratioEkinAll.pdf}
%\includegraphics[width=0.45\textwidth]{figures/compBGreweighted/ratioEkinMuons.pdf}
\includegraphics[width=0.45\textwidth]{figures/compBGreweighted/ratioPhiEnAll.pdf}
%\includegraphics[width=0.45\textwidth]{figures/compBGreweighted/ratioPhiEnMuons.pdf}
\caption{Reweighted beam-gas distributions in the 2012 Run I and 2015 Run II scenario for all particles and muons showing the energy spectrum (top) and the azimuthal distribution (bottom).
  \label{fig:compBGreweighted12}}
\end{figure}




\begin{figure}%[!htb]
\centering
\includegraphics[width=0.45\textwidth]{figures/compBGreweighted/ratioRadNAll.pdf}
\includegraphics[width=0.45\textwidth]{figures/compBGreweighted/ratioRadNMuons.pdf}
\includegraphics[width=0.45\textwidth]{figures/compBGreweighted/ratioRadEnAll.pdf}
\includegraphics[width=0.45\textwidth]{figures/compBGreweighted/ratioRadEnMuons.pdf}
\caption{Reweighted beam-gas distributions in the 2012 Run I and 2015 Run II scenario for all particles and muons showing radial positions and energy in $r$.
  \label{fig:compBGreweighted2}}
\end{figure}



\begin{figure}
  \begin{center}
  \includegraphics[width=0.49\textwidth]{figures/compBHB1_4TeV_vs_6p5TeV/perTCThit/ratioEkinMuons.pdf}
  \includegraphics[width=0.49\textwidth]{figures/compBHB1_4TeV_vs_6p5TeV/perTCThit/ratioPhiEnMuons.pdf}
  \includegraphics[width=0.49\textwidth]{figures/compBHB2_4TeV_vs_6p5TeV/perTCThit/ratioEkinMuons.pdf}
  \includegraphics[width=0.49\textwidth]{figures/compBHB2_4TeV_vs_6p5TeV/perTCThit/ratioPhiEnMuons.pdf}
\end{center}
\vspace{-0.6cm}
 \caption{Similar to Fig.~\ref{fig:compBHrun1run2} but shown per TCT interactions for B1 (top) and B2 (bottom).
  \label{fig:compBHrun1run2PerTCT}}
\end{figure}



\begin{figure}%[!htb]
\begin{center}
  \includegraphics[width=0.49\textwidth]{figures/compBHB1_4TeV_vs_6p5TeV/normalised/ratioEkinAll.pdf}
  \includegraphics[width=0.49\textwidth]{figures/compBHB1_4TeV_vs_6p5TeV/normalised/ratioPhiEnAll.pdf}
  \includegraphics[width=0.49\textwidth]{figures/compBHB2_4TeV_vs_6p5TeV/normalised/ratioEkinAll.pdf}
  \includegraphics[width=0.49\textwidth]{figures/compBHB2_4TeV_vs_6p5TeV/normalised/ratioPhiEnAll.pdf}
\end{center}
\vspace{-0.6cm}
 \caption{Comparison of halo induced background at 4 and 6.5~TeV of properties of all particles at the interface plane for B1 (top) and B2 (bottom).
  \label{compBHrun1run22}}
\end{figure}



\begin{figure}
  \centering
    %\includegraphics[width=0.495\textwidth]{figures/XYNMuons_BG_4TeV_20MeV_bs.pdf}
    \includegraphics[width=0.495\textwidth]{figures/XYNMuons_BG_6500GeV_flat_20GeV_bs.pdf}  
    %\includegraphics[width=0.495\textwidth]{figures/XYNMuons_BH_4TeV_B1_20MeV.pdf}
    \includegraphics[width=0.495\textwidth]{figures/XYNMuons_BH_6500GeV_haloB1_20MeV.pdf}
  \caption{Spatial distribution of muons in an beam-gas (left) and beam-halo (right) scenario for a 6.5~TeV beam in Run II 2015. 
    \label{fig:XYNMuons2}}
\end{figure}

\begin{figure}
\begin{center}
  \includegraphics[width=0.492\textwidth]{figures/HLRunII/cv78_EkinAll.pdf}
  \includegraphics[width=0.492\textwidth]{figures/HLRunII/cv78_PhiEnAll.pdf}
%  \includegraphics[width=0.492\textwidth]{figures/HLRunII/cv78_EkinMuons.pdf}
%  \includegraphics[width=0.492\textwidth]{figures/HLRunII/cv78_PhiEnMuons.pdf}
  \includegraphics[width=0.492\textwidth]{figures/HLRunII/cv78_EkinProtons.pdf}
  \includegraphics[width=0.492\textwidth]{figures/HLRunII/cv78_PhiEnProtons.pdf}
  \includegraphics[width=0.492\textwidth]{figures/HLRunII/cv78_EkinPhotons.pdf}
  \includegraphics[width=0.492\textwidth]{figures/HLRunII/cv78_PhiEnPhotons.pdf}
\end{center}
\vspace{-0.6cm}
 \caption{Comparison of beam-gas (BG) in Run II and beam-halo in HL using the baseline layout (TCT5s in, \twosigmaret~settings) and round beam optics.
  \label{fig:hlrun22}}
\end{figure}




\clearpage

\begin{figure}
\begin{center}
  \includegraphics[width=0.492\textwidth]{figures/HLRunII/cv78_EkinAll.pdf}
  \includegraphics[width=0.492\textwidth]{figures/HLRunII/cv78_PhiEnAll.pdf}
  \includegraphics[width=0.492\textwidth]{figures/HLRunII/cv78_RadNAll.pdf}
  \includegraphics[width=0.492\textwidth]{figures/HLRunII/cv78_RadEnAll.pdf}      
\end{center}
\vspace{-0.6cm}
 \caption{Comparison of properties of all particles at the interface plane in the 2015 Run~II and HL-LHC scenario.
  \label{fig:compHLRun2All}}
\end{figure}

%------------------

%% \begin{figure}
%% \begin{center}
%%   \includegraphics[width=0.49\textwidth]{figures/4TeV/bs_20MeV/RadNMuons_BG_4TeV_20MeV_bs.pdf}
%%   \includegraphics[width=0.49\textwidth]{figures/6500GeV/20MeV/RadNMuons_BG_6500GeV_flat_20MeV_bs.pdf}
%%   \includegraphics[width=0.49\textwidth]{figures/4TeV/haloB1_20MeV/RadNMuons_BH_4TeV_B1_20MeV.pdf}
%%   \includegraphics[width=0.49\textwidth]{figures/4TeV/haloB2_20MeV/RadNMuons_BH_4TeV_B2_20MeV.pdf}
%%   \includegraphics[width=0.49\textwidth]{figures/4TeV/offmom/20MeV/RadNMuons_offplus500Hz_4TeV_B2_20MeV.pdf}
%%   \includegraphics[width=0.49\textwidth]{figures/4TeV/offmom/20MeV/RadNMuons_offmin500Hz_4TeV_B2_20MeV.pdf}
%%   \includegraphics[width=0.49\textwidth]{figures/BH_run2/b1/RadNMuons_BH_6500GeV_haloB1_20MeV.pdf}
%%   \includegraphics[width=0.49\textwidth]{figures/BH_run2/b2/RadNMuons_BH_6500GeV_haloB2_20MeV.pdf}
%% \end{center}
%% \vspace{-0.6cm}
%%  \caption{Comparison 
%%   \label{compRadNMuonsRun12}}
%% \end{figure}



% ----------------------


%% \begin{figure}
%%   \centering
%% %  \includegraphics[width=0.495\textwidth]{figures/xzTCT5s.pdf}
%% %  \includegraphics[scale=0.49]{figures/HL/xy_z2760_-1.pdf}
%% %  \includegraphics[width=0.49\textwidth]{figures/4TeV/bs_20MeV/XYNMuons_BG_4TeV_20MeV_bs.pdf}
%%   \caption{
%%     \label{}}
%% \end{figure}
