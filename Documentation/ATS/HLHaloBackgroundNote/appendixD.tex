\section{Additional plots for Sect.~\ref{evolut} \label{evolutApp}}
\begin{figure}
\begin{center}
  \includegraphics[width=0.492\textwidth]{figures/6500GeV/20MeV/PhiEnMu_BG_6500GeV_flat_20GeV_bs.pdf}
  \includegraphics[width=0.492\textwidth]{figures/BH_run2/b1/PhiEnMu_BH_6500GeV_haloB1_20MeV.pdf}      
\end{center}
\vspace{-0.6cm}
 \caption{Muons sorted by radius.
  \label{fig:PhiEnMu}}
\end{figure}



\begin{figure}
\begin{center}
  \includegraphics[width=0.492\textwidth]{figures/6500GeV/20MeV/PhiEnMuPM_BG_6500GeV_flat_20GeV_bs.pdf}
    \includegraphics[width=0.492\textwidth]{figures/BH_run2/b1/PhiEnMuPM_BH_6500GeV_haloB1_20MeV.pdf}            
\end{center}
\vspace{-0.6cm}
 \caption{Muons separated by charge. A clear effect of the combination dipole is visible. 
  \label{fig:PhiEnMuPM}}
\end{figure}



\begin{figure}
\begin{center}
  \includegraphics[width=0.42\textwidth]{figures/4TeV/compAllBKG/EkinAll.pdf}
  \includegraphics[width=0.42\textwidth]{figures/4TeV/compAllBKG/PhiEnAll.pdf}
%  \includegraphics[width=0.42\textwidth]{figures/4TeV/compAllBKG/EkinMuons.pdf}
%  \includegraphics[width=0.42\textwidth]{figures/4TeV/compAllBKG/PhiEnMuons.pdf}
  \includegraphics[width=0.42\textwidth]{figures/4TeV/compAllBKG/EkinProtons.pdf}
  \includegraphics[width=0.42\textwidth]{figures/4TeV/compAllBKG/PhiEnProtons.pdf}
  \includegraphics[width=0.42\textwidth]{figures/4TeV/compAllBKG/EkinPhotons.pdf}
  \includegraphics[width=0.42\textwidth]{figures/4TeV/compAllBKG/PhiEnPhotons.pdf}
\end{center}
\vspace{-0.6cm}
 \caption{Comparison of all background sources at 4 TeV normalised per interaction showing energy spectrum and energy in $\phi$.
  \label{fig:compAllBKG_perInt1}}
\end{figure}

\begin{figure}
\begin{center}
  \includegraphics[width=0.42\textwidth]{figures/4TeV/compAllBKG/RadNAll.pdf}
  \includegraphics[width=0.42\textwidth]{figures/4TeV/compAllBKG/RadEnAll.pdf}
  \includegraphics[width=0.42\textwidth]{figures/4TeV/compAllBKG/RadNMuons.pdf}
  \includegraphics[width=0.42\textwidth]{figures/4TeV/compAllBKG/RadEnMuons.pdf}
  \includegraphics[width=0.42\textwidth]{figures/4TeV/compAllBKG/RadNProtons.pdf}
  \includegraphics[width=0.42\textwidth]{figures/4TeV/compAllBKG/RadEnProtons.pdf}
  \includegraphics[width=0.42\textwidth]{figures/4TeV/compAllBKG/RadNPhotons.pdf}
  \includegraphics[width=0.42\textwidth]{figures/4TeV/compAllBKG/RadEnPhotons.pdf}
\end{center}
\vspace{-0.6cm}
 \caption{Comparison of all background sources at 4 TeV normalised per interaction showing radial distributions and energy in $r$.
  \label{fig:compAllBKG_perInt2}}
\end{figure}

\begin{figure}
\begin{center}
  \includegraphics[width=0.42\textwidth]{figures/4TeV/reweighted/cv78_EkinAll.pdf}
  \includegraphics[width=0.42\textwidth]{figures/4TeV/reweighted/cv78_PhiEnAll.pdf}
%  \includegraphics[width=0.42\textwidth]{figures/4TeV/reweighted/cv78_EkinMuons.pdf}
%  \includegraphics[width=0.42\textwidth]{figures/4TeV/reweighted/cv78_PhiEnMuons.pdf}
  \includegraphics[width=0.42\textwidth]{figures/4TeV/reweighted/cv78_EkinProtons.pdf}
  \includegraphics[width=0.42\textwidth]{figures/4TeV/reweighted/cv78_PhiEnProtons.pdf}
  \includegraphics[width=0.42\textwidth]{figures/4TeV/reweighted/cv78_EkinPhotons.pdf}
  \includegraphics[width=0.42\textwidth]{figures/4TeV/reweighted/cv78_PhiEnPhotons.pdf}
\end{center}
\vspace{-0.6cm}
 \caption{comparison of all background sources at 4 TeV normalised to a rate.
  \label{compAllBKG4TeV_rates}}
\end{figure}

\begin{figure}
\begin{center}
  \includegraphics[width=0.42\textwidth]{figures/4TeV/reweighted/cv78_RadNAll.pdf}
  \includegraphics[width=0.42\textwidth]{figures/4TeV/reweighted/cv78_RadEnAll.pdf}
  \includegraphics[width=0.42\textwidth]{figures/4TeV/reweighted/cv78_RadNMuons.pdf}
  \includegraphics[width=0.42\textwidth]{figures/4TeV/reweighted/cv78_RadEnMuons.pdf}
  \includegraphics[width=0.42\textwidth]{figures/4TeV/reweighted/cv78_RadNProtons.pdf}
  \includegraphics[width=0.42\textwidth]{figures/4TeV/reweighted/cv78_RadEnProtons.pdf}
  \includegraphics[width=0.42\textwidth]{figures/4TeV/reweighted/cv78_RadNPhotons.pdf}
  \includegraphics[width=0.42\textwidth]{figures/4TeV/reweighted/cv78_RadEnPhotons.pdf}
\end{center}
\vspace{-0.6cm}
 \caption{comparison of all background sources at 4 TeV normalised to a rate.
  \label{compAllBKG4TeV_rates2}}
\end{figure}

\begin{figure}
\begin{center}
  \includegraphics[width=0.8\textwidth]{figures/cv87_allenergies_OrigZAll.pdf}
%  \includegraphics[width=0.8\textwidth]{figures/cv87_allenergies_OrigZMuon.pdf}
\end{center}
\vspace{-0.6cm}
 \caption{
  \label{fig:OrigZMuonAllEn2}} 
\end{figure}



\begin{figure}
\begin{center}
  \includegraphics[width=0.42\textwidth]{figures/6500GeV/reweighted/cv78_EkinAll.pdf}
  \includegraphics[width=0.42\textwidth]{figures/6500GeV/reweighted/cv78_PhiEnAll.pdf}
%  \includegraphics[width=0.42\textwidth]{figures/6500GeV/reweighted/cv78_EkinMuons.pdf}
 % \includegraphics[width=0.42\textwidth]{figures/6500GeV/reweighted/cv78_PhiEnMuons.pdf}
  \includegraphics[width=0.42\textwidth]{figures/6500GeV/reweighted/cv78_EkinProtons.pdf}
  \includegraphics[width=0.42\textwidth]{figures/6500GeV/reweighted/cv78_PhiEnProtons.pdf}
 \includegraphics[width=0.42\textwidth]{figures/6500GeV/reweighted/cv78_EkinPhotons.pdf}
 \includegraphics[width=0.42\textwidth]{figures/6500GeV/reweighted/cv78_PhiEnPhotons.pdf}
\end{center}
\vspace{-0.6cm}
 \caption{comparison of all background sources at 6.5~TeV.
  \label{compAllBKG_6.52}}
\end{figure}

\begin{figure}%[!htb]
\centering
\includegraphics[width=0.45\textwidth]{figures/compBGreweighted/ratioEkinAll.pdf}
%\includegraphics[width=0.45\textwidth]{figures/compBGreweighted/ratioEkinMuons.pdf}
\includegraphics[width=0.45\textwidth]{figures/compBGreweighted/ratioPhiEnAll.pdf}
%\includegraphics[width=0.45\textwidth]{figures/compBGreweighted/ratioPhiEnMuons.pdf}
\caption{Reweighted beam-gas distributions in the 2012 Run I and 2015 Run II scenario for all particles and muons showing the energy spectrum (top) and the azimuthal distribution (bottom).
  \label{fig:compBGreweighted12}}
\end{figure}




\begin{figure}%[!htb]
\centering
\includegraphics[width=0.45\textwidth]{figures/compBGreweighted/ratioRadNAll.pdf}
\includegraphics[width=0.45\textwidth]{figures/compBGreweighted/ratioRadNMuons.pdf}
\includegraphics[width=0.45\textwidth]{figures/compBGreweighted/ratioRadEnAll.pdf}
\includegraphics[width=0.45\textwidth]{figures/compBGreweighted/ratioRadEnMuons.pdf}
\caption{Reweighted beam-gas distributions in the 2012 Run I and 2015 Run II scenario for all particles and muons showing radial positions and energy in $r$.
  \label{fig:compBGreweighted2}}
\end{figure}



\begin{figure}
  \begin{center}
  \includegraphics[width=0.4\textwidth]{figures/compBHB1_4TeV_vs_6p5TeV/perTCThit/ratioEkinMuons.pdf}
  \includegraphics[width=0.4\textwidth]{figures/compBHB1_4TeV_vs_6p5TeV/perTCThit/ratioPhiEnMuons.pdf}
  \includegraphics[width=0.41\textwidth]{figures/compBHB2_4TeV_vs_6p5TeV/perTCThit/ratioEkinMuons.pdf}
  \includegraphics[width=0.41\textwidth]{figures/compBHB2_4TeV_vs_6p5TeV/perTCThit/ratioPhiEnMuons.pdf}
\end{center}
\vspace{-0.6cm}
 \caption{Similar to Fig.~\ref{fig:compBHrun1run2} but shown per TCT interactions for B1 (top) and B2 (bottom).
  \label{fig:compBHrun1run2PerTCT}}
\end{figure}



\begin{figure}%[!htb]
\begin{center}
  \includegraphics[width=0.49\textwidth]{figures/compBHB1_4TeV_vs_6p5TeV/normalised/ratioEkinAll.pdf}
  \includegraphics[width=0.49\textwidth]{figures/compBHB1_4TeV_vs_6p5TeV/normalised/ratioPhiEnAll.pdf}
  \includegraphics[width=0.49\textwidth]{figures/compBHB2_4TeV_vs_6p5TeV/normalised/ratioEkinAll.pdf}
  \includegraphics[width=0.49\textwidth]{figures/compBHB2_4TeV_vs_6p5TeV/normalised/ratioPhiEnAll.pdf}
\end{center}
\vspace{-0.6cm}
 \caption{Comparison of halo induced background at 4 and 6.5~TeV of properties of all particles at the interface plane for B1 (top) and B2 (bottom).
  \label{compBHrun1run22}}
\end{figure}



\begin{figure}
\begin{center}
  \includegraphics[width=0.42\textwidth]{figures/HLRunII/cv78_EkinAll.pdf}
  \includegraphics[width=0.42\textwidth]{figures/HLRunII/cv78_PhiEnAll.pdf}
%  \includegraphics[width=0.42\textwidth]{figures/HLRunII/cv78_EkinMuons.pdf}
%  \includegraphics[width=0.42\textwidth]{figures/HLRunII/cv78_PhiEnMuons.pdf}
  \includegraphics[width=0.42\textwidth]{figures/HLRunII/cv78_EkinProtons.pdf}
  \includegraphics[width=0.42\textwidth]{figures/HLRunII/cv78_PhiEnProtons.pdf}
  \includegraphics[width=0.42\textwidth]{figures/HLRunII/cv78_EkinPhotons.pdf}
  \includegraphics[width=0.42\textwidth]{figures/HLRunII/cv78_PhiEnPhotons.pdf}
\end{center}
\vspace{-0.6cm}
 \caption{Comparison of beam-gas (BG) in Run II and beam-halo in HL using the baseline layout (TCT5s in, \twosigmaret~settings) and round beam optics.
  \label{fig:hlrun22}}
\end{figure}



\begin{figure}
  \centering
    %\includegraphics[width=0.495\textwidth]{figures/XYNMuons_BG_4TeV_20MeV_bs.pdf}
    \includegraphics[width=0.495\textwidth]{figures/XYNMuons_BG_6500GeV_flat_20GeV_bs.pdf}  
    %\includegraphics[width=0.495\textwidth]{figures/XYNMuons_BH_4TeV_B1_20MeV.pdf}
    \includegraphics[width=0.495\textwidth]{figures/XYNMuons_BH_6500GeV_haloB1_20MeV.pdf}
  \caption{Spatial distribution of muons in an beam-gas (left) and beam-halo (right) scenario for a 6.5~TeV beam in Run II 2015. 
    \label{fig:XYNMuons2}}
\end{figure}


\clearpage
\begin{figure}
\begin{center}
  \includegraphics[width=0.492\textwidth]{figures/6500GeV/20MeV/RadNMuons_BG_6500GeV_flat_20GeV_bs.pdf}
  \includegraphics[width=0.492\textwidth]{figures/BH_run2/b1/RadNMuons_BH_6500GeV_haloB1_20MeV.pdf}      
\end{center}
\vspace{-0.6cm}
 \caption{Radii of muons of different energies.
  \label{fig:PhiEnMuComp}}
\end{figure}

\begin{figure}
\begin{center}
  \includegraphics[width=0.492\textwidth]{figures/HLRunII/cv78_EkinAll.pdf}
  \includegraphics[width=0.492\textwidth]{figures/HLRunII/cv78_PhiEnAll.pdf}
  \includegraphics[width=0.492\textwidth]{figures/HLRunII/cv78_RadNAll.pdf}
  \includegraphics[width=0.492\textwidth]{figures/HLRunII/cv78_RadEnAll.pdf}      
\end{center}
\vspace{-0.6cm}
 \caption{Comparison of properties of all particles in Run~II and HL.
  \label{fig:compHLRun2All}}
\end{figure}
