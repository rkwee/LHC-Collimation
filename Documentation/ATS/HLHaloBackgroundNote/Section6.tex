\newpage
\section{Evolution of background in the LHC and comparison to HL--LHC\label{evolut}}

We focus on the evolution of beam-gas and halo showers in IR1 for Run~I, II and HL-LHC scenario, highlighting muons as they are usually the most important particles causing background from the machine side. Other selected particle distributions can be found in the Appendix~\ref{evolutApp}.

\subsection{Comparison of background sources in 2012 and 2015}


We analyse properties of muons created in beam-gas and halo interactions. A comparison of the shape of the distributions at the interface plane, normalised per initial event, is made in Fig.~\ref{fig:compAllBKG_muons} for different background sources. They were simulated for Run~I at 4~TeV showing the energy spectrum of muons and the azimuthal distribution of their energy. One notices in both plots, that sources from the TCT impacts as betatron and off-momentum halo give rise to very similar distributions at the interface plane. The beam-gas spectrum of the left figure has a few distinct features, it rises more sharply with energy at low energies and then decreases between 10 GeV and 100 GeV, but for energies close to the beam energy, beam-gas events give again the highest muon contribution per event. The right part of Fig.~\ref{fig:compAllBKG_muons} exhibits another feature: Muons from beam-gas give highest contributions at $0$ and $\pm \pi$, but lowest in between, which means their energy is mostly horizontally distributed. Concering betatron halo, one can see that the contributions from B2 are higher than from B1 when normalised to a TCT interaction.\\

\begin{figure}%[htp]
\begin{center}
  \includegraphics[width=0.42\textwidth]{figures/4TeV/compAllBKG/EkinMuons.pdf}
  \includegraphics[width=0.42\textwidth]{figures/4TeV/compAllBKG/PhiEnMuons.pdf}
\end{center}
\vspace{-0.6cm}
 \caption{Comparison of all background sources at 4 TeV normalised per interaction showing energy spectrum and energy in $\phi$ for muons.
  \label{fig:compAllBKG_muons}}
\end{figure}
To visualise the spatial distribution of muons at the interface plane, we show them for beam-gas (left) and betatron halo on the TCTs (right) in Fig.~\ref{fig:XYNMuons} for the same scenario at 4~TeV. One can recognise a geometrical effect of the beginning of the ATLAS cavern, a rectangular shape with a higher particle flux inside, see also Fig.~\ref{flukaGeo_nominal} for tunnel, i.e. concrete part in the shown section is the same in LHC and HL-LHC. At $s \approx 28~m$, a concrete shielding starts blocking particles up to around $\pm 100~$cm in $x$ and $y$. The quadrupole of the triplet closest to the IP (MQXA or Q1) has a radius of $r \approx 46~$~cm (including tank and vacuum vessel) which is also visible in that figure. While muons in beam-gas collisions are created inside the vacuum tube and most of them stay inside, muons from halo showers are rather shielded by the vessel and higher rates can be found outside.


\begin{figure} %[htp]
  \centering
  \includegraphics[width=0.495\textwidth]{figures/XYNMuons_BG_4TeV_20MeV_bs.pdf}
  %\includegraphics[width=0.495\textwidth]{figures/XYNMuons_BG_6500GeV_flat_20GeV_bs.pdf}  
  \includegraphics[width=0.495\textwidth]{figures/XYNMuons_BH_4TeV_B1_20MeV.pdf}
  %\includegraphics[width=0.495\textwidth]{figures/XYNMuons_BH_6500GeV_haloB1_20MeV.pdf}
  \caption{Spatial distribution of muons of all energies in an beam-gas (left) and beam-halo (right) scenario for a 4~TeV beam in Run I 2012. At 6.5~TeV the distributions look similar, see Fig.~\ref{fig:XYNMuons2}.
    \label{fig:XYNMuons}}
\end{figure}


We normalise the results to rates towards ATLAS, but do not discuss off-momentum halo rates here as a prelimary data analysis of losses measured on primary collimators in IR7 and IR3~\cite{belenOffMom} indicate that IR3 losses are at least a factor 10~smaller if not 50 compared to losses in IR7 in 2015 and 2016, thus fully negligible as background source to ATLAS. Using Eq.~\ref{eqNormHalo}, the run parameteres as listed in Tab.~\ref{paramsRun12} and IR7 leakages from Tab.~\ref{tab:leakageFactorsIR7} we present the rates for 4~TeV in Fig.~\ref{compAllBKG_run12}. In all cases, beam-gas contributions are significantly higher than those from betatron halo. At 4~TeV, betatron halo contributions of both beams are below a percent-level for particle counts and around 2~\% for energy entering the interface plane. This increases for Run~II, where we observe double, 2~\%, halo-muons, but significantly more energy with 17~\% when comparing to only \textit{half} of beam-gas induced muons\footnote{Contributions from beam-gas have to be doubled when considering both sides.}. However, in absolute terms beam-gas rates are lower than in 2012, as mentioned due to the improved vacuum quality. The simulations also show that at 4~TeV, B2 produced twice as many beam-halo muons as B1, while in 2015 both contributions were almost equal. 

Beam-gas distributions for 2012 and 2015, normalised with the pressure to rates, are directly compared in Fig.~\ref{fig:compBGreweighted1}. Generally the vacuum stability overall improved producing less particles with lower energy streaming towards IP1 at 2015 runs' at 6.5~TeV. One clearly notices differences of the shapes. The energy spectrum of muons (left) suggests there were less high-energy muons in 2012 at 4~TeV than after restart: The ratio plot on the bottom shows that from around 20 to 150~GeV there were up to a factor 5 more muons at 4~TeV. In the right figure one can see that the energy difference carried by muons was reduced thereby almost factor 4. The shape is clearly different: while most of muons are horizontally distributed at 4~TeV, they show at 6.5~TeV now peaks in the vertical plane as well. This regular structure has emerged after applying the 2015 pressure profile and hints to muons inside the beampipe where magnetic fields act.
% --------------------------------------------------------------------------------------------

\begin{figure}
\begin{center}
  \includegraphics[width=0.49\textwidth]{figures/4TeV/reweighted/cv78_EkinMuons.pdf}
  \includegraphics[width=0.49\textwidth]{figures/4TeV/reweighted/cv78_PhiEnMuons.pdf}
  \includegraphics[width=0.49\textwidth]{figures/6500GeV/reweighted/cv78_EkinMuons.pdf}
  \includegraphics[width=0.49\textwidth]{figures/6500GeV/reweighted/cv78_PhiEnMuons.pdf}
\end{center}
\vspace{-0.6cm}
 \caption{Rate comparison of betatron halo and beam-gas induced muons at 4~TeV (top) and 6.5~TeV (bottom). The numbers in the legend indicate the fraction to the beam-gas data of the respective scenario. They are formed from the integral of the distributions.
  \label{compAllBKG_run12}}
\end{figure}

\begin{figure}%[!htb]
\centering

\includegraphics[width=0.45\textwidth]{figures/compBGreweighted/ratioEkinMuons.pdf}
\includegraphics[width=0.45\textwidth]{figures/compBGreweighted/ratioPhiEnMuons.pdf}
\caption{Distributions of muons from beam-gas events, normalised to rates using the pressure profile, in the 2012 Run I and 2015 Run II scenario for muons showing the energy spectrum (left) and the azimuthal distribution of the energy (right).
  \label{fig:compBGreweighted1}}
\end{figure}


\begin{figure}
\begin{center}
%  \includegraphics[width=0.8\textwidth]{figures/cv87_allenergies_OrigZAll.pdf}
  \includegraphics[width=0.8\textwidth]{figures/cv87_allenergies_OrigZMuon.pdf}
\end{center}
\vspace{-0.6cm}
 \caption{Evolution of muon rates for different run scenarios in 2011 at 3.5~TeV and 2012 at 4~TeV and one in 2015 at 6.5~TeV. Dedicated scrubbing periods already 2012 were held during machine recommissioning which had a positive effect on the vacuum quality. A similar evolution is visible for all particles as shown in Fig.~\ref{fig:OrigZAll}.
  \label{fig:OrigZMuon}} 
\end{figure}


The evolution of muon rates over different LHC runs early on of the LHC commissioning is illustrated in Fig.~\ref{fig:OrigZMuon}. One can see an improvement from 3.5~TeV towards 4~TeV due to dedicated ``scrubbing periods''~\cite{iadarolaEvian2012} to reduce the secondary electron yield by electron bombardment of the inner surfaces of the beam screen. Towards 6.5~TeV there has been a global improvement of the vacuum stability. It is clear from the pressure map in the bottom of Fig.~\ref{pressure2015} that inside the triplet the pressure is slightly higher in 2015. Very likely this is due to enhanced synchrotron radiation at 6.5~TeV. 

\begin{figure}
\centering
  \includegraphics[width=0.4\textwidth]{figures/compBHB1_4TeV_vs_6p5TeV/normalised/ratioEkinMuons.pdf}
  \includegraphics[width=0.4\textwidth]{figures/compBHB1_4TeV_vs_6p5TeV/normalised/ratioPhiEnMuons.pdf}
  \includegraphics[width=0.41\textwidth]{figures/compBHB2_4TeV_vs_6p5TeV/normalised/ratioEkinMuons.pdf}
  \includegraphics[width=0.41\textwidth]{figures/compBHB2_4TeV_vs_6p5TeV/normalised/ratioPhiEnMuons.pdf}
 \caption{Comparison of rates at the interface plane from betatron halo at 4 and 6.5~TeV showing the energy spectra and the azimuthal distributions of muons for B1 (top) and B2 (bottom). Both plots have the same y-scale, so one can easily see that the difference in B1 is higher.
  \label{fig:compBHrun1run2}}
\end{figure}

Betatron halo background is shown in Fig.~\ref{fig:compBHrun1run2} for B1 (top) and B2 (bottom). We notice from Tab.~\ref{tab:leakageFactorsIR7} that the leakage of B2 to IR1 went down from $2.4 \times 10^{-5}$ by 50~\% to $1.6 \times 10^{-5}$ due to the more open settings of nearly 5~$\sigma$ difference. However, significantly more bunches were present in Run~II which also causes the higher factor of rates at the interface plane as one can see from the ratio numbers. B1 in 2012 and 2015 is more different: With very similar leakage factors of $1.5 \times 10^{-5}$ and $1.2 \times 10^{-5}$ the overall increase is about a factor 3. Apart from the higher number of bunches, also the two different input distributions play a relevant role, see supporting figures in the Appendix Fig.~\ref{fig:compBHrun1run2PerTCT}. We also notice that the shapes in 2012 and 2015 look extremely similar on a log-scale for both, energy spectrum and azimuthal distribution. The ratio plots on the bottom directly show that with higher beam energy one can indeed expect a constant offset of particles, whereas the energy distribution highlights, more energetic particles are in the vertical plane (as can be expected due to the particles on the ideal orbit).

For betatron halo, one can conclude that there is no general tendancy of one beam or the other producing higher rates to the experimental area despite the higher leakages of B2 to IR1 for the investigated cases (Run~I 2012, Run~II 2015 and even in HL-LHC).

\subsection{Comparisons to background sources in HL--LHC}


We compare sources of background from Run II to those expected in HL-LHC. Direct comparisons of the two main sources are shown in Fig.~\ref{fig:HLR2Muons} for muons, with fractional numbers of each source w.r.t.~the estimated beam-gas in HL-LHC. We have chosen the optimistic case in HL-LHC after machine conditioning (a.~c.) from Ref.~\cite{ipac2015_rkh} and compare to the reweighted beam-gas from Run~II which is also on the lower limit of pressures as mentioned. Also we only display one beam, B1, as Fig.~\ref{compAllBKG_run12} already proves, betatron halo induced muons distributions were very similar independent of the beam. For the halo simulations we chose those the \twosigmaret~settings with TCT5s\footnote{The name has officially changed to TCT6 in a more recent version.} in which is the most likely to be deployed~\cite{layoutProcRod}.

Beam-gas background seems almost 6 times higher in Run~II than in HL-LHC. However, one has to consider the simulations were performed until 547~m while in HL-LHC it is only up to 140~m due to uncertain knowledge of the layout. This could add a significant part to HL-LHC beam-gas. The beam-gas shape of the azimuthal distribution in Run~II seems to persist in the HL-LHC layout with contributions in the horizontal and vertical plane.


Betatron halo induced background from Run~II is the lowest but one can already see in HL-LHC, the simulations predict that it could increase and be of about equal importance to beam-gas. Looking at the leakage of protons from IR7 to the TCTs, as simulationed with SixTrack, for various scenarios in HL-LHC in Tab.~\ref{leakageFactorsIR7} one notices they jump from Run~II to HL-LHC by one order of magnitude being in HL-LHC of the order $10^{-4}$ using the baseline, i.e.~\twosigmaret.~collimator settings. 

\begin{figure}
\begin{center}
  \includegraphics[width=0.42\textwidth]{figures/HLRunII/cv78_EkinMuons.pdf}
  \includegraphics[width=0.42\textwidth]{figures/HLRunII/cv78_PhiEnMuons.pdf}
\end{center}
\vspace{-0.6cm}
 \caption{Comparison of energy spectra of muons (left) and azimuthal energy distributions (right) for local beam-gas (BG) and betatron halo (Halo) in Run II and the basline HL-LHC (TCT5s in, \twosigmaret~settings, round beam optics) scenarios. Only Halo B2 is shown, as B1 looks very similar to B2. Characteristics of all particles are shown in Fig.~\ref{fig:compHLRun2All}.
  \label{fig:HLR2Muons}}
\end{figure}
