\newpage
\section{Evolution of Background in LHC and Comparison to HL--LHC\label{evolut}}

We focus on the evolution of beam-gas and halo showers in IR1 for the Run~I, II and HL scenario highlighting muons as they are usually the most important particles for measuring background from the machine side. Other selected particle distributions can be found in the App.~\ref{evolutApp}.

\subsection{Comparison of background sources in 2012 and 2015}


We analyse properties of muons created in beam-gas and halo interactions. A shape comparison is made in Fig.~\ref{fig:compAllBKG_muons} for different background sources simulated for the Run~I at 4~TeV showing the energy spectrum of muons and the azimuthal distribution of their energy. One notices in both plots, that sources from the TCT impacts as halo and offmomentum induced are very similar distributions at the interface plane. The beam-gas shape of the left figure has a few distinct features, it starts ealier at low energy and has one a dint at 10 to 100~GeV, but beam-gas overtakes the tctimpacts for the energies just as close as the beam energy. The other figure exhibits another feature: Muons from beam-gas give highest contributions at $0$ and $\pm \pi$, but lowest in between, which means their energy is mostly horizontally distributed.\\

\begin{figure}%[htp]
\begin{center}
  \includegraphics[width=0.42\textwidth]{figures/4TeV/compAllBKG/EkinMuons.pdf}
  \includegraphics[width=0.42\textwidth]{figures/4TeV/compAllBKG/PhiEnMuons.pdf}
\end{center}
\vspace{-0.6cm}
 \caption{Comparison of all background sources at 4 TeV normalised per interaction showing energy spectrum and energy in $\phi$ for muons.
  \label{fig:compAllBKG_muons}}
\end{figure}
To visualise the spatial distribution of muons, we show them for beam-gas (left) and halo on the TCTs (right) in Fig.~\ref{fig:XYNMuons} for the same scenario at 4~TeV. One can recognise a geometrical effect of the beginning of the ATLAS cavern, a rectangular shape with a higher particle flux inside. At $s \approx 28~m$, a concrete shielding starts blocking particles leaving air only up to $\pm 100~$cm in $x$ and $y$. The last quadrupole of the triplet before the IP (MQXA or Q1) has a radius of $r \approx 46~$~cm (including tank and vacuum vessel) which is also visible in that figure. While muons in beam-gas collisions are created inside the vacuum tube and most of them stay inside, muons from halo showers are rather shielded by the vessel and higher rates can be found outside.


\begin{figure} %[htp]
  \centering
  \includegraphics[width=0.495\textwidth]{figures/XYNMuons_BG_4TeV_20MeV_bs.pdf}
  %\includegraphics[width=0.495\textwidth]{figures/XYNMuons_BG_6500GeV_flat_20GeV_bs.pdf}  
  \includegraphics[width=0.495\textwidth]{figures/XYNMuons_BH_4TeV_B1_20MeV.pdf}
  %\includegraphics[width=0.495\textwidth]{figures/XYNMuons_BH_6500GeV_haloB1_20MeV.pdf}
  \caption{Spatial distribution of muons of all energies in an beam-gas (left) and beam-halo (right) scenario for a 4~TeV beam in Run I 2012. At 6.5~TeV the distributions look similar, see Fig.~\ref{fig:XYNMuons2}.
    \label{fig:XYNMuons}}
\end{figure}


We normalise the results to rates towards ATLAS, using Eq.~\ref{eqNormHalo}, the run parameteres as listed in Tab.~\ref{paramsRun12} and IR7 leakages from Tab.~\ref{leakageFactorsIR7} and present the rates for 4~TeV in Fig.~\ref{compAllBKG_run12}. In all cases, beam-gas contributions are significantly higher than from those from halos. At 4~TeV, halo contributions are below a percent-level for particles and around 2~\% for energy produced from what is produced in beam-gas. This increases by about an order of magnitude for Run~II, where halo-muons form 20~\% and 25~\% of beam-gas muons. However, in absolute terms beam-gas rates are lower than in 2012, as mentioned due to the improved vacuum stability. One can also read off that figure that at 4~TeV, B2 produced double the amount of B1, while in 2015 both contributions are almost equal.

Reweighted beam-gas distributions are directly compared in Fig.~\ref{fig:compBGreweighted1}. Generally the vacuum stability overall improved producing less particles with lower energy streaming towards IP1 at 2015 runs' at 6.5~TeV. One clearly notices differences of the shapes. The energy spectrum of muons (left) suggests there were less higher energy muons in 2012 at 4~TeV than after restart: The ratio plot on the bottom shows that from around 20 to 150~GeV there were up to a factor 5 more muons at 4~TeV. In the right figure one can see that the energy difference carried by muons was reduced thereby almost factor 4. The shape is clearly different: while most of muons are horizontally distributed at 4~TeV, they show at 6.5~TeV now peaks in the vertical plane as well. This regular structure has emerged after applying the 2015 pressure profile and hints to muons inside the beampipe where magnetic fields act.
% --------------------------------------------------------------------------------------------

\begin{figure}
\begin{center}
  \includegraphics[width=0.49\textwidth]{figures/4TeV/reweighted/cv78_EkinMuons.pdf}
  \includegraphics[width=0.49\textwidth]{figures/4TeV/reweighted/cv78_PhiEnMuons.pdf}
  \includegraphics[width=0.49\textwidth]{figures/6500GeV/reweighted/cv78_EkinMuons.pdf}
  \includegraphics[width=0.49\textwidth]{figures/6500GeV/reweighted/cv78_PhiEnMuons.pdf}
\end{center}
\vspace{-0.6cm}
 \caption{Rate comparison of halo and beam-gas at 4~TeV (top) and 6.5~TeV (bottom). The numbers in the legend indicate the fraction to the beam-gas data of the respective scenario. They are formed from the integral of the distributions.
  \label{compAllBKG_run12}}
\end{figure}

\begin{figure}%[!htb]
\centering

\includegraphics[width=0.45\textwidth]{figures/compBGreweighted/ratioEkinMuons.pdf}
\includegraphics[width=0.45\textwidth]{figures/compBGreweighted/ratioPhiEnMuons.pdf}
\caption{Reweighted beam-gas distributions in the 2012 Run I and 2015 Run II scenario for muons showing the energy spectrum (left) and the azimuthal distribution of the energy (right).
  \label{fig:compBGreweighted1}}
\end{figure}


\begin{figure}
\begin{center}
%  \includegraphics[width=0.8\textwidth]{figures/cv87_allenergies_OrigZAll.pdf}
  \includegraphics[width=0.8\textwidth]{figures/cv87_allenergies_OrigZMuon.pdf}
\end{center}
\vspace{-0.6cm}
 \caption{Evolution of muon rates for different run scenarios in 2011 at 3.5~TeV and 2012 at 4~TeV and one in 2015 at 6.5~TeV after LS1. Dedicated scrubbing periods were held during machine commissioning which had a positive effect on muon rates as they went down and allowed for 25~ns operation in Run II.
  \label{fig:OrigZMuon}} 
\end{figure}


The evolution of muon rates over different LHC runs early on of the LHC commissioning is illustrated in Fig.~\ref{fig:OrigZMuon}. One can see an improvement from 3.5~TeV towards 4~TeV due to dedicated ``scrubbing periods''~\cite{scrubbingreportGianni} to reduce the secondary electron yield by electron bombardment of the inner surfaces of the beam screen. Towards 6.5~TeV there has been a global improvement of the vacuum stability. It is clear from the pressure map in the bottom of Fig.~\ref{pressure2015} that inside the triplet the pressure is slightly higher in 2015. Very likely this is due to enhanced synchrotron radiation at 6.5~TeV. 

\begin{figure}
\centering
  \includegraphics[width=0.4\textwidth]{figures/compBHB1_4TeV_vs_6p5TeV/normalised/ratioEkinMuons.pdf}
  \includegraphics[width=0.4\textwidth]{figures/compBHB1_4TeV_vs_6p5TeV/normalised/ratioPhiEnMuons.pdf}
  \includegraphics[width=0.41\textwidth]{figures/compBHB2_4TeV_vs_6p5TeV/normalised/ratioEkinMuons.pdf}
  \includegraphics[width=0.41\textwidth]{figures/compBHB2_4TeV_vs_6p5TeV/normalised/ratioPhiEnMuons.pdf}
 \caption{Comparison of rates from tertiary halo at 4 and 6.5~TeV showing the energy spectra and the azimuthal distributions of muons for B1 (top) and B2 (bottom). Both plots have the same y-scale, so one can easily see that the difference in B1 is higher.
  \label{fig:compBHrun1run2}}
\end{figure}

Halo background is shown in Fig.~\ref{fig:compBHrun1run2} for B1 (top) and B2 (bottom). We notice from Tab.~\ref{leakageFactorsIR7} that the leakage of B2 to IR1 has gone down from $2.4 \times 10^{-5}$ by 50~\% to $1.6 \times 10^{-5}$ due to the more open settings of nearlly 4~$\sigma$ difference. However, almost the double number of bunches were present in Run~II which also cause a factor 2 higher rates at the interface plane as one can see from the ratio numbers. B1 in 2012 and 2015 is more different: With very similar leakage factors of $1.5 \times 10^{-5}$ and $1.2 \times 10^{-5}$ the overall increase is about a factor 4 and even slightly more in energy. Half of this increase can be explained by the number of bunches while the remaining factor is due to the two different input distributions, see supporting figures in the appendix Fig.~\ref{fig:compBHrun1run2PerTCT}. We also notice that the azimuthal shapes in 2012 and 2015 look extremely similar, at least for B1, but the ratio shows they are not the same likely due to the similar initial distributions in both cases, in Fig.~\ref{inel4TeV} and \ref{inel6.5}.

More distributions are shown in the appendix in Fig.~\ref{compBHrun1run22}.

We can conclude there is no general tendancy of one beam or the other producing higher leakages. 

\subsection{Comparisons to background sources in HL--LHC}


We compare background from Run II to those expected in HL-LHC. Direct comparisons of the two main sources are shown in Fig.~\ref{fig:HLR2Muons} for muons, with fractional numbers of each source w.r.t.~the beam-gas in HL. We have chosen the optimistic case in HL after machine conditioning (a.~c.) from Ref.~\cite{ipac2015_rkh} and compare to the reweighted beam-gas from Run~II which is also on the lower limit of pressures as mentioned. Also we only display one beam, B1, as Fig.~\ref{compAllBKG_run12} already proves, halo induced muons distributions were very similar independent of the beam. For the halo simulations we chose those the \twosigmaret~settings with TCT5s\footnote{The name has officially changed to TCT6 in a more recent version.} in which is the most likely to be deployed~\cite{layoutProcRod}.

Beam-gas background seems almost 6 times higher in Run~II than in HL. However, one has to consider the simulations were performed until 547~m while in HL it is only up to 140~m due to uncertain knowledge of the layout. This could add a significant part to HL beam-gas. Also note, there were other relatively less significant differences which make the comparisons to HL beam-gas only limited valid. The beam-gas shape of the azimuthal distribution in Run~II seems to persist in the HL layout with contributions in the horizontal and vertical plane.


Halo induced background from Run~II is the lowest but one can already see halo in HL becomes comparable to beam-gas in HL and even in Run~II. Looking at the leakages for various scenarios in HL in Tab.~\ref{leakageFactorsIR7} one notices they jump from Run~II to HL by one order of magnitude being in HL of the order $10^{-4}$. However, this is already the order of magnitude of IR7 leakages in 2016 with smaller $\beta^*$ optics to push the performance.


\begin{figure}
\begin{center}
  \includegraphics[width=0.42\textwidth]{figures/HLRunII/cv78_EkinMuons.pdf}
  \includegraphics[width=0.42\textwidth]{figures/HLRunII/cv78_PhiEnMuons.pdf}
\end{center}
\vspace{-0.6cm}
 \caption{Comparison of beam-gas (BG) in Run II and beam-halo in HL using the baseline layout (TCT5s in, \twosigmaret~settings) and round beam optics. Characteristics of all particles is shown in Fig.~\ref{fig:compHLRun2All}.
  \label{fig:HLR2Muons}}
\end{figure}
